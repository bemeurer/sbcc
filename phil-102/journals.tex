\documentclass[12pt,letterpaper]{article}
\usepackage[utf8]{inputenc}
\usepackage{setspace}
\usepackage[bottom]{footmisc}
\usepackage[colorinlistoftodos]{todonotes}
\usepackage[margin=1in]{geometry}
\usepackage[hidelinks]{hyperref}
\usepackage{dialogue}
\usepackage{graphicx,float,wrapfig}
\usepackage[notes,backend=biber]{biblatex-chicago}
\bibliography{journals}

\graphicspath{{./images}}

\title{Journals of Comparative World Religions}
\author{Bernardo Meurer}
\date{\today}

\begin{document}
\begin{spacing}{2}
    \maketitle
    \newpage
    \tableofcontents

    \newpage
    \section{My Religious Views}
        I was born in a farm in the south-eastern part of Brazil, in the state
        of Minas Gerais to be exact. Much like every other family in the state,
        mine was Roman Catholic. I was baptized, as is customary, at a very
        young age, and learned as a kid, from my grandmother, a handful of
        prayers for some different situations. Even though I was raised to be a
        catholic, I no longer identify as one today.

        I think I first started rejecting Catholicism at around the age of 11,
        when my father began insisting that I performed my First Communion.
        Doing so required me to take religion classes twice a week at the local
        church, something I was definitely not interested in. Time went by, I
        never actually performed the first communion, and so I also never
        performed my confirmation at a later age. While one can obviously still
        be a Catholic despite not passing through these processes I feel like
        this first encounter with the bureaucratic apparatus of the Church was
        the beginning of the end for me.

        My first questioning of God came at around the age of 12. I had learned
        from a teacher about Occam's Razor, and so it seemed to me that the
        assumption of God's existence was a fierce violation of that. I could
        not see how presuming the existence of an omnipotent being could be a
        simpler explanation than abstaining from tackling the unfalsifiable
        question of the provenance of existence. It seemed, to me, like a
        comfortable shortcut to some meaningful understanding of the universe as
        it is. At the time I had purchased Schopenhauer's ``The Horrors and
        Absurdities of Religion'', which was a short book of compiled oeuvres.
        From this book one sentence has stuck with me in perpetuity, ``in the
        eyes of the friend of the truth every fraud, however pious, is still a
        fraud''.

        From 12 to about 15 I didn't give the issue of Religion and spirituality
        much thought, it was just not something that was particularly
        interesting to me. When I was 15 my studies into Complexity Theory led
        me to the idea of an Undecidable Problem and to G\"odel's incompleteness
        theorem. I wasn't entirely sure how to concrete relate those to God, but
        I could see that the question of whether or not God existed was a
        decision problem (here in the technical meaning of the term) and that it
        seemed to fall under the undecidable category. I began to notice weak
        parallels between how people usually discussed and spoke of God and the
        recursive proof that G\"odel used to show incompleteness. Later on
        reading Descartes' work this became once again apparent.

        I currently identify as a Spinozist, as in Baruch Spinoza, which
        fundamentally means I believe that our definitions of God and of the
        universe should coincide. I absolutely do not believe in God as a
        conscious entity of any kind, and in many ways Spinozism is nothing but
        a methodical and analytic approach to agnosticism. It is the
        acknowledgment that God is in and of itself existence, and vice-versa.
        In Spinoza's words:

        \begin{description}
            \item[Proposition I] A substance is prior in nature to its affections.
            \item[Proposition II] Two substances having different attributes
                have nothing in common with one another. (In other words, if
                two substances differ in nature, then they have nothing in
                common).
            \item[Proposition III] If things have nothing in common with one
                another, one of them cannot be the cause of the other.
            \item[Proposition IV] Two or more distinct things are distinguished
                from one another, either by a difference in the attributes
                (i.e., the natures or essences) of the substances or by a
                difference in their affections (i.e., their accidental
                properties).
            \item[Proposition V] In nature, there cannot be two or more
                substances of the same nature or attribute.
            \item[Proposition VI] One substance cannot be produced by another
                substance.
            \item[Proposition VII] It pertains to the nature of a substance to
                exist.
            \item[Proposition VIII] Every substance is necessarily infinite.
            \item[Proposition IX] The more reality or being each thing has,
                the more attributes belong to it.
            \item[Proposition X] Each attribute of a substance must be
                conceived through itself.
            \item[Proposition XI] God, or a substance consisting of infinite
                attributes, each of which expresses eternal and infinite
                essence, necessarily exists. (The proof of this proposition
                consists simply in the classic ``ontological proof for God’s
                existence''. Spinoza writes that ``if you deny this, conceive,
                if you can, that God does not exist. Therefore, by axiom VII
                (`If a thing can be conceived as not existing, its essence
                does not involve existence'), his essence does not involve
                existence. But this, by proposition VII, is absurd. Therefore,
                God necessarily exists, Q.E.D.'')
            \item[Proposition XII] No attribute of a substance can be truly
                conceived from which it follows that the substance can be
                divided.
            \item[Proposition XIII] A substance which is absolutely infinite
                is indivisible.
            \item[Proposition XIV] Except God, no substance can be or be
                conceived.
        \end{description}

        The propositions in and of themselves do not constitute a proof of
        Spinoza's ultimate theory (one can read his \textit{Ethics} for that) but
    rather serve to clarify and outlay what I mean by it being an `analytical'
    approach to agnosticism. Finally, and again in Spinoza's words, ``Whatever
    is, is in God, and nothing can be or be conceived without God.''

    \newpage
    \section{Reflections on the Universal Mother}
    Before talking about the Universal Mother specifically, I'd like to talk a
    little about primal religions in general. It is at least mildly curious that
    we tend to call these religious `myths', it sounds like a way to lower their
    value as compared to being a true, institutionalized Religion. Curiously, I
    find that the opposite is true to me, it is exactly their lack of a unified
    apparatus that I find attractive. I deeply believe that the
    institutionalization of Religion is one of the worst things to have happened
    to humanity. Having an institution who operates on a power legitimized by
    themselves and that cannot be forfeited is a horrible idea, as we've seen
    over and over again in the pages of history.

    I am not implying, of course, that primal religions are non-institutional
    due to some conscious decision taken by their respective peoples; I think
    that mostly what kept them from acquiring an operational apparatus was the
    fact that those societies hadn't yet invented bureaucracy.

    Not surprisingly primal religions focus mostly on Nature and it's immediate
    effects on their lives. They are concerned about rain and drought, about
    their crops and livestock, the motions that our environment carries us
    through. I think it is nothing but human to, when faced with great loss
    induced by a natural cause, to attempt to rationalize it. Somehow the
    thought that misfortune might have chosen us randomly and we simply had a
    struck of bad luck is \emph{more} painful than to machine a logical cause
    for it.

    When the crops do fail to yield, then, we seek reason. It must be because
    \emph{I} failed to fulfill some duty, and am now being punished. The success
    of the crop is no longer an unnerving uncertainty bound to the entropy of
    natural behavior, but rather a deterministic side effect of our overall
    actions. Then, if I pray, worship, offer I shall have a good crop. If I
    devote myself to God then He shall reward me with the returns I desire. This
    is more comfortable than to live with the idea that our life hangs on a
    balance with no metric, and that God \emph{does} play dice. This concept,
    however, has always seemed to me somewhat petty or utilitarian; it seems
    like we are almost using God as wishing well to grant us desires. I have
    never been comfortable with the concept that God cares about us in
    particular, or that He would be bothered to inflect on faith based on our
    own actions.

    The Universal Mother is no different from the general outline of primal
    religions contained in the previous paragraph. As the Kagaba said She is
    ``the mother of all our seed, [\ldots] the mother of thunder, the mother of
    streams, [\ldots] the mother of the fruits of the earth, [\ldots] She alone
    is the mother of the fire and the Sun [\ldots] She is the mother of rain.''
    All of this is very much on point with my thoughts on myth as a tool of
    rationalizing Nature's inherent chaos.

    The utilitarian aspect of deity becomes even more apparent on the one
    recorded prayer on text, which says: ``Our mother of growing fields, our
    mother of streams, will [you] have pity on us? \ldots'' It is clear here how
    they position the Mother as the ruler of their fate, the ruler of their
    performance. It's perhaps slightly ironic that the great fear of Humans
    isn't the fact that they control their destiny, for they don't, but rather
    the crushing reality that we do not. And if we do not control the crops and
    their yield, then someone must, and thus She, the Mother, must be managed
    and adored in order to provide for us.

    These fears that I talk about are so deeply entrenched in Human nature that
    here I sit thousands of years later and yet I sometimes find myself wanting
    of someone in control of the odds of my life as well. In the words of
    Antonius Block, from The Seventh Seal, ``We carve an idol out of fear and
    call it God''. What else if not fear and uncertainty motivates us to find
    comfort within God.

    Another quote from Block that I find reflected in the text is his unknowing
    confession to Death himself, ``No man can live faced with Death, knowing
    everything is nothingness''. In the prayer we read ``For [to] whom do we
    belong? Whose seeds are we? To our mother alone do we belong.'' The first
    quote is a powerful scream into the impending void we face as humans; how
    can we deal with our own apparent insignificance and pointlessness? The
    second is a possible answer echoing from the void; we may deal with it by
    accepting our unity with Nature, by removing the virtual layer we've created
    that separates us from the eternal chaos of reality.

    \newpage
    \section{On Hinduism}
    I think the main aspect of Hinduism that I have a gripe with is the
    overwhelming focus on idols and idolatry. I really do not like all of this
    fantasy and anthropomorphism with God, it feels to me almost like a
    marketing strategy to attract the most of followers. It is, in some ways,
    similar to what the Catholic church would do with saints, creating them left
    and right in order to appease everyone they possibly could. God is a
    concept, and a unary one.

    Now, for my dislike of their general mythology I do think there is quite a
    lot of good knowledge in the Vedas, the passages contained within the
    slides, for example, are a small proof of that; ``Do not be led by others,
    awaken your own mind, amass your own experience, and decide for yourself
    your own path''. I think that advice in particular is relevant today where I
    find that most people, in one way or another, live by proxy, wasting away
    while they watch others live through their phones.

    Continuing with the Hindu texts, most of my thoughts on the Upanishads are
    expressed in the next section, but in general I find that the ones I read
    had a boringly mundane, anthropocentric, focus that takes away from the
    truly deep discussions surrounding God.

    It's this same note that I find repeated through Hindu texts and practices
    that somewhat annoy me; it's all far too focused on making God's philosophy
    accessible at all costs. There are stories with fantastical characters,
    there are meanings for all the colors, there are so many Gods one cannot
    keep track, there are multiple reincarnations, and so on. All of this buries
    the \emph{real}, complex, aspects of the concept of God under so much cruft
    and visual aid that it becomes almost pointless. I, for one, do not seek to
    study God to be entertained or engage with a story, I have lots of other
    books for that, I seek God for knowledge, and I'd rather it not be
    suffocated under the weight of so much allegory.

    The last thing I'd like to express about Hinduism is how obviously
    utilitarian and corrupt, and I mean utilitarian in the worst possible sense,
    it is. This becomes obvious the second one looks at the Caste system, which
    might just be the most incredibly evil and effective mechanism of forced
    segregation and social immobilization I have ever encountered. If God wants
    the Caste system, which as a statement in and of itself is flawed for God
    has no desires, then I defy and deny God, for He is nothing but evil.

    The Caste system operates on five different levels (or castes) that form a
    simple, direct hierarchy. On top of the power/holiness distribution are the
    Brahmins (priests), followed by the Kshatriyas (warriors and rulers), the
    Vaisyas (skilled traders, merchants), the Sudras (unskilled workers), and
    finally by the Harijans (outcasts, untouchables). While at first glance it
    might seem that the caste system simply reflects the natural hierarchy of
    society, which is mostly true, it's true evil lies in how caste movement
    works, or rather does not.

    When you are born, you are born in a Caste, usually attached to your family,
    and you stay within that Caste until you die and then perhaps reincarnate
    into a higher Caste. This not only takes away any possibility that through
    real effort one can improve their lives, for they are stuck in their low
    Caste, but does so in a perverse way that forces people to still comply so
    as to no reincarnate into an even lower Caste. It is simply an evil form of
    making sure the masses comply with the oppression of their rulers, and I
    think the most obvious indicator that the whole system is a sham is that
    \emph{of course} the priests are on top of the hierarchy.

    I could go on about what's wrong with the Caste system for days on end, but
    I think I have made my general thoughts on it rather clear: disgust. I have
    not much else to say about Hinduism, other that I have no interest in
    something that creates and endorses such a system.

    \newpage
    \section{Thoughts on \emph{Katha Upanishads}}
    The thing I find most interesting about the Katha Upanishads texts is the
    similarities in plot to Ingmar Bergman's \emph{The Seventh Seal}. In the
    text we are told the story of Nachiketas, a yogi who receives the blessing
    of three wishes from Death. Death grants his first two wishes without
    reprieve, but the third goes differently. ``When a man dies, this doubt
    arises: some say `he is' and some say `he is not'. Teach me the truth.''
    Asks Nachiketas. ``Even the gods had this doubt in times of old; for
    mysterious is the law of life and death. Ask for another boon. Release me
    from this''. Replied Death.

    This reminds me deeply of one of the last exchanges between Block and Death.
    \begin{dialogue}
        \speak{Death} When next we meet, the hour will strike for you and your friends.
        \speak{Block} And you will reveal your secrets?
		\speak{Death} I have no secrets.
		\speak{Block} So you know nothing?
		\speak{Death} I am unknowing.
    \end{dialogue}

	Ever since I first heard this dialogue it has struck me as one of the
	scariest and most unnerving thoughts possible. The idea that Death Himself
	does not know truly what awaits those whom he reaps is a deep stare into
	the void that is the Unknown, for if Death is unknowing, then who can claim
    otherwise? The thought that our ultimate destiny is unknown to all is truly
    terrifying.

    Katha Upanishads, however, takes a different turn than Bergman's movie; in
    the text Death does know what happens after the reaping. On the omnipresent
    theme of seeking mechanism of control over the seemingly random reality, in
    the story Death provides Nachiketas with the knowledge of immortality.

    If taken at face value, which is as I understand the Hindus take the story,
    I don't really find it interesting or productive. These attempts to take
    control of that which lies beyond us by definition always seem to me as a
    cheap way to avoid the true issue, which is our inherent unknowing.

    Another interpretation of the text, and one that I am personally more fond
    of, it to take it as an allegory. Here we take immortality not literally,
    but rather to mean that if one acts correctly and lives a wise, humble, good
    life than they will become immortal; not in and of themselves but through
    the memory and love from the world around them.

    In the text, the path is described as ``Only when Manas (mind) with
    thoughts and the five senses stand still, and when Buddhi (intellect,
    power to reason) does not waver, that they call the highest path. That is
    what one calls Yoga, the stillness of the senses, concentration of the mind,
    It is not thoughtless heedless sluggishness, Yoga is creation and
    dissolution''. In my reading this is the preaching of a life in abstention
    of the mundane and of deeply focused introspection.

    One section of the text I truly dislike is when Death says to Nachiketas
    ``This sacred knowledge is not attained by reasoning; but it can be given by
    a true Teacher''. For me the moment some knowledge cannot be attained by
    reasoning then it is not knowledge, but fraud and, to cite Schopenhauer
    again, a fraud, however pious, is still a fraud.

    Note that what I mean by `fraud' in the previous paragraph I don't
    mean to say that the statement is \emph{false}, that would also be a fraud,
    all I mean is that the statement is either unfalsifiable, or that the text
    does not truly provide backing or reason for it. This is part of why I so
    heavily dislike the ``This sacred knowledge is not attained by reasoning''
    passage; knowledge can only be attained by reason, and to say that some
    knowledge escapes that rule is to escape the burden of proof that should lie
    on all those who make true statements.

    \newpage
    \section{Reflections on Jainism and Sikhism}
    I am a deep believer in `productive' faith. What I mean by this is that I
    think faith should lead the disciple to not just internal betterment and
    happiness, but also to improving the world around him and focusing on making
    their lifetime impacting. I think that based on this parameter it should
    come as no surprise that I was not very keen on the ideals of Jainism.

    First of all I agree with the principle of no aggression; to a degree. I
    \emph{do} think that a hypothetical world without violence but just
    diplomacy would be better, but as I think is accepted, such world is not
    ours. I think, then, that Jainism in it's search for non-violence goes
    simply too far. For one to have to brush while one walks, and to wear a mask
    so as not to disturb the insects is simply counter productive. There is so
    much more good one could be doing with their time and effort than wasting
    away on these minimalisms.

    Now, the five ethical principles of Jainism are:
    \begin{enumerate}
        \item Nonviolence
        \item Truthfulness
        \item Non-stealing
        \item Chastity
        \item Non-possession, non-attachment from people, places, things
    \end{enumerate}

    While 1--3 seem to be very reasonable ethical principles, and overall
    good basis for a good society, thew remainder are quite the opposite. 4 is
    if nothing else a recipe for human extinction, if we are to expect all to be
    ethical, and 5 is the key ingredient in a society which doesn't advance, and
    doesn't improve. The curious thing is that element 5 is in many ways
    directly opposed to 1, since technology and it's comforts spare us so much
    pain and violence. These might be feasible or desirable principles for a
    monk, although I would question that as well, but they are most definitely
    not desirable values for society as a whole.

    For all the flaws that I see in Jainism, I see perfection in Sikhism. They,
    for one, affirm that God is beyond human comprehension, which is to me one
    of the few reasonable assumptions to be made about God. Secondly, they
    present a heavy focus on the individual path towards spirituality, ``there
    is One Truth, each must find their own path to it''. The idea that many
    other religious doctrines have that there is one true path to God and all
    must follow it equally is, to me, counter productive and leads to a general
    feeling of inadequacy (something which I felt deeply from Catholicism).

    Sikhs solve one of the main issues with Religion, and in particular with
    institutionalized religion, which is the idolatry, in the more literal sense
    of God. The conversion of God as a concept and goal into God as an idol of
    worship, usually also attached to the anthropomorphization of God, is the
    spark that lights religious wars. Idolatry makes the mind weak, it shifts
    the focus away from the things that truly matter withing the concept of God.

    In the same line, I find that the three guiding principles of Sikhism are
    almost as good as one could hope for, as opposed to the nonsensical five of
    Jainism. Namely, they are:
    \begin{enumerate}
        \item \emph{Naam Japna} --- Remember God at all times; be aware of
            yourself as if you were watching someone.
        \item \emph{Kirat Kana} --- Honest wealth; participate in life. To
            engage in life, work and earn a livelihood honestly is how we grow
            (rejection of asceticism).
        \item \emph{Vand Chakna} --- Sharing wealth, becoming a guardian of
            one's community.
    \end{enumerate}
    These principles yield the absolute opposite of what I criticized the
    Jainist ones for, they are good seeds for a working, productive, healthy
    society and life. The first principle reminds us to be mindful, and to never
    let the goal of spirituality off our minds. The second principle encourages
    an honorable life, a life of work, development and growth. The third
    principle reminds us that we are not islands, that we are always part of a
    community and that we should do out very best to improve it and perfect it,
    to help those within it without pride or reprieves.

    Finally, I think my overall liking of Sikhism can be well summed in this
    phrase: ``The primary subject of the Guru Granth Sahib is Truth, and how to
    become a person of truth. It enables men and women to lead meaningful lives
    while being members of a society''. This is exactly what I believe Religion
    should focus on, not petty idolatry but how we can find our meaning in life.

    \newpage
    \section{Meditations on Buddhism}
    I'd like to start my thoughts on Buddhism by focusing on the Eightfold Path,
    and primarily on what I believe to be it's fundamental flaw; the fact that
    it's entirely prescriptive and not \emph{descriptive}. What I mean by this
    is that while the Eightfold Path tells us \emph{what} to do, it does not
    illuminate us on \emph{how} to do it, which arguably is the most important
    piece of information.

    \textbf{Right View:} The first step states that one must have an
    understanding of the `problem'. The issue here is, I think, clear; often
    times it is precisely understanding the problem that is the most challenging
    step. I, for one, have wasted away effort attempting to solve a problem I
    did not have, or that I misunderstood, and here the Eightfold Path gives us
    no aid.

    \textbf{Right Intent:} This might be the one path that could be seen as
    \emph{both} descriptive and prescriptive, but sadly this is only true
    because it is so vacuous. It simply states one's need to focus on the
    process, which is a fair prescription to almost anything.

    \textbf{Right Speech:} Yet again the path tells us to be aware of what our
    speech reveals of our character, but provides no clues on how exactly should
    one be able to evaluate that. Without some form of metric or guidance this
    becomes nigh impossible; we can never truly see ourselves through the eyes
    of the other. As Lacan would perhaps put it, our perception of ourselves is
    itself an Alter Ego, bundled together with the other Alter Egos that
    encompass our perceptions of others.

    \textbf{Right Conduct:} Here, the Buddha gives what could generally be seen
    as good advice; to become aware of one's own motives and to move towards
    simplicity, but this is mostly tainted by, again, the lack of description.
    The real challenge here is how do we become aware of our motives, truly, and
    a solution to it is nowhere to be found in the Path. Moreover the
    ruling of chastity before marriage and of teetotalism are simply counter
    productive to a truthful, working society.

    \textbf{Right Livelihood:} This is the one path where the Buddha is truly
    descriptive; don't slave, don't prostitute yourself, and don't sell poison
    (drugs). If expanded, however, it is difficult to argue that most jobs don't
    oppress other beings in one way or another.

    \textbf{Right Effort:} Here the Buddha focuses on \emph{how} one must strive
    for enlightenment. Unfortunately, however, he is oblivious regarding how can
    one know when they are ``seeing things are they are''.

    \textbf{Right Mindfulness:} My biggest gripe here is with the idea that the
    mind is powerful enough that it can master all things. This is simply not
    true, in Mathematics we are often faced with concepts which the mind cannot
    master, which is why we have notation to aid us. The mind cannot master the
    general motion of 11-dimensional linear transformations on a convex
    manifold, for example.

    \textbf{Right Concentration:} This Path is actually very much descriptive,
    and for the most part it details the process of meditation. The one thing
    that is opaque here is the idea of how can one know they have mastered
    meditation and concentration; the parameters are simply too loosely defined
    and difficult to verify.

    Now, while I may question the usefulness of the Eightfold Path, I cannot
    disagree the Marks of existence. All three of its statements, that all
    formations are impermanent, subject to suffering, and without a self, are
    very precise, and I think a good guide on the nature of our own existence.

    %\printbibliography[heading=bibintoc,title={Bibliography}]
\end{spacing}
\end{document}
