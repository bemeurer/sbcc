\documentclass{article}
    %%%%%%%%%%%%%%%%%%%%%%%%%%%%%%%%%%%%%%%%%%%%%%%%%%%%%%%%%%%%%
    % Lecture Specific Information to Fill Out
    %%%%%%%%%%%%%%%%%%%%%%%%%%%%%%%%%%%%%%%%%%%%%%%%%%%%%%%%%%%%%
    \newcommand{\LectureTitle}{Lecture \#4 Notes}
    %\newcommand{\LectureDate}{\today}
    \newcommand{\LectureDate}{August\ 31,\ 2017}
    \newcommand{\LectureClassName}{ECON\ 102}
    \newcommand{\LatexerName}{Bernardo\ Meurer}
    %%%%%%%%%%%%%%%%%%%%%%%%%%%%%%%%%%%%%%%%%%%%%%%%%%%%%%%%%%%%%
    
    % Change "article" to "report" to get rid of page number on title page
    
    \usepackage[utf8]{inputenc}
    \usepackage{amsmath,amsfonts,amsthm,amssymb}
    \usepackage{setspace}
    \usepackage{Tabbing}
    \usepackage{fancyhdr}
    \usepackage{lastpage}
    \usepackage{extramarks}
    \usepackage{chngpage}
    \usepackage{todonotes}
    \usepackage{soul,color}
    \usepackage{graphicx,float,wrapfig}
    \usepackage{afterpage}
    \usepackage{abstract}
    
    % In case you need to adjust margins:
    \topmargin=-0.45in
    \evensidemargin=0in
    \oddsidemargin=0in
    \textwidth=6.5in
    \textheight=9.0in
    \headsep=0.25in
    
    % Setup the header and footer
    \pagestyle{fancy}
    \lhead{\LatexerName}
    \chead{\LectureClassName: \LectureTitle}
    \rhead{\LectureDate}
    \lfoot{\lastxmark}
    \cfoot{}
    \rfoot{Page\ \thepage\ of\ \pageref{LastPage}}
    \renewcommand\headrulewidth{0.4pt}
    \renewcommand\footrulewidth{0.4pt}
        
%%%%%%%%%%%%%%%%%%%%%%%%%%%%%%%%%%%%%%%%%%%%%%%%%%%%%%%%%%%%%%%%%%%%%%%%%%%
    
\begin{document}
\begin{spacing}{1.2}
	\newpage
	\todo{Add image of PPF}The PPF, at it's basic, shows the different
	combinations of two goods a society can produce. The capability
	of one to produce is determined by two factors: resources and
	technology. There are four types of resources (factors, inputs):
	\begin{itemize}
		\item Hours of Labor [L]
		\item Physical Capital [K]
		\item Human Capital [H]
		\item Natural Resources [N]
	\end{itemize}
	By technology it is meant a method, knowledge, of how to convert
	inputs into outputs. So, overall, a PPF represents the possible
	combinations of two goods that an economic agent can produce given
  resources and technology. Points outside of the frontier are not feasible, we
  lack resources and/or technology. Points inside the frontier are inefficient,
  and finally point on the frontier are feasible and efficient. \todo{Add graph}
  Economic growth is when the PPF moves ``outwards'', the converse is also true,
  when the frontier moves ``inwards'' it indicated that the economy is 
  shrinking. 

  Absolute Advantage is when an economic agent, in the production of a good, 
  uses the fewest resources.
  Comparative Advantage is when an economic agent, in the production of a good,
  if it can produce it at the \emph{lowest} opportunity cost.

  The absolute value of the slope will \emph{always} represent the opportunity
  cost on the x axis.
  \begin{equation}
    |\frac{dx}{dy}| \text{PPF} = OC_x
  \end{equation}
  \begin{equation}
    |\frac{dx}{dy}|^{-1} \text{PPF} = OC_y
  \end{equation}
  In trade, the cost for the products must fall \emph{between} the \(OC\) for
  both parties. \emph{Terms of trade}: are the range of prices between which
  trade is mutually beneficial. 
  \begin{equation}
    |\frac{dx}{dy}| A_{PPF} < T < |\frac{dx}{dy}| B_{PPF}
  \end{equation}
  Where A, B are agents and T is the terms of trade.
\end{spacing}
\end{document}