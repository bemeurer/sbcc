\documentclass{article}
    %%%%%%%%%%%%%%%%%%%%%%%%%%%%%%%%%%%%%%%%%%%%%%%%%%%%%%%%%%%%%
    % Lecture Specific Information to Fill Out
    %%%%%%%%%%%%%%%%%%%%%%%%%%%%%%%%%%%%%%%%%%%%%%%%%%%%%%%%%%%%%
    \newcommand{\LectureTitle}{Lecture \#6 Notes}
    %\newcommand{\LectureDate}{\today}
    \newcommand{\LectureDate}{September\ 11,\ 2017}
    \newcommand{\LectureClassName}{ART\ 103}
    \newcommand{\LatexerName}{Bernardo\ Meurer}
    %%%%%%%%%%%%%%%%%%%%%%%%%%%%%%%%%%%%%%%%%%%%%%%%%%%%%%%%%%%%%
    
    % Change "article" to "report" to get rid of page number on title page
    \usepackage{amsmath,amsfonts,amsthm,amssymb}
    \usepackage{setspace}
    \usepackage{Tabbing}
    \usepackage{fancyhdr}
    \usepackage{lastpage}
    \usepackage{extramarks}
    \usepackage{chngpage}
    \usepackage{soul,color}
    \usepackage{graphicx,float,wrapfig}
    \usepackage{afterpage}
    \usepackage{abstract}
    
    % In case you need to adjust margins:
    \topmargin=-0.45in
    \evensidemargin=0in
    \oddsidemargin=0in
    \textwidth=6.5in
    \textheight=9.0in
    \headsep=0.25in
    
    % Setup the header and footer
    \pagestyle{fancy}
    \lhead{\LatexerName}
    \chead{\LectureClassName: \LectureTitle}
    \rhead{\LectureDate}
    \lfoot{\lastxmark}
    \cfoot{}
    \rfoot{Page\ \thepage\ of\ \pageref{LastPage}}
    \renewcommand\headrulewidth{0.4pt}
    \renewcommand\footrulewidth{0.4pt}

        %%%%%%%%%%%%%%%%%%%%%%%%%%%%%%%%%%%%%%%%%%%%%%%%%%%%%%%%%%%%%
    % Some tools
    \newcommand{\enterTopicHeader}[1]{\nobreak\extramarks{#1}{#1 continued on next page\ldots}\nobreak%
    \nobreak\extramarks{#1 (continued)}{#1 continued on next page\ldots}\nobreak}
\newcommand{\exitTopicHeader}[1]{\nobreak\extramarks{#1 (continued)}{#1 continued on next page\ldots}\nobreak%
   \nobreak\extramarks{#1}{}\nobreak}

\newlength{\labelLength}
\newcommand{\labelAnswer}[2]
{\settowidth{\labelLength}{#1}
\addtolength{\labelLength}{0.25in}
\changetext{}{-\labelLength}{}{}{}
\noindent\fbox{\begin{minipage}[c]{\columnwidth}#2\end{minipage}}
\marginpar{\fbox{#1}}

% We put the blank space above in order to make sure this
% \marginpar gets correctly placed.
\changetext{}{+\labelLength}{}{}{}}

\setcounter{secnumdepth}{0}
\newcommand{\TopicName}{}
\newcounter{TopicCounter}
\newenvironment{Topic}[1][Problem \arabic{TopicCounter}]
{\stepcounter{TopicCounter}
\renewcommand{\TopicName}{#1}
\section{\TopicName}
\enterTopicHeader{\TopicName}}
{\exitTopicHeader{\TopicName}}

\setcounter{secnumdepth}{0}
\newcommand{\ExampleSectionName}{}
\newcounter{ExampleSectionCounter}[TopicCounter]
\newenvironment{ExampleSection}[1][Example \arabic{ExampleSectionCounter}]
{\stepcounter{ExampleSectionCounter}
\renewcommand{\ExampleSectionName}{#1}
\section{\ExampleSectionName}
\enterTopicHeader{\ExampleSectionName}}
{\exitTopicHeader{\ExampleSectionName}}

\setcounter{secnumdepth}{0}
\newcounter{ExampleBoxCounter}[TopicCounter]
\newcommand{\examplebox}[1]
{
% We put this space here to make sure we're disconnected from the previous
% passage
\stepcounter{ExampleBoxCounter}
\noindent\fbox{\begin{minipage}[c]{\columnwidth}#1\end{minipage}}\enterTopicHeader{\ExampleSectionName}\exitTopicHeader{\ExampleSectionName}\marginpar{\fbox{\#\arabic{ExampleBoxCounter}}}
% We put the blank space above in order to make sure this
% \marginpar gets correctly placed.
\vskip10pt%
}

\renewcommand{\contentsname}{{\normalsize Topics Covered}}
\renewcommand{\abstractname}{\LectureTitle\ Summary}
\renewcommand{\absnamepos}{flushleft}

%%%%%%%%%%%%%%%%%%%%%%%%%%%%%%%%%%%%%%%%%%%%%%%%%%%%%%%%%%%%%
    \begin{document}
    \begin{spacing}{1.2}
    \newpage
    \section{Egypt}
    \subsection{Narmer to Cleopatra}
    When we talk about Egypt, we have to talk about the idea of immortality, because it's a huge idea with the Egyptians. Immortality and death is an important theme in their architecture and art.
    Another pivotal point in egyptian culture is the Nile. It is important to remember the flooding behavior of the river, since the society operated around the flooding seasons.
    
    %\paragraph{Predynastic \& Early Dynastic Periods 3500 -- 2575 BCE}
    
    (3 -- 2) This palette, we believe, was used for mixing the coal and water 
    that the egyptians used around their eyes. It is believed that they painted 
    their eyes to reduce the reflection of light into their eyes. King Narmar 
    united lower and upper egypt, and that is represented in this palette by 
    the lions with uniting necks. Also, he is depicted wearing the crowns of 
    both lower and upper egypt. As in Mesopotamian art, the Egyptian artist's portrayal of Narmar combines both frontal and profile views.

    (3 -- 5) Imhotep was the builder to king Djoser, and he's the first 
    recorded name of an artist in the world. These are his stepped pyramids, 
    the idea of making pyramids came from the wish for something grander and 
    more durable. These are different than the ones we found in Mesopotamia, 
    they were meant to protect the deceased's tomb, and to allow for his \emph
    {ka}\footnote{Soul} to go in and out of the tomb. The pyramid was part of a 
    much larger complex (3 --6), and access to it was extremely limited.
    (3 -- 7) In these columns we can see the resemblance to the papyrus, both 
    in the fact that they have three sides, as well as their conical top. Also 
    important to notice is that these are \emph{engaged columns}, meaning they 
    are not freestanding, and instead are embedded into the wall.
    
    \subsection{The Old Kingdom}
    \begin{flushright}
        ca. 2500 BCE
    \end{flushright}
    (3 -- 8) Here, in the Great Pyramids of Gizeh, we see that the sides are 
    now ``smoothly'' triangular. The new tomb reflects the shape of the 
    influence of Heliopolis, the seat of the powerful cult of \emph{Re}, the 
    Sun God, his symbol was a pyramidal stone. The four sides of the pyramid, 
    like Djoser's pyramid, are oriented according to the four cardinal 
    directions. The Sun's rays were the ramps that they believed the sould used 
    to ascend to the heavens. There is a lot of debate on how they were built. 
    One of the common theories is that they used tree trunks to roll the blocks 
    from the nile, and then built ramps to raise the blocks. It is important to note that the pyramids are part of a Necropolis, involving tombs for family members and the Great Sphinx.

    (3 -- 11) The Great Sphinx was carved out of a static rock formation, which 
    leads us to believe there was a quarry in the neighborhood. The lion body 
    was associated with the Sun God as well.
    
    (3 -- 12) This is Khafre enthroned, it was carved out of Diorite which, as 
    we noted earlier in (2 -- 16), is an extremely hard material. We can 
    observe in Khafre's head the same sort of hat and beard as is present in 
    the Sphinx.

    (3 -- 13) We believe this to be Menkaure and Khamerernebty, although there 
    is a discussion whether the female figure is actually the divine mother of 
    all Kings. There are traces of pain left on this.

    (3 -- 14) Here we can see a sculpture of a seated scribe. We know this is \emph{not} someone of importance due to the fact that he's seating on the ground, he's not as idealized (has a tummy), looks older, and he has no crown, staff, or symbol of power.
\end{spacing}
    \end{document}