\documentclass{article}
    %%%%%%%%%%%%%%%%%%%%%%%%%%%%%%%%%%%%%%%%%%%%%%%%%%%%%%%%%%%%%
    % Lecture Specific Information to Fill Out
    %%%%%%%%%%%%%%%%%%%%%%%%%%%%%%%%%%%%%%%%%%%%%%%%%%%%%%%%%%%%%
    \newcommand{\LectureTitle}{Lecture \#7 Notes}
    %\newcommand{\LectureDate}{\today}
    \newcommand{\LectureDate}{September\ 18,\ 2017}
    \newcommand{\LectureClassName}{ART\ 103}
    \newcommand{\LatexerName}{Bernardo\ Meurer}
    %%%%%%%%%%%%%%%%%%%%%%%%%%%%%%%%%%%%%%%%%%%%%%%%%%%%%%%%%%%%%
    
    % Change "article" to "report" to get rid of page number on title page
    \usepackage{amsmath,amsfonts,amsthm,amssymb}
    \usepackage{setspace}
    \usepackage{Tabbing}
    \usepackage{fancyhdr}
    \usepackage{lastpage}
    \usepackage{extramarks}
    \usepackage{chngpage}
    \usepackage{soul,color}
    \usepackage{graphicx,float,wrapfig}
    \usepackage{afterpage}
    \usepackage{abstract}
    
    % In case you need to adjust margins:
    \topmargin=-0.45in
    \evensidemargin=0in
    \oddsidemargin=0in
    \textwidth=6.5in
    \textheight=9.0in
    \headsep=0.25in
    
    % Setup the header and footer
    \pagestyle{fancy}
    \lhead{\LatexerName}
    \chead{\LectureClassName: \LectureTitle}
    \rhead{\LectureDate}
    \lfoot{\lastxmark}
    \cfoot{}
    \rfoot{Page\ \thepage\ of\ \pageref{LastPage}}
    \renewcommand\headrulewidth{0.4pt}
    \renewcommand\footrulewidth{0.4pt}

        %%%%%%%%%%%%%%%%%%%%%%%%%%%%%%%%%%%%%%%%%%%%%%%%%%%%%%%%%%%%%
    % Some tools
    \newcommand{\enterTopicHeader}[1]{\nobreak\extramarks{#1}{#1 continued on next page\ldots}\nobreak%
    \nobreak\extramarks{#1 (continued)}{#1 continued on next page\ldots}\nobreak}
\newcommand{\exitTopicHeader}[1]{\nobreak\extramarks{#1 (continued)}{#1 continued on next page\ldots}\nobreak%
   \nobreak\extramarks{#1}{}\nobreak}

\newlength{\labelLength}
\newcommand{\labelAnswer}[2]
{\settowidth{\labelLength}{#1}
\addtolength{\labelLength}{0.25in}
\changetext{}{-\labelLength}{}{}{}
\noindent\fbox{\begin{minipage}[c]{\columnwidth}#2\end{minipage}}
\marginpar{\fbox{#1}}

% We put the blank space above in order to make sure this
% \marginpar gets correctly placed.
\changetext{}{+\labelLength}{}{}{}}

\setcounter{secnumdepth}{0}
\newcommand{\TopicName}{}
\newcounter{TopicCounter}
\newenvironment{Topic}[1][Problem \arabic{TopicCounter}]
{\stepcounter{TopicCounter}
\renewcommand{\TopicName}{#1}
\section{\TopicName}
\enterTopicHeader{\TopicName}}
{\exitTopicHeader{\TopicName}}

\setcounter{secnumdepth}{0}
\newcommand{\ExampleSectionName}{}
\newcounter{ExampleSectionCounter}[TopicCounter]
\newenvironment{ExampleSection}[1][Example \arabic{ExampleSectionCounter}]
{\stepcounter{ExampleSectionCounter}
\renewcommand{\ExampleSectionName}{#1}
\section{\ExampleSectionName}
\enterTopicHeader{\ExampleSectionName}}
{\exitTopicHeader{\ExampleSectionName}}

\setcounter{secnumdepth}{0}
\newcounter{ExampleBoxCounter}[TopicCounter]
\newcommand{\examplebox}[1]
{
% We put this space here to make sure we're disconnected from the previous
% passage
\stepcounter{ExampleBoxCounter}
\noindent\fbox{\begin{minipage}[c]{\columnwidth}#1\end{minipage}}\enterTopicHeader{\ExampleSectionName}\exitTopicHeader{\ExampleSectionName}\marginpar{\fbox{\#\arabic{ExampleBoxCounter}}}
% We put the blank space above in order to make sure this
% \marginpar gets correctly placed.
\vskip10pt%
}

\renewcommand{\contentsname}{{\normalsize Topics Covered}}
\renewcommand{\abstractname}{\LectureTitle\ Summary}
\renewcommand{\absnamepos}{flushleft}

%%%%%%%%%%%%%%%%%%%%%%%%%%%%%%%%%%%%%%%%%%%%%%%%%%%%%%%%%%%%%
    \begin{document}
    \begin{spacing}{1.2}
    \newpage
    \subsection{The Old Kingdom}

    (3 -- 15) The important thing to note here is the \emph{Strict canon}, which is the set of proportions they used to depict the Faraoh. Also once again we see the hierarchy of scale, with the Faraoh depicted much larger than others. They are hunting Hippos because they were known to eat the crops.

    \subsection{Middle Kingdom}

    During this period the egyptians begin to question the power of the Faraoh. 
    There's a general change in mentality during this period, although in art 
    they use most of the conventions from the Old Kingdom, however they try to 
    be more naturalistic. One of the images that shows that clearly is this 
    depiction of Senusret III

    (3 -- 17) Here, one can see clear signs on age, which is definitely a break 
    with the Old Kingdom standard of representing the Faraoh as always youthful.

    (3 -- 19) During this period they move their tombs into the cliff-sides, 
    they are rock-cut tombs as opposed to the large constructions of the Old 
    Kingdom. We believe the main reasons for this shit were tomb robberies, and 
    the easiness of construction. Another interesting thing here is the fluting 
    around the decorative pillars, some believe this was an attempt to imitate 
    reed columns, which have natural fluting, while other argue it was due to 
    the use of an \emph{adze} for carving. 

    \subsection{The New Kingdom}

    (3 -- 20) Here in the tomb of Hatshepsut we see an interesting mix between the Middle Kingdom and the Old Kingdom tombs, with part of the construction being rock-cut, while another part is was constructed outside the rock.

    
\end{spacing}
    \end{document}