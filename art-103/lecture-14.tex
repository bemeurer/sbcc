\documentclass{article}
    %%%%%%%%%%%%%%%%%%%%%%%%%%%%%%%%%%%%%%%%%%%%%%%%%%%%%%%%%%%%%
    % Lecture Specific Information to Fill Out
    %%%%%%%%%%%%%%%%%%%%%%%%%%%%%%%%%%%%%%%%%%%%%%%%%%%%%%%%%%%%%
    \newcommand{\LectureTitle}{Lecture \#14 Notes}
    %\newcommand{\LectureDate}{\today}
    \newcommand{\LectureDate}{October\ 16,\ 2017}
    \newcommand{\LectureClassName}{ART\ 103}
    \newcommand{\LatexerName}{Bernardo\ Meurer}
    %%%%%%%%%%%%%%%%%%%%%%%%%%%%%%%%%%%%%%%%%%%%%%%%%%%%%%%%%%%%%
    
    % Change "article" to "report" to get rid of page number on title page
    \usepackage{amsmath,amsfonts,amsthm,amssymb}
    \usepackage{setspace}
    \usepackage{Tabbing}
    \usepackage{fancyhdr}
    \usepackage{lastpage}
    \usepackage{extramarks}
    \usepackage{chngpage}
    \usepackage{soul,color}
    \usepackage{graphicx,float,wrapfig}
    \usepackage{afterpage}
    \usepackage{abstract}
    
    % In case you need to adjust margins:
    \topmargin=-0.45in
    \evensidemargin=0in
    \oddsidemargin=0in
    \textwidth=6.5in
    \textheight=9.0in
    \headsep=0.25in
    
    % Setup the header and footer
    \pagestyle{fancy}
    \lhead{\LatexerName}
    \chead{\LectureClassName: \LectureTitle}
    \rhead{\LectureDate}
    \lfoot{\lastxmark}
    \cfoot{}
    \rfoot{Page\ \thepage\ of\ \pageref{LastPage}}
    \renewcommand\headrulewidth{0.4pt}
    \renewcommand\footrulewidth{0.4pt}

        %%%%%%%%%%%%%%%%%%%%%%%%%%%%%%%%%%%%%%%%%%%%%%%%%%%%%%%%%%%%%
    % Some tools
    \newcommand{\enterTopicHeader}[1]{\nobreak\extramarks{#1}{#1 continued on next page\ldots}\nobreak%
    \nobreak\extramarks{#1 (continued)}{#1 continued on next page\ldots}\nobreak}
\newcommand{\exitTopicHeader}[1]{\nobreak\extramarks{#1 (continued)}{#1 continued on next page\ldots}\nobreak%
   \nobreak\extramarks{#1}{}\nobreak}

\newlength{\labelLength}
\newcommand{\labelAnswer}[2]
{\settowidth{\labelLength}{#1}
\addtolength{\labelLength}{0.25in}
\changetext{}{-\labelLength}{}{}{}
\noindent\fbox{\begin{minipage}[c]{\columnwidth}#2\end{minipage}}
\marginpar{\fbox{#1}}

% We put the blank space above in order to make sure this
% \marginpar gets correctly placed.
\changetext{}{+\labelLength}{}{}{}}

\setcounter{secnumdepth}{0}
\newcommand{\TopicName}{}
\newcounter{TopicCounter}
\newenvironment{Topic}[1][Problem \arabic{TopicCounter}]
{\stepcounter{TopicCounter}
\renewcommand{\TopicName}{#1}
\section{\TopicName}
\enterTopicHeader{\TopicName}}
{\exitTopicHeader{\TopicName}}

\setcounter{secnumdepth}{0}
\newcommand{\ExampleSectionName}{}
\newcounter{ExampleSectionCounter}[TopicCounter]
\newenvironment{ExampleSection}[1][Example \arabic{ExampleSectionCounter}]
{\stepcounter{ExampleSectionCounter}
\renewcommand{\ExampleSectionName}{#1}
\section{\ExampleSectionName}
\enterTopicHeader{\ExampleSectionName}}
{\exitTopicHeader{\ExampleSectionName}}

\setcounter{secnumdepth}{0}
\newcounter{ExampleBoxCounter}[TopicCounter]
\newcommand{\examplebox}[1]
{
% We put this space here to make sure we're disconnected from the previous
% passage
\stepcounter{ExampleBoxCounter}
\noindent\fbox{\begin{minipage}[c]{\columnwidth}#1\end{minipage}}\enterTopicHeader{\ExampleSectionName}\exitTopicHeader{\ExampleSectionName}\marginpar{\fbox{\#\arabic{ExampleBoxCounter}}}
% We put the blank space above in order to make sure this
% \marginpar gets correctly placed.
\vskip10pt%
}

\renewcommand{\contentsname}{{\normalsize Topics Covered}}
\renewcommand{\abstractname}{\LectureTitle\ Summary}
\renewcommand{\absnamepos}{flushleft}

%%%%%%%%%%%%%%%%%%%%%%%%%%%%%%%%%%%%%%%%%%%%%%%%%%%%%%%%%%%%%
\begin{document}
    \begin{spacing}{1.2}
    \newpage
        \section{The Etruscans}
        The Etruscans were the people who occupied the middle of Italy in the early Roman period.

        (6 -- 1) Here the interesting thing is the concept of a necropolis. The Etruscans had a very peculiar tomb culture, where they would basically create homes underground. Here we can see the Etruscans eating, and it is important to note that the Etruscan women could eat with them men.

        (6 -- 2) This is what we call a Fibula, one can see the orientalized lions in it which were typical of this time (650-640 BCE). The Etruscans were fantastic metal workers

        \subsection{Archaic Art and Architecture}
        \begin{flushright}
            600-480 BCE
        \end{flushright}
        The Etruscan temples were made of mud brick and wood, with columns and stairs. 

        (6 -- 3) We can see here clearly \emph{some} Greek influence on their temples. The ratios for the architecture were different from the greek, and they places a lot of statues on their roofs. The roofs were made of tiles, and they only had columns in the front. The temples were made of mud bricks and wood, meaning none remain today. 

        (6 -- 4) Here, even though the sculpture is Etruscan, we can see the Archaic smile that was typical of the greek. The drapery, however, is different, mostly due to how animated it is. The patterning of the hair,
        and the way the eyes are is also different than the archaic greek.

        (6 -- 6) Here we see a Sarcophagus made in terracotta. This is very Etruscan because here we see the man and the woman together. 

        (6 -- 7) Tumuli in a necropolis. The rock there is called Tufa, and is fairly easy to carve.
        
        (6 -- 8) Here we see an Etruscan tomb, and the curious thing is they were made to look like Etruscan homes. The living, however, would also make use of the tombs, unlike in other cultures, and celebrate there.

        (6 -- 11) This is a mural painting at Tarquinia, and we believe it represents everyday life.

        (6 -- 12) This is a depiction of Romulus and Remus, the founders of Rome, who were supposed to have been nursed by a wolf.

        (6 -- 13) This is the Chimaera of Arezzo. The idea here is that this Chimaera, which was a mythical creature composed of a lion's head, a goat's body and a serpent's tail, is going to be defeated. Here the body language of the Chimaera is extremely defensive.

        (6 -- 15) Here we see something that we did not see with the Greeks: an Arch.

        (6 -- 17) This is a sarcophagus. It's a very Etruscan motif that depicts demons bashing the deceased with hammers. 
    \end{spacing}
\end{document}