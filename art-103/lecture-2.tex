\documentclass{article}
    %%%%%%%%%%%%%%%%%%%%%%%%%%%%%%%%%%%%%%%%%%%%%%%%%%%%%%%%%%%%%
    % Lecture Specific Information to Fill Out
    %%%%%%%%%%%%%%%%%%%%%%%%%%%%%%%%%%%%%%%%%%%%%%%%%%%%%%%%%%%%%
    \newcommand{\LectureTitle}{Lecture \#2 Notes}
    %\newcommand{\LectureDate}{\today}
    \newcommand{\LectureDate}{August\ 28,\ 2017}
    \newcommand{\LectureClassName}{ART\ 103}
    \newcommand{\LatexerName}{Bernardo\ Meurer}
    %%%%%%%%%%%%%%%%%%%%%%%%%%%%%%%%%%%%%%%%%%%%%%%%%%%%%%%%%%%%%
    
    % Change "article" to "report" to get rid of page number on title page
    \usepackage{amsmath,amsfonts,amsthm,amssymb}
    \usepackage{setspace}
    \usepackage{Tabbing}
    \usepackage{fancyhdr}
    \usepackage{lastpage}
    \usepackage{extramarks}
    \usepackage{chngpage}
    \usepackage{soul,color}
    \usepackage{graphicx,float,wrapfig}
    \usepackage{afterpage}
    \usepackage{abstract}
    
    % In case you need to adjust margins:
    \topmargin=-0.45in
    \evensidemargin=0in
    \oddsidemargin=0in
    \textwidth=6.5in
    \textheight=9.0in
    \headsep=0.25in
    
    % Setup the header and footer
    \pagestyle{fancy}
    \lhead{\LatexerName}
    \chead{\LectureClassName: \LectureTitle}
    \rhead{\LectureDate}
    \lfoot{\lastxmark}
    \cfoot{}
    \rfoot{Page\ \thepage\ of\ \pageref{LastPage}}
    \renewcommand\headrulewidth{0.4pt}
    \renewcommand\footrulewidth{0.4pt}
    
    %%%%%%%%%%%%%%%%%%%%%%%%%%%%%%%%%%%%%%%%%%%%%%%%%%%%%%%%%%%%%
    % Some tools
    \newcommand{\enterTopicHeader}[1]{\nobreak\extramarks{#1}{#1 continued on next page\ldots}\nobreak%
                                        \nobreak\extramarks{#1 (continued)}{#1 continued on next page\ldots}\nobreak}
    \newcommand{\exitTopicHeader}[1]{\nobreak\extramarks{#1 (continued)}{#1 continued on next page\ldots}\nobreak%
                                       \nobreak\extramarks{#1}{}\nobreak}
    
    \newlength{\labelLength}
    \newcommand{\labelAnswer}[2]
      {\settowidth{\labelLength}{#1}
       \addtolength{\labelLength}{0.25in}
       \changetext{}{-\labelLength}{}{}{}
       \noindent\fbox{\begin{minipage}[c]{\columnwidth}#2\end{minipage}}
       \marginpar{\fbox{#1}}
    
       % We put the blank space above in order to make sure this
       % \marginpar gets correctly placed.
       \changetext{}{+\labelLength}{}{}{}}
    
    \setcounter{secnumdepth}{0}
    \newcommand{\TopicName}{}
    \newcounter{TopicCounter}
    \newenvironment{Topic}[1][Problem \arabic{TopicCounter}]
      {\stepcounter{TopicCounter}
       \renewcommand{\TopicName}{#1}
       \section{\TopicName}
       \enterTopicHeader{\TopicName}}
      {\exitTopicHeader{\TopicName}}
    
    \setcounter{secnumdepth}{0}
    \newcommand{\ExampleSectionName}{}
    \newcounter{ExampleSectionCounter}[TopicCounter]
    \newenvironment{ExampleSection}[1][Example \arabic{ExampleSectionCounter}]
      {\stepcounter{ExampleSectionCounter}
       \renewcommand{\ExampleSectionName}{#1}
       \section{\ExampleSectionName}
       \enterTopicHeader{\ExampleSectionName}}
      {\exitTopicHeader{\ExampleSectionName}}
    
    \setcounter{secnumdepth}{0}
    \newcounter{ExampleBoxCounter}[TopicCounter]
    \newcommand{\examplebox}[1]
      {
      % We put this space here to make sure we're disconnected from the previous
       % passage
       \stepcounter{ExampleBoxCounter}
       \noindent\fbox{\begin{minipage}[c]{\columnwidth}#1\end{minipage}}\enterTopicHeader{\ExampleSectionName}\exitTopicHeader{\ExampleSectionName}\marginpar{\fbox{\#\arabic{ExampleBoxCounter}}}
       % We put the blank space above in order to make sure this
       % \marginpar gets correctly placed.
       \vskip10pt%
       }
    
    \renewcommand{\contentsname}{{\normalsize Topics Covered}}
    \renewcommand{\abstractname}{\LectureTitle\ Summary}
    \renewcommand{\absnamepos}{flushleft}
    
    %%%%%%%%%%%%%%%%%%%%%%%%%%%%%%%%%%%%%%%%%%%%%%%%%%%%%%%%%%%%%
    
    \begin{document}
    \begin{spacing}{1.2}
    \newpage
    \section{Neolithic and Paleolithic Art}
    In Jordan we see the first human statues that are larger (Image 1--14).
    They are much larger than, for example, then Venus of Willendorf 
    (Image 1--4)

    In the constructions in Turkey, the houses had no doors or conventional 
    access paths. We believe they entered the houses and walked through the
    roofs. What's particularly interesting about these constructions is the
    ``shrines'' we found; houses with cow skulls and other artifacts.

    Also in Turkey (Catal Hoyuk) we start seeing the first images containing
    groups of people in activity together; in this case hunting deer. 
    (Image 1--15)

    Image 1--16 shows what we believe to be the very first \emph{landscape}.
    Landscape here means the depiction of a landscape in it's own right.

    Shifting to Western Europe, we don't find any civilizations as advanced
    as the ones in Catal Hoyuk, but starting ca. 2000 BCE we begin finding
    ``Henges'', large, \emph{megalithic}, stone constructions. Newgrange is
    one of the few stone constructions that is actually preserve; and it is
    huge. In the main gallery of the Newgrange burial chamber (Image 1--17)
    the ceiling is in a spiral ``shape'', and in fact spiral shapes are
    found throughout the whole construction.
    
    One of the biggest neolithic finds that we still look at today is the
    Stonehenge (Image 1--20). There have been many attempts at understanding
    the use of it, we believe it had some sort of astronomical usage, likely 
    in relation to the summer solstice. It is not known how the neolithic
    Britons built the Stonehenge, how they moved the huge stones. 
    
    (Image 1--1A) Is this art? This is a water-worn pebble, that happens to 
    resemble a human face. Where do we draw the line? This is one of the first
    records we have of a human recognizing itself in something else, in this
    case a pebble.

    (Image 1--6) We find this to be one of the first shows of \emph{intent},
    someone explicitly \emph{wanted} to make this work of art and represent
    these Bisons. 

    (Image 1--9) From the Paleolithic period, and found in the Pech-Merle 
    cave in France. The hand markings are a \emph{negative}, we believe that
    they spit around their hands. It is also important to notice that the work
    of ark integrates with the natural shape of the rock, the head of the horse
    follows the surface that was available.

    (Image 1--14) The distinctive factor on this statue is it's size, much 
    larger than other humanoid statues. The construction method is also much
    more advanced, there is a straw core that was plastered, shaped, and
    painted. The eyes were also added, and are detailed. This statue is also
    ungendered (unlike the Venus of Willendorf, which is clearly a female).

    (Image 1--15) The important characteristic here is that there is a
    \emph{group} of people doing an activity together, in this case hunting.
    They seem to be working in a coherent group around one or more creatures.
    This is the first record we have of something of this kind. The painting is
    in \emph{composite view}, meaning a mixture between frontal and profile views.
    
    (Image 1--16) This is the first depiction of a Landscape in and of itself.
    There is no story being told, this is just the pure landscape, there is at
    most just a contextual information on the landscape. In the backgroud there
    is what we believe to be a volcanic eruption. The meaningfulness of this lies
    in that it requires a certain mindset, which the neolithic man had achieved, 
    to have the \emph{will} to represent the larger picture; the landscape.
    \end{spacing}
    \end{document}