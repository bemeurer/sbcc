\documentclass[12pt,letterpaper]{article}
    \usepackage[utf8]{inputenc}
    \usepackage{ifpdf}
    \usepackage{mla}
    \usepackage{todonotes}
    \usepackage[hidelinks]{hyperref}

    \urlstyle{same}

    \begin{document}
    \begin{mla}{Bernardo}{Meurer}{Professor Martin}{English 110}{November 21, 2017}%
        {On Sherman Alexie's Dichotomic Indian }
        \noindent\textbf{Introduction}\\
        Throughout Sherman Alexie's ``Ten Little Indians'' we see a deeply divided depiction of the modern American Indian. While on one hand stories such as ``The Search Engine'' depict a modern Indian, a person like any other, without any mystical aspirations or ritualistic oddities, on the other, we have passages such as ``What You Pawn I Will Redeem'', that show a deeply magical Indian, truly different from the ``white folk'', imbued in tradition. Alexie, in his work, is inconsistent with regards to the identity of the Native American in modern society, leading to a contradictory and confusing overall aspect of the book.

        \noindent\textbf{P1: The Search Engine --- An analysis}\\
        We begin by analyzing the first short story of the book, ``The Search Engine.'' In it, we are immersed into the story of Corliss, a young Spokane indian studying at Washington State University on a quest to find a supposedly Spokane author whom she discovered from a book she randomly encountered in the library. At a certain point in the story the narrator tells us: ``White people, no matter how smart, were too romantic about Indians. White people looked at the Grand Canyon, Niagara Falls, the full moon, newborn babies, and Indians with the same goofy sentimentalism. Being a smart Indian, Corliss had always taken advantage of this romanticism \ldots'' Here we see very clearly Alexie's ``modern indian'' someone who rejects faux-mysticism implied upon Native Americans, and in fact makes use of it for their own benefit. This, however, comes in stark contrast with the idea of Indian Alexie himself brings forth in other stories, such as ``Do Not Go Gentle,'' or ``What You Pawn I Will Redeem.''

        \noindent\textbf{What You Pawn I Will Redeem --- The mystical}\\
        In the eighth, and penultimate, story of the book ``What You Pawn I Will Redeem,'' we watch the (short) journey of a man trying to recover his grandmother's powwow regalia, once he finds it displayed in a pawn shop. Here, as the story progresses, we are met with a strikingly different concept of Indian than the one given in the first story of the book. Unlike Corliss, Jackson is a deeply mystical person, who believes in all of his Spokane lore, and feels as if his own magical connection to his ancestors were the cause to his ultimate failure in life, he reminisces the death of his grandmother as a turning point for worse in his life. This all culminates, and is made explicit, towards the end of the tale when he, finally, is reunited with his rightful regalia and, as Alexie tells, ``I took my grandmother’s regalia and walked outside. I knew that solitary yellow bead was part of me. I knew I was that yellow bead in part. Outside, I wrapped myself in my grandmother’s regalia and breathed her in. I stepped off the sidewalk and into the intersection. Pedestrians stopped. Cars stopped. The city stopped. They all watched me dance with my grandmother. I was my grandmother, dancing.'' Here we see the dramatic break with the first story very clearly, Jackon is deeply submerged into magic and mysticism, and treats it as if it were part of reality. It is obvious, and I think that no one would claim otherwise, that the death of a loved one, despite being traumatic, should not destroy a (healthy person's) life, as happened to Jackson.

        \noindent\textbf{The Contrast}

        \noindent\textbf{White Culpability --- The Native American Scapegoat}

        \noindent\textbf{Conclusion --- Alexie is inconsistent and opportunistic}
    \end{mla}
    \end{document}