\documentclass[12pt,letterpaper]{article}
    \usepackage[utf8]{inputenc}
     \usepackage{ifpdf}
    \usepackage{mla}
    \usepackage{todonotes}
    
    
        \begin{document}
        \begin{mla}{Bernardo}{Meurer}{Professor Martin}{English}{October 5, 2017}%
            {Essay \#2 Outline}
        \noindent \textbf{Thesis:}
        
            DRM is an instrument to protect Copyright, and therefore it should be implemented in such way that it respects the rights on users and owners alike, otherwise it should not be put in place at all.
        
        \noindent \textbf{Paragraph 1:}

            \begin{itemize}
                \item Define copyright
                \begin{itemize}
                    \item Maybe use Locke? More research is needed on the best source for a solid definition of copyright.
                    \item Hegel is a good candidate (Philosophy of Right), however the edition available on the Library is archaic and poorly commented, my edition arrives Thursday October 5.
                \end{itemize}
                \item Define DRM
                \item Correlate DRM and Copyright
                \item Problematize
                \begin{itemize}
                    \item Show how DRM damages users
                    \item Argue that it has become ``less about copyright and more about protecting profit''
                \end{itemize}

            \end{itemize}

        \noindent \textbf{Paragraph 2:}

            \begin{itemize}
                \item Focus on the interweaving of Lockean philosophy and property
                \item Show that intellectual property is no different from material property
                \item  Make use of Wendy Gordon's work (see Annotated Bibliography) for a good reference.
                \item Argue for Copyright's duty towards users and owners alike
                \begin{itemize}
                    \item Locke's ``Two Treatises of Government'' is supposedly the reference on the matter. My Cambridge edition with comments should arrive Tuesday, October 10.
                    \item Point out the misconception that copyright is ``against users'' 
                \end{itemize}
            \end{itemize}

        \noindent \textbf{Paragraph 3:}

            \begin{itemize}
                \item Deepen the reasoning behind why copyright exists
                \item Hegelian Natural light
                \item Economic factor
                \item Social factor
                \begin{itemize}
                    \item Focus on the democracy-enhancing characteristics of copyright
                \end{itemize}
                \item Educational factor
                \begin{itemize}
                    \item See Britain's 1710 ``Statute of Anne'' legislation
                \end{itemize}
                \item Marcella Favale and Wendy Gordon's works will be crucial here.
                \item Further investigation of Hegel's Philosophy of Right is needed. In particular his arguments for intellectual property relating to a creator's personality.
            \end{itemize}
        
        \noindent \textbf{Paragraph 4:}

            \begin{itemize}
                \item Having had copyright and DRM well established and connected, now it is time to show how DRM goes against copyright, and not in defense of it as intended.
                \item See Annotated Bibliography for sources containing examples of this
                \item The general flow of the argumentation should go
                \begin{enumerate}
                    \item Define Copyright and DRM
                    \item Argue in favor of Copyright
                    \item More arguments for Copyright
                    \item Show how DRM goes against Copyright
                    \item Conclude against DRM. 
                \end{enumerate}
            \end{itemize}
        

        \end{mla}
        \end{document}