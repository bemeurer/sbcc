\documentclass[12pt,letterpaper]{article}
    \usepackage[utf8]{inputenc}
    \usepackage{ifpdf}
    \usepackage{mla}
    \usepackage{todonotes}    
    \begin{document}
    \begin{mla}{Bernardo}{Meurer}{Professor Martin}{English 110}{October 24, 2017}{Reflections on Essay \#1}
        On the revised draft most of the improvements went into the length and depth of analysis of the paragraphs. Being a non-native english writer, some of the aspects of an academic argumentative essay were unknown to me. In my native language, Portuguese, paragraphs are expected to contain one single idea, and therefore you are expected to aim at a length of no more than 8-10 lines per paragraph. The concept is that if you need more than that, then probably your idea can be distilled into simpler concepts, and each should have it's individual paragraph. This caused me to feel almost guilty when writing longer paragraphs, even though this is perfectly fine in English and in fact much preferred. Writing the essay, and subsequently reviewing it, taught me that in English the message is king, and if it takes you a monolithic page to express a concept then so be it. Re-reading the essay, I was happy to see I managed to write over a thousand words on a topic that I was supremely disinterested in without using nonsense to fill the gaps. While this may seem like a non-point, the truth is often times one must write about what needs to be written about, despite how interesting or not that may be, and knowing how to cope with that is a useful skill. In conclusion, the revision process was helpful because it made me realise which concepts from my native writing translate into English, and which don't, and that I will most definitely carry onto my future writings.
    
    \end{mla}
    \end{document}