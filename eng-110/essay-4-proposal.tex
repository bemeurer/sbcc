\documentclass{article}

    \usepackage[utf8]{inputenc}
    \usepackage{fancyhdr}
    \usepackage{extramarks}
    \usepackage{amsmath}
    \usepackage{amsthm}
    \usepackage{amsfonts}
    \usepackage{enumitem}
    \usepackage{tikz}
    \usepackage[plain]{algorithm}
    \usepackage{algpseudocode}

    \usetikzlibrary{automata,positioning}

    %
    % Basic Document Settings
    %

    \topmargin=-0.45in
    \evensidemargin=0in
    \oddsidemargin=0in
    \textwidth=6.5in
    \textheight=9.0in
    \headsep=0.25in

    \linespread{1.1}

    \pagestyle{fancy}
    \lhead{\hmwkAuthorName}
    \chead{\hmwkClass\ (\hmwkClassInstructor): \hmwkTitle}
    \rhead{\firstxmark}
    \lfoot{\lastxmark}
    \cfoot{\thepage}

    \renewcommand\headrulewidth{0.4pt}
    \renewcommand\footrulewidth{0.4pt}

    \setlength\parindent{0pt}

    %
    % Create Problem Sections
    %

    \newcommand{\enterProblemHeader}[1]{
        \nobreak\extramarks{}{Problem \arabic{#1} continued on next page\ldots}\nobreak{}
        \nobreak\extramarks{Problem \arabic{#1} (continued)}{Problem \arabic{#1} continued on next page\ldots}\nobreak{}
    }

    \newcommand{\exitProblemHeader}[1]{
        \nobreak\extramarks{Problem \arabic{#1} (continued)}{Problem \arabic{#1} continued on next page\ldots}\nobreak{}
        \stepcounter{#1}
        \nobreak\extramarks{Problem \arabic{#1}}{}\nobreak{}
    }

    \setcounter{secnumdepth}{0}
    \newcounter{partCounter}
    \newcounter{homeworkProblemCounter}
    \setcounter{homeworkProblemCounter}{1}
    \nobreak\extramarks{Problem \arabic{homeworkProblemCounter}}{}\nobreak{}

    %
    % Homework Problem Environment
    %
    % This environment takes an optional argument. When given, it will adjust the
    % problem counter. This is useful for when the problems given for your
    % assignment aren't sequential. See the last 3 problems of this template for an
    % example.
    %
    \newenvironment{homeworkProblem}[1][-1]{
        \ifnum#1>0
            \setcounter{homeworkProblemCounter}{#1}
        \fi
        \section{Problem \arabic{homeworkProblemCounter}}
        \setcounter{partCounter}{1}
        \enterProblemHeader{homeworkProblemCounter}
    }{
        \exitProblemHeader{homeworkProblemCounter}
    }

    %
    % Homework Details
    %   - Title
    %   - Due date
    %   - Class
    %   - Section/Time
    %   - Instructor
    %   - Author
    %

    \newcommand{\hmwkTitle}{Essay 4 Proposal}
    \newcommand{\hmwkDueDate}{November 16, 2017}
    \newcommand{\hmwkClass}{English 110}
    \newcommand{\hmwkClassTime}{}
    \newcommand{\hmwkClassInstructor}{Professor Joanne Martin}
    \newcommand{\hmwkAuthorName}{\textbf{Bernardo Meurer}}

    %
    % Title Page
    %

    \title{
        \vspace{2in}
        \textmd{\textbf{\hmwkClass:\ \hmwkTitle}}\\
        \normalsize\vspace{0.1in}\small{Due\ on\ \hmwkDueDate\ at 3:55pm}\\
        \vspace{0.1in}\large{\textit{\hmwkClassInstructor\ \hmwkClassTime}}
        \vspace{3in}
    }

    \author{\hmwkAuthorName}
    \date{}

    \renewcommand{\part}[1]{\textbf{\large Part \Alph{partCounter}}\stepcounter{partCounter}\\}

\begin{document}
\maketitle
\pagebreak
\begin{homeworkProblem}
    \textbf{What is the specific topic/idea you want to write about?}

    4. Analyze how past oppression of Native Americans affects Alexie’s contemporary Indian characters. In other words, how does past colonialism and genocide influence the way the character(s) you chose see themselves and their place in the world?
\end{homeworkProblem}
\begin{homeworkProblem}
    \textbf{What is your main interpretation and analysis about this topic/idea?}

    There is some sort of placebo effect in which because they think they have to be affected and depressed by what happened to their ancestors, they are. I don't think the ``oppression of the white man'' on native americans should have the impact it has on them. As someone who comes from a part German, part Jewish, part Lebanese family, I feel like I ought to be able to identify with the crippling memory of past genocide, and yet I don't. I think Alexie's contemporary Indian characters are weak, fantasy images of what a real modern Indian is. I also hate all the ``indian mysticism'' in the story, as if they were somehow magical and different. They are just people, every bit as mundane and mediocre as anyone else (me included).
\end{homeworkProblem}
\begin{homeworkProblem}
    \textbf{What specific sections/incidents/ideas from out text, Ten Little Indians will you focus your analysis on in order to provide evidence and support for your essay idea? (Provide specific details to answer this question, which should provide a good foundation to build an essay on).}

    Not sure yet.
\end{homeworkProblem}
\begin{homeworkProblem}
    \textbf{What is your ‘working thesis’?}

    ``Alexie's characters seem to blame their disgraced life and communities on the genocide of Indians. Although horrible, and without question reprehensible, to claim that the modern underperformance of any Indian is due to killings that took place well over a hundred years ago, is bogus.''

    Thoughts:

    Think if a Jewish man today claimed to be an alcoholic due to the Holocaust, I don't think most people would think to themselves ``of course, oh poor thing.'' I think the vast majority of people would not buy it, but somehow Native Americans seem to get special treatment.
\end{homeworkProblem}
\begin{homeworkProblem}
    \textbf{What will what your essay need to discuss/accomplish in order to support this thesis?}

    I need to categorically prove that the melancholy displayed by the characters in the stories comes from a flaw of character, and a lack of concern with their own success. I need to show that failure is only the fault of oneself, and that victimism is a comfortable lie.
\end{homeworkProblem}


\end{document}