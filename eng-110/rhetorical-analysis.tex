\documentclass[12pt,letterpaper]{article}
    \usepackage[utf8]{inputenc}
    \usepackage{ifpdf}
    \usepackage{mla}
    \usepackage{todonotes}
    \usepackage[hidelinks]{hyperref}

    \urlstyle{same}

    \begin{document}
    \begin{mla}{Bernardo}{Meurer}{Professor Martin}{English 110}{December 7, 2017}%
        {Richard Stallman: Free Software, Free Society}
        On the socio-political sphere of the world Freedom has been largely agreed upon as being not only good, but a necessity. This principle, that one must be truly and ultimately free, however, has not yet fully propagated into the digital space. Richard Stallman, a Harvard graduate, is the founder and main spokesperson for the Free Software Movement, and in ``Free Software, Free Society'' he talks about how what Free Software is, why it is important, and how one can help liberate themselves and others. Throughout the argument Stallman uses mainly Logos and Pathos, while delegating the Ethos optic of his argument to his own credentials.

        The Logos aspect of Stallman's speech is solid, and it relies on the fundamentally agreed upon idea that Freedom is good and necessary. Logos, being the use of pure or categorical logic, is inherent to any argument, and generally relies on data and purely analytical facts; that is not the case here. It is extremely hard to provide hard-data when trying to argue for Freedom. It is so conceptual, and to us Westerns so present, that it becomes hard, if not impossible, to provide reliable data that would convince one that Freedom is good, perhaps because convincing one of that is not truly necessary. Stallman, therefore, focuses on using Logos purely in the sense of applied logic, and employs it to show what exactly Freedom means on the digital world, which suffices as proof that anything else is non-ideal. To do so, he defines the four fundamental freedoms, ``\ldots Freedom 0 is the freedom to run the program as you wish, for whatever purpose. Freedom 1 is the freedom to study the source code of the program, and change it so it does the computing you want it to do\ldots Freedom 2 is the freedom to redistribute exact copies\ldots Freedom 3 is similar but it's for your modified versions\ldots'' Having established his concept of Freedom, Stallman proceeds to show concrete examples of how these rights are violated by software constantly. Logos, throughout the speech, is used exactly in this form, of giving negative examples as proof of the positive objective through contrasting.

        Throughout his talk, Stallman largely and throughly fails to make use of any Ethos whatsoever; this is intentional. Firstly Stallman is the founder of the movement and ideology which he defends in the speech, which therefore makes himself the ultimate authority on the subject, largely liberating him from the need to quote other thinkers, if not to validate his own ideas. Secondly, the Free Software Foundation, and the movement behind it, are so ubiquitously respected among their target public that it becomes unnecessary to attempt to show some form of credentials while describing their philosophy. Lastly, and perhaps most importantly, it would be completely incongruent with Stallman's personal philosophy to appeal to Ethos in his argument. This is evident once one understands the true meaning of his movement, he is after all often times called an anarchist, although he denies that himself. Therefore, how could a man whose entire life has been dedicated to freedom implement in his own speech directly what he considers to be a freedom-reducing aspect of argument, the fallacy of appeal to authority. In contrast with his speech, however, his slides do make use of Ethos, once again as seen in Logos in a negative form, by alluding to large companies and common technologies as being the perpetrators of freedom in his examples. If there is any Ethos, it is to be found there, in the negative example set by the auxiliary imagery.

        Pathos, being mostly concerned with the appeal to human emotion, is not usually found in technical discussions such as Stallman's, however a closer look at his speech does show us a very clear use of Pathos, even if intertwined with Logos. Firstly, Pathos comes embedded in the visual aids used in the presentation. Although this is not a grammar-based linguistic usage of the strategy, it is undeniably a linguistic communication that creates emotions on the observer, even if the language is a graphical one. In this sense, the Pathos appeal becomes clear when he shows illustrations of, for example, users chained to software, in the case where they are not free. This is very clearly a direct appeal to the strong mental image westerners have of slavery and the crimes against humanity committed in that time, and it causes us to immediately sympathize with his cause in a will to unchain those users who have been slaved. On a second, and arguably deeper, pass the use of Pathos can be traced in the very subject at hand, and the way it is attacked. His target, western, audience has an extremely passionate relation with Freedom as concept and reality, and therefore to simply treat a subject that impacts on it is in itself an act of appeal to Pathos, because it concerns that which is dearest to us. Fundamentally the use of Pathos in the speech is secondary, but very powerful nonetheless and creates a strong impression and sympathy from the listener.

        Overall Stallman's argument is extremely solid, and provides sufficient and agreeable logic to his points. In the areas where he undeniably lacks, such as Ethos in general, and statistical support within Logos, he does so in a way that is either unavoidable or not harmful to his overall point. Examples of this are his lack of hard data, which is hard to even conceive when the discussion regards something as fundamental as freedom, and his avoidance of appealing to authority, a fallacy in its own right. His use of Pathos is subtle and extremely powerful, and his application of Logos is consistent and avoids any form of aristotelian fallacy throughout. Overall Richard Stallman provides us with a solid rhetoric in favor and defense of Human Freedom.

        \newpage
        \begin{workscited}
            \bibent
            Stallman, Richard, ``Free software, free society: Richard Stallman at TEDxGeneva 2014'', \url{https://www.youtube.com/watch?v=Ag1AKIl_2GM}
        \end{workscited}
    \end{mla}
    \end{document}