\documentclass[12pt,letterpaper]{article}
    \usepackage[utf8]{inputenc}
     \usepackage{ifpdf}
    \usepackage{mla}
    \usepackage{todonotes}
    \usepackage[hidelinks]{hyperref}
    \urlstyle{same}
    
    
        \begin{document}
        \begin{mla}{Bernardo}{Meurer}{Professor Martin}{English 110}{October 17, 2017}%
            {DRM Considered Harmful}
        %%%%%%%%%%%%%%%%%%%%%%%%%%%%%%%%%%%%%%%%%%%%%%%%%%%%%%%%%%%%%%%%%%%%%%%
        % Introduction
        In the Digital Era, copyright has become harder to enforce than ever before. Digital works are subject to copying without any quality degradation, which encourages piracy and can lead to market failure (Favale 118). While some optimists (Goldstein 236) claim that digital rights management (DRM) is a successful technology, protecting copyrights in the digital world, and safeguarding markets and creators alike, the truth is quite the opposite. Since current implementations of DRM unjustly and broadly harden access to copyrighted works, they represent a disconnect with the fundamental concepts of copyright, breaking with the Lockean principles for appropriation which are the bedrock for property rights. DRM should be an instrument to defend copyright, not hinder it, and therefore it should either be implemented such that it respects the rights of users and owners, or not be implemented at all.

        %%%%%%%%%%%%%%%%%%%%%%%%%%%%%%%%%%%%%%%%%%%%%%%%%%%%%%%%%%%%%%%%%%%%%%%
        % Paragraph 1
        % One Paraphrase
        Intellectual works have always been bound to a physical medium. The Digital Era began when this changed, and copyrighted works transitioned into the virtual, making it possible for them to be copied freely and without degradation, which represents a grave threat to the rights of creators and to market stability (Favale 117-118). In response to this threat, and in an attempt to halt unwarranted copying of copyrighted material, copyright holders created DRM. Digital Rights Management aims to guarantee that certain digital content can only be used by someone insofar as the copyright owner allows them to. Practically, this means that DRM attempts to guarantee that the video being streamed from Netflix goes into your screen and subsequently to your eyes, and not to some form of recording device. Enforcing this, however, is fundamentally unfeasible, since there is no way for software alone to determine whether or not a camera is being pointed at the screen in order to copy the media in display. Furthermore, even the more feasible attempts, such as ensuring the output device is a bona-fide screen and not a capture device, have failed. Several companies and standards' bodies have proposed and implemented a plethora of DRM schemes, none of which stood the test of time, usually getting cracked in a matter of days. DRM's ultimate failure manifests itself in the fact that not one single movie, song, or video game released in the past decade that reached a minimal degree of success is unavailable on the web free of charge. Therefore, it is safe to say that if the entertainment industry, with all its resources and manpower, could not come up with a reliable implementation of DRM after over 20 years of constant development, then DRM as intended by them is not achievable.

        %%%%%%%%%%%%%%%%%%%%%%%%%%%%%%%%%%%%%%%%%%%%%%%%%%%%%%%%%%%%%%%%%%%%%%%
        % Paragraph 2
        % One quote, two paraphrases
        According to Locke, ``The Labor of one's Body and the Work of his Hands, we may say, are properly his'' (287-288). From this notion, that one appropriates something from the common by applying labor onto it, common law established the concept of property, and the rights that come with it. Lockean property rights, however, don't protect just the laborer, be it an author or a hunter, but also the remaining public. While Locke grants the owner, and him alone, the rights of use, transfer, and exclusion of his property, it also gives the public rights to the remaining common so that the community can thrive (Favale 119). This falls from the idea that, according to Locke, appropriation is only valid insofar as it leaves ``enough, and as good'' (291) of the common to the community. Hegel's take on property right is similar to Locke's in a sense, as Hegel will argue that property is the expression of Man's personality (84-88). With this, and drawing a parallel between Lockean and Hegelian philosophy, one can firmly state that intellectual property is just another form of property, since it's an expression of the personality of the author. This is not to say, however, that intellectual and material property share the exact same attributes, namely it is unique to the former that it lacks extension. This lack, its immaterialness, makes it subject to copying, which can harm the owner even though it doesn't deprive him of his creation. From this follows that there must be some form of anti-copying protection applied to intellectual property in order to safeguard the author's rights. Said protection is what modern law calls copyright, and it is that which DRM attempts to enforce on the digital realm.
        
        %%%%%%%%%%%%%%%%%%%%%%%%%%%%%%%%%%%%%%%%%%%%%%%%%%%%%%%%%%%%%%%%%%%%%%%
        % Paragraph 3
        % One paraphrase, two quotes
        Western society is fundamentally based on strong property rights, the concept of appropriation, and the enforcement of said rights are the basis of modern capitalism. Property rights grant the holder a range of privileges (transfer, use, and so on), but it also grants the public a set of complementary rights, namely some protection from an malicious party, as a rights holder can only constrain others from using his property insofar as that doesn't impair a community's ability to thrive (Gordon 1555-1560). Hence an individual's property should not harm the public's access to the common. Moreover, in Lockean theory the interests of the community must prevail over those of the individual when they clash (Favale 120), which reveals something fundamental, and often overlooked, about copyright which is, as Neil Netanel points out, that ``while copyright may operate in the market, copyright's fundamental goals are not of the market'' (341). Copyright, therefore, has goals which go beyond the market itself. Natanel goes on to claim that copyright is a vehicle of democracy, pointing once more towards interpreting copyright as protecting the public first, and the author second. Furthermore, the very first copyright legislation, the Copyright Act 1710, had as its subtitle ``An Act for the Encouragement of Learning ...'' (Wortley), meaning that since its very beginning copyright had the public's interest in mind first and foremost. With this it becomes clear that copyright goes far beyond both markets and authors, enlightening the fundamental flaw of DRM, which is its ignorance towards the rights of the remaining public.

        %%%%%%%%%%%%%%%%%%%%%%%%%%%%%%%%%%%%%%%%%%%%%%%%%%%%%%%%%%%%%%%%%%%%%%%
        % Paragraph 4
        As previously outlined, Lockean philosophy constrains one's property to leave ``enough and as good'' (Locke 291) to the public. While here, ``enough and as good'' is referring to the common, by which Locke meant that which Nature provides us, this concept can easily be generalized to apply to the immaterial as well. Fundamentally, this means that in the same way there is a common of things, given to us by Nature, there is also a common of intangibles, ideas, that every individual must be able to use freely. As Gordon points out, ``Our common encompasses not only our physical country but our culture as well'' (1555). An example of a non-physical common of ideas is public domain, which is composed of works that never were, or that no longer are, protected by copyright. It follows from this that no appropriation can be made of that which is in the common of ideas, and therefore no copyright can be claimed. Since the fundamental goal of copyright law lies beyond the market, being that it should encourage learning and democratic discourse, it must always protect the common, of resources and of ideas, from wrongful appropriation. It is here, in the public's entitlement to the common, that DRM falls short. Since the protection schemes have no expiration date, the material can't be accessible to the public even after the copyright has expired, unless the author re-releases it striped of DRM. Moreover, given that DRM acts on the display of content, it blocks even legitimate uses of the material. Take the fair use doctrine for example, a practice of the legislature to allow use of copyrighted works, despite the lack of permission from the owner, insofar as the usage satisfies some criteria, namely being transformative in nature, and not being a direct market competitor to the original. It is clear that in order for fair use to happen the user must have access to the work. If, however, said work is protected by DRM, then no access to the content can happen without authorization from the owner, hence no fair use can happen on the content, breaking with the doctrine. It is thus clear that DRM infringes on the public common, and that therefore it infringes on copyright itself, even if its intention was to safeguard it.
    
        %%%%%%%%%%%%%%%%%%%%%%%%%%%%%%%%%%%%%%%%%%%%%%%%%%%%%%%%%%%%%%%%%%%%%%%
        % Conclusion
        The dawn of the Digital Era did not change the fundament of copyright law. As shown, both Lockean and Hegelian philosophies agree that copyright has first and foremost a social function, a duty towards the public, outweighing the rights it gives the author. The immaterial attribute of intellectual property has caused an increase in free-riding behavior, a reaping by who has not sown, in its attempt to solve this issue, DRM trumps public entitlements and shifts protection in favour of the rights holder. DRM in its current state is fundamentally anti-copyright, despite its intent, since its interest and value are contained in the market, for it aims only to protect the owner's profits, unlike the copyrights which it claims to enforce. DRM should not exist legally, for it infringes on the fundamental property rights our society is based on, and takes away from the public's access to the common of intangibles. As said by Salman Rushdie, ``those who do not have power over the story that dominates their lives, power to retell it, rethink it, deconstruct it, joke about it, and change it as times change, truly are powerless, because they cannot think new thoughts,'' and no technology, no matter how lucrative, should take this power away from us, the people.
        

        %%%%%%%%%%%%%%%%%%%%%%%%%%%%%%%%%%%%%%%%%%%%%%%%%%%%%%%%%%%%%%%%%%%%%%%
        \begin{workscited}
            \bibent
            Favale, Marcella. ``Death and Resurrection of copyright between Law and Technology.'' Information \& Communications Technology Law, vol. 23, no. 2, Apr. 2014, pp. 117–135., doi:10.1080/13600834.2014.925631. 

            \bibent
            Gordon, Wendy J. ``A Property Right in Self-Expression: Equality and Individualism in the Natural Law of Intellectual Property.'' \textit{The Yale Law Journal}, vol. 102, no. 7, 1993, pp. 1533–1609. JSTOR, \url{www.jstor.org/stable/796826}.

            \bibent
            Hegel, Georg Wilhelm Friedrich. ``Elements of the Philosophy of Right.'' Edited by Allen W. Wood. Translated by Hugh Barr Nisbet, Cambridge University Press, 2012.

            \bibent 
            Locke, John. ``Second Treatise.'' \textit{Locke: Two Treatises of Government}, edited by Peter Laslett, Cambridge University Press, Cambridge, 1988, pp. 265–428. Cambridge Texts in the History of Political Thought.

            \bibent
            Netanel, Neil Weinstock, ``Locating Copyright Within the First Amendment'' Skein. \textit{Stanford Law Review}, Vol. 54, October 2001. Available at SSRN: \url{www.ssrn.com/abstract=267848} or \url{www.dx.doi.org/10.2139/ssrn.267848}

            \bibent
            Goldstein, Paul. ``The Future of Copyright in a Digital Environment: Summary of Discussion,'' in \textit{The Future of Copyright in a Digital Environment: Proceedings of the Royal Academy Colloquium} (Amsterdam, 6-7 July 1995), Kluwer Law International, 1996.

            \bibent
            Rushdie, Salman. ``Excerpts From Rushdie's Address: 1,000 Days `Trapped Inside a Metaphor'.'' The New York Times, 12 Dec. 1991, \url{www.nytimes.com/books/99/04/18/specials/rushdie-address.html}. 
            %\newpage

            \bibent
            Wortley, Edward. ``Statute of Anne.'' The Avalon Project : The Statute of Anne; April 10, 1710, \url{www.avalon.law.yale.edu/18th_century/anne_1710.asp}.
        
        \end{workscited}
        \end{mla}
        \end{document}