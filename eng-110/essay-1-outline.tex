\documentclass{article}
    
    \usepackage[utf8]{inputenc}
    \usepackage{fancyhdr}
    \usepackage{extramarks}
    \usepackage{amsmath}
    \usepackage{amsthm}
    \usepackage{amsfonts}
    \usepackage{enumitem}
    \usepackage{tikz}
    \usepackage[plain]{algorithm}
    \usepackage{algpseudocode}
    
    \usetikzlibrary{automata,positioning}
    
    %
    % Basic Document Settings
    %
    
    \topmargin=-0.45in
    \evensidemargin=0in
    \oddsidemargin=0in
    \textwidth=6.5in
    \textheight=9.0in
    \headsep=0.25in
    
    \linespread{1.1}
    
    \pagestyle{fancy}
    \lhead{\hmwkAuthorName}
    \chead{\hmwkClass\ (\hmwkClassInstructor): \hmwkTitle}
    \rhead{\firstxmark}
    \lfoot{\lastxmark}
    \cfoot{\thepage}
    
    \renewcommand\headrulewidth{0.4pt}
    \renewcommand\footrulewidth{0.4pt}
    
    \setlength\parindent{0pt}
    
    %
    % Create Problem Sections
    %
    
    \newcommand{\enterProblemHeader}[1]{
        \nobreak\extramarks{}{Problem \arabic{#1} continued on next page\ldots}\nobreak{}
        \nobreak\extramarks{Problem \arabic{#1} (continued)}{Problem \arabic{#1} continued on next page\ldots}\nobreak{}
    }
    
    \newcommand{\exitProblemHeader}[1]{
        \nobreak\extramarks{Problem \arabic{#1} (continued)}{Problem \arabic{#1} continued on next page\ldots}\nobreak{}
        \stepcounter{#1}
        \nobreak\extramarks{Problem \arabic{#1}}{}\nobreak{}
    }
    
    \setcounter{secnumdepth}{0}
    \newcounter{partCounter}
    \newcounter{homeworkProblemCounter}
    \setcounter{homeworkProblemCounter}{1}
    \nobreak\extramarks{Problem \arabic{homeworkProblemCounter}}{}\nobreak{}
    
    %
    % Homework Problem Environment
    %
    % This environment takes an optional argument. When given, it will adjust the
    % problem counter. This is useful for when the problems given for your
    % assignment aren't sequential. See the last 3 problems of this template for an
    % example.
    %
    \newenvironment{homeworkProblem}[1][-1]{
        \ifnum#1>0
            \setcounter{homeworkProblemCounter}{#1}
        \fi
        \section{Problem \arabic{homeworkProblemCounter}}
        \setcounter{partCounter}{1}
        \enterProblemHeader{homeworkProblemCounter}
    }{
        \exitProblemHeader{homeworkProblemCounter}
    }
    
    %
    % Homework Details
    %   - Title
    %   - Due date
    %   - Class
    %   - Section/Time
    %   - Instructor
    %   - Author
    %
    
    \newcommand{\hmwkTitle}{Essay 1 Outline}
    \newcommand{\hmwkDueDate}{September 10, 2017}
    \newcommand{\hmwkClass}{English}
    \newcommand{\hmwkClassTime}{}
    \newcommand{\hmwkClassInstructor}{Professor Joanne Martin}
    \newcommand{\hmwkAuthorName}{\textbf{Bernardo Meurer}}
    
    %
    % Title Page
    %
    
    \title{
        \vspace{2in}
        \textmd{\textbf{\hmwkClass:\ \hmwkTitle}}\\
        \normalsize\vspace{0.1in}\small{Due\ on\ \hmwkDueDate\ at 11:59pm}\\
        \vspace{0.1in}\large{\textit{\hmwkClassInstructor\ \hmwkClassTime}}
        \vspace{3in}
    }
    
    \author{\hmwkAuthorName}
    \date{}
    
    \renewcommand{\part}[1]{\textbf{\large Part \Alph{partCounter}}\stepcounter{partCounter}\\}
    
\begin{document}
\maketitle
\pagebreak
\section{Introduction Paragraph}
\begin{itemize}
    \item Present the topic: ``Is college worth the costs?''
    \begin{itemize}
        \item Mention high cost
        \item Mention time to completion
        \item Mention uncertainty of graduating
    \end{itemize}
    \item Thesis
    \begin{itemize}
        \item College, despite it's cost, is an investment of unquestionable 
        worth both financially and personally.
    \end{itemize}
\end{itemize}
\textbf{Over the course of the past century, college enrollment 
has had a steady increase, independent of the ludicrous raise in cost
of attendance. Pricing, in combination with duration, and the uncertainty
of graduation makes it is easy to place doubt on the choice of 
obtaining a bachelor's degree. Unintuitively, however, the worth of
college has not yet morphed into debt, quite the opposite. College, 
despite the sacrifices it entices, is an investment of unquestionable
yields, which are not constrained solely to the financial aspects of life.
}
\section{Body Paragraph \#1}
\begin{itemize}
    \item Topic Sentence --- \textbf{The cost of college has, throughout the 
    last century, risen well above the Consumer Price Index, yet it's Return on
    Investment remains comfortably positive.}
    \item Support
    \begin{itemize}
        \item See 6.2 and 6.3 of Essay 1 Proposal
        \item “Is college worth it? Clearly, new data say” by David Leonhart
        \item “Is it still worth going to college?” by Mary C. Daly \& Leila Bengali
        \item This is probably the most empirical point in the essay. Here, 
        clear data alone should suffice.
    \end{itemize}
\end{itemize}
\section{Body Paragraph \#2}
\begin{itemize}
    \item Topic Sentence --- \textbf{In the current geopolitical arena, where
    the main battle is for technological supremacy, skilled workers are more 
    valuable than ever, making the struggle for a degree often times a 
    patriotic one.}
    \item Support
    \begin{itemize}
        \item See 6.1 of Essay 1 Proposal
        \item “What’s the value of a college education” by Jennifer Barrett
        \item The point here must be made in a purer form, relying less on
        empiricism. An interesting point to make here is to show how the Cold War
        was won with technology, which relies on graduates and universities,
        and how today a new race is pacing with China, and once more it is only
        with a qualified work force that the US will be able to maintain it's global
        leadership.
    \end{itemize}
\end{itemize}
\newpage
\section{Body Paragraph \#3}
\begin{itemize}
    \item Topic Sentence --- \textbf{College has immaterial benefits, which is where it's
    most fundamental value lies, for no other experience will provide you with the growth 
    and improvement that college does.}
    \item Support
    \begin{itemize}
        \item Here the argument should surround how attending college makes you a better
        individual. Interesting points in this are:
        \begin{itemize}
            \item Better civic behavior (voting, running for local office, demonstrating)
            \item Communication skills
            \item Higher ability to help others in academic matters
        \end{itemize}
        \item Quotations for this point are scarce, however, Practical Argument does
        contain a few segments in which better voting habits for college graduates is 
        mentioned. If needed, seeking that out could be interesting.
    \end{itemize}
\end{itemize}
\section{Refutation of Opposing Arguments}
\begin{itemize}
    \item ``College is too expensive'' --- The return on investment is still good, although 
    not instantaneous
    \item ``College takes too long'' --- The U.S. median life expectancy is of 80 years. 
    The 4 years of college are a mere 5\% of that, and it has a highly positive 
    impact on the future years. In fact, being a college graduate places you in a group 
    with higher life expectancy.
    \item ``You don't need College to be successful'' --- Point out fallacious 
    cherry-picking of data, argue that, statistically, college \emph{does} improve your 
    chances of success. Fundamentally, you don't truly \emph{need} anything to be successful, 
    point being that the statement is meaningless. 
\end{itemize}
\section{Conclusion}
\begin{enumerate}[label=\Alph*.]
    \item Few experiences in life present one with an opportunity to attain both
    financial and personal growth, and college is perhaps the most quintessential 
    of them.
    \item College rewards one economically, mentally, socially, and -- perhaps most
    importantly -- helps one's country thrive. It's cost is but a speck in the vast
    universe of benefits it provides.
    \item The choice of getting a degree is under more fire now than ever before,
    and yet it is still intrinsically valuable. Graduating is an enlightening and 
    molding experience, that has shaped many of great women and men of this nation.  
\end{enumerate}
\end{document}  