\documentclass[12pt,letterpaper]{article}
    \usepackage[utf8]{inputenc}
    \usepackage{ifpdf}
    \usepackage{mla}
    \usepackage{todonotes}
    \usepackage{setspace}

    \begin{document}
    \begin{mla}{Bernardo}{Meurer}{Professor Martin}{English 110}{November 14, 2017}{Reflection}
    \begin{singlespace}
        \textbf{``Pedestrians stopped. Cars stopped. The city stopped. They all watched me dance with my grandmother. I was my grandmother, dancing.''}
    \end{singlespace}

    I have, by now, grown painfully tired of the supposed mysticism surrounding Indians. On the first story we saw Corliss, also a Spokane Indian, who confessed to us that the pseudo-magical aspect of Indians was complete ``white-people'' nonsense. She confessed to the fact that it was lore, and yet in this story we see Alexie trying to sell us that very lore. I suppose he's very much aware of the stupid amazement that the ``white folk'' have for mysticism in general, and therefore he composed a story surrounded by pseudo-magical brouhaha, which is certain to make the eyes of the reader shine with happiness. Assuming the reader does Yoga and eats avocado toasts on a regular basis. Oh, and is white, if that wasn't obvious from the two characteristics just mentioned. I suppose what I find uninteresting in the story, and it is what the quote reveals, is how Alexie \emph{knows} his audience, know what they want to read, and gives them that, despite my belief that he himself finds all of this magic to be silly.

    \end{mla}
    \end{document}