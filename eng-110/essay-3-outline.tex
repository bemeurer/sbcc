\documentclass[12pt,letterpaper]{article}
    \usepackage[utf8]{inputenc}
     \usepackage{ifpdf}
    \usepackage{mla}
    \usepackage{todonotes}
    
    
        \begin{document}
        \begin{mla}{Bernardo}{Meurer}{Professor Martin}{English 110}{October 26, 2017}%
            {Essay \#3 Outline}
        \noindent \textbf{Introduction:}

        In the Digital Era, copyright has become harder to enforce than ever before. Digital works are subject to copying without any quality degradation, which encourages piracy and can lead to market failure (Favale 118). While some optimists (Goldstein 236) claim that digital rights management (DRM) is a successful technology, protecting copyrights in the digital world, and safeguarding markets and creators alike, the truth is quite the opposite. Since current implementations of DRM unjustly and broadly harden access to copyrighted works, they represent a disconnect with the fundamental concepts of copyright, breaking with the Lockean principles for appropriation which are the bedrock for property rights. DRM should be an instrument to defend copyright, not hinder it, and therefore it should either be implemented such that it respects the rights of users and owners, or not be 
        implemented at all.

        \noindent \textbf{Defining the Problem:}

        \begin{enumerate}
            \item Introduce what DRM is
            \begin{itemize}
                \item Netflix example, CD ripping example
            \end{itemize}
            \item Define Property
            \item Use Locke to have a careful philosophical outline of what qualifies as property and appropriation.
            \begin{itemize}
                \item Example of the gatherer in the forest
            \end{itemize}
            \item Raise the issue of the immaterialness of intellectual property.
            \item Bring Hegel's theory on property rights to solve the question (Property as an expression of Man's personality)
            \begin{itemize}
                \item Relate to the example given with Locke's theory
            \end{itemize}
            \item Conclude that intellectual property is subject to the very same rights than material property.
            \item Introduce copyright as a protection to IP.
            \item Show how DRM infringes on Copyright
        \end{enumerate}
        
        \noindent \textbf{Causes of the Problem:}

        The causes will probably be bundled with the definition
        \begin{enumerate}
            \item Lack of government legislation. Companies get to implement DRM as they want without respecting the users.
            \item Lack of understanding by the public of their own rights and entitlements guaranteed by copyright
            \item Misinterpretation of the rationale behind Copyright law (See British legislation)
        \end{enumerate}

        \noindent \textbf{Consequences of the Problem:}
        
        \begin{enumerate}
            \item The public is kept from accessing what is rightfully theirs (the common of ideas)
            \item Subversion of the fundamental intent of the legislation 
            \item Break with fair-use doctrine
            \item Infringing on the natural property-decay into public domain.
            \item Correlate DMCA, DRM, and the prohibition on scientific research on the topic. Correlate with privacy concerns.
        \end{enumerate}

        \noindent \textbf{Solutions:}
        
        Edward Felten proposed through The Association for Computer Machinery a set of solutions that I believe would work. I just have to connect each one to the consequences and causes previously shown. 
        \begin{enumerate}
            \item ``Public policy should enable a variety of DRM approaches to emerge, should facilitate competition between them, and encourage interoperability among them.''
            \item ``DRM should not be used to restrict the rights of consumers. Policymakers must actively monitor the use of DRM and amend policies as necessary to protect the public's entitlements''
            \item ``Public policy should ensure that DRM systems collect, store, and redistribute private information about users only to the extent required for their proper operation, that they follow fair information practices, and that they are subject to informed consent by users.''
            \item ``DRM systems and policies should not interfere with legitimate research or with discourse about research results or other matters of public concern. Laws concerning DRM should contain explicit exceptions to protect this principle.''
            \item ``Policies meant to reinforce copyright should be limited to applications where copyright interests are actually at stake.''
        \end{enumerate}

        \noindent \textbf{Anticipation of Opposing Arguments:}

        More research is needed to understand what are the arguments in favor of DRM, since generally the excuses given by the likes of MPAA and RIAA completely ignore any point made and revert to non-sequiturs and strawmans. It is hard to counterargument against nothing. Favale and Gordon point out some ``DRM optimists'' whom if I can find articles by, I should be able to counter.

        \noindent \textbf{Conclusion:}

        The dawn of the Digital Era did not change the fundament of copyright law. As shown, both Lockean and Hegelian philosophies agree that copyright has first and foremost a social function, a duty towards the public, outweighing the rights it gives the author. The immaterial attribute of intellectual property has caused an increase in free-riding behavior, a reaping by who has not sown, in its attempt to solve this issue, DRM trumps public entitlements and shifts protection in favour of the rights holder. DRM in its current state is fundamentally anti-copyright, despite its intent, since its interest and value are contained in the market, for it aims only to protect the owner's profits, unlike the copyrights which it claims to enforce. DRM should not exist legally, for it infringes on the fundamental property rights our society is based on, and takes away from the public's access to the common of intangibles. As said by Salman Rushdie, ``those who do not have power over the story that dominates their lives, power to retell it, rethink it, deconstruct it, joke about it, and change it as times change, truly are powerless, because they cannot think new thoughts,'' and no technology, no matter how lucrative, should take this power away from us, the people.
        
        \end{mla}
        \end{document}