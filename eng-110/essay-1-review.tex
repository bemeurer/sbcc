\documentclass[12pt,letterpaper]{article}
    \usepackage[utf8]{inputenc}
     \usepackage{ifpdf}
    \usepackage{mla}
    \usepackage{todonotes}
    
    
        \begin{document}
        \begin{mla}{Bernardo}{Meurer}{Professor Martin}{English 110}{October 24, 2017}{More than paper: The value of a Bachelor's degree}
        
        Obtaining a college degree, the promise of higher earnings and education, has always been an integral part of the american dream. As such, over the course of the past century college enrollment has seen a steady increase, despite the ludicrous rise in cost of attendance. This rise in cost, in combination with the time requirement of a degree, and the uncertainty of graduation, makes it easy to place doubt on the choice of obtaining a Bachelor's degree. At first glace, it seems one should avoid the struggle of college altogether, and aim for an early entrance into the labor force, valuing experience over formal education. Unintuitively, however, the worth of college has not yet morphed into debt. College, despite the sacrifices it entices, is an investment of unquestionable yields, which are not constrained solely to the financial aspects of life.
    
%%%%%%%%%%%%%%%%%%%%%%%%%%%%%%%%%%%%%%%%%%%%%%%%%%%%%%%%%%%%%%%%%%%%%%%%%%%%%%%
        The cost of college has, throughout the last century, rose well above the Consumer Price Index (CPI), yet it's Return on Investment (ROI) remains comfortably positive. According to the National Center for Education Statistics, the cost of tuition has increased by almost a trifold, even when corrected for inflation. In line with this, however, comes a rise in the average salary of graduates, which not only compensates the price of tuition, but far surpass it. In his essay ``Is College Worth it?'', David Leonhart points out that ``The pay gap between college graduates and everyone else reached a record high last year [2013].'' This is indeed a common theme in discussion on the worth of college, Mary C. Daly \& Leila Bengali in their essay ``Is it Still Worth Going to College?'' state that ``For most Americans the path to higher future earnings involves a four-year college degree,'' and if one looks at the average earnings of those with a bachelor's degree against those without this becomes exceedingly clear. Moreover, Daly \& Leila also show that the average college graduate can recover the cost of attendance in under 20 years, which is well in line with the ROI data provided by PayScale, who shows some schools have a one million dollar return in just 20 years. Fundamentally this means that the seemingly astronomical rise in cost of attendance has come hand-in-hand with an even greater rise in salaries for college graduates, and perhaps most importantly, that this rise has not happened to unqualified jobs. In other words, in terms of return on investment there has never been a better time to have a college degree, and the inverse also holds, we are at the worst point to not have a college degree.
    
%%%%%%%%%%%%%%%%%%%%%%%%%%%%%%%%%%%%%%%%%%%%%%%%%%%%%%%%%%%%%%%%%%%%%%%%%%%%%%%
        In the current geopolitical arena, where the main battle is for technological supremacy, skilled workers are more valuable than ever, making the struggle for a degree often times a patriotic one. During the 20th century, the world saw an ideological clash of titans: the Cold War. In it, the United States were undeniably the victorious party over the machiavellian Soviet regime. Evidently, what made said victory possible was the technological edge the Americans had over the Soviets; in their essay ``Who Won the Cold War?'', Daniel Deudney and Gilford Ikenberry confirm this: ``the West held a decisive technological edge''. Presently, a new technological race is pacing, and the contender is now China, a country with vastly more human capital than the United States. The only hope for yet another American victory is for us to specialize efficiently, and outperform them. Jennifer Barret alludes to this in ``What's the value of a college education?'', where she states that ``the country will need 250,000 more engineers over the next 10 years.'' It is trivial to see that, unless each and every American commit strongly to their education, we will not be able to keep up, and yet this begs the question of how does one ``specialize efficiently,'' the answer to which lies in quality formal education. The edge of scientific advance has, for a long time now, been completely out of reach for the lone, independent student, long gone are the days of basement discoveries by the likes of Fermat (a lawyer by training). Today, in order to advance science one must work with others and dedicate years to training. College is molded exactly after this necessity, of having a scientific community that can both train others and allow individuals to work together, and therefore it follows that the best way achieve said specialization is by dedicating oneself to college education.
    
%%%%%%%%%%%%%%%%%%%%%%%%%%%%%%%%%%%%%%%%%%%%%%%%%%%%%%%%%%%%%%%%%%%%%%%%%%%%%%%
        College has immaterial benefits, which is where it's most fundamental value lies, for no other experience will provide individuals with the growth and improvement that college does. One interesting dimension of this is found in observing the civic behavior of those with a bachelor's degree. We find that they, on average, are more active voters, are more likely to run for office, and are more willing to demonstrate and protest, activities that are fundamental for a working democracy. Moreover, college education leads to a more well informed individual, and one who is likely to make more rational and well-rounded political decisions. Jennie Le in her essay ``What Does it Mean to be a College Grad?'' elucidates the representativeness of a degree. Le states, ``this degree symbolizes my family being able to make and take the opportunities that we've been given in America.'' Le's thoughts echo that of many Americans who see their degree as a validation of the struggle taken by their families to make such an achievement possible. The central point here is that college can be seen not as the means to an end, but as an end in itself. The experience of college (and graduation subsequently) is more than a pathway to employment, it molds your personality and morals, and is a deeply transmogrifying process. It is clear and distinct (in a Descartian sense) that no person who truly and fully subjects themselves to college will remain unchanged, on the contraty they will experience growth and introspection.
    
%%%%%%%%%%%%%%%%%%%%%%%%%%%%%%%%%%%%%%%%%%%%%%%%%%%%%%%%%%%%%%%%%%%%%%%%%%%%%%%
        Independent of the amount of data available, the path of pursuing a college degree will always have an opposition. The economically-minded will argue that it's too high of a cost, ignoring the overwhelming return the investment brings in the long term. Similarly, some will defend that it takes too long to complete, overlooking the fact that a 4-year endeavour represents, according to the CDC, less than 4\% of the median life expectancy. Moreover, having a college degree increases one's life expectancy on average. Lastly, to Aristotle's despair, there are those who will cherry-pick data to point out that a degree is not needed for one to be successful. This statement, albeit true, holds very little meaning since one could argue that, fundamentally, nothing is truly needed for one to achieve success in general. Moreover, the general argumentation behind this is guilty of the Composition fallacy, which turns it from meaningless to nonsensical.
    
%%%%%%%%%%%%%%%%%%%%%%%%%%%%%%%%%%%%%%%%%%%%%%%%%%%%%%%%%%%%%%%%%%%%%%%%%%%%%%%
        Few experiences in life present one with an opportunity to attain both financial and personal growth, and college is perhaps the most quintessential of them. Attending university rewards one economically, mentally, socially, and -- perhaps most importantly -- helps one's country thrive. Cost is but a speck in the vast universe of benefits college provides. Finally, the choice of getting a degree is under more fire than ever before, and yet it is still intrinsically valuable. Graduating is and enlightening and molding experience, that has shaped many of the great women and men of this nation.

%%%%%%%%%%%%%%%%%%%%%%%%%%%%%%%%%%%%%%%%%%%%%%%%%%%%%%%%%%%%%%%%%%%%%%%%%%%%%%%
        \begin{workscited}
        \bibent
        Barret, Jennifer. ``What's the value of a college education? It depends'' \textit{CNBC}, 2015. https://www.cnbc.com/2015/06/19/is-a-college-degree-overvalued.html
    
        \bibent
        Deudney, Daniel, and G. John Ikenberry. ``Who Won the Cold War?'' \textit{Foreign Policy}, no. 87, 1992, pp. 123–-138. JSTOR www.jstor.org/stable/1149164.
    
        \bibent
        Hansen, Hans. ``Fallacies'', \textit{The Stanford Encyclopedia of Philosophy (Fall 2017 Edition)}, Edward N. Zalta (ed.) https://plato.stanford.edu/archives/fall2017/entries/fallacies.
        
        \bibent
        Le, Jennie. ``What Does it Mean to be a College Grad?'' \textit{Practical Argument: A Text and Anthology}. Ed. Laurie G. Kirszner, Stephen R. Mandell. Boston: Bedford/St. Martin’s, 2014. 41-42.
    
        \bibent
        National Center for Health Statistics. ``Health, United States, 2016: With Chartbook on Long-term Trends in Health''. Hyattsville, MD. 2017. pp. 116 https://www.cdc.gov/nchs/data/hus/hus16.pdf
        
        \bibent
        National Center for Health Statistics. `` Health, United States, 2016: With Chartbook on Long-term Trends in Health''. Hyattsville, MD. 2017. pp. 57 https://www.cdc.gov/nchs/data/hus/hus11.pdf
    
        \bibent
        PayScale, ``College ROI Report: Best Value Colleges''. PayScale, 2017, www.payscale.com/college-roi. 
    
        \bibent
        Snyder, T.D., de Brey, C., and Dillow, S.A. (2016). ``Digest of Education Statistics 2015 (NCES 2016-014)''. \textit{National Center for Education Statistics}, Institute of Education Sciences, U.S. Department of Education. Washington, DC. https://nces.ed.gov/pubs2016/2016014.pdf
        
        \end{workscited}
        \end{mla}
        \end{document}