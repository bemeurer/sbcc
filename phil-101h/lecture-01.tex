\documentclass{article}
    %%%%%%%%%%%%%%%%%%%%%%%%%%%%%%%%%%%%%%%%%%%%%%%%%%%%%%%%%%%%%
    % Lecture Specific Information to Fill Out
    %%%%%%%%%%%%%%%%%%%%%%%%%%%%%%%%%%%%%%%%%%%%%%%%%%%%%%%%%%%%%
    \newcommand{\LectureTitle}{Lecture \#1 Notes}
    %\newcommand{\LectureDate}{\today}
    \newcommand{\LectureDate}{January\ 24,\ 2017}
    \newcommand{\LectureClassName}{PHIL\ 101H}
    \newcommand{\LatexerName}{Bernardo\ Meurer}
    %%%%%%%%%%%%%%%%%%%%%%%%%%%%%%%%%%%%%%%%%%%%%%%%%%%%%%%%%%%%%

    % Change "article" to "report" to get rid of page number on title page
    \usepackage{amsmath,amsfonts,amsthm,amssymb}
    \usepackage{setspace}
    \usepackage{Tabbing}
    \usepackage{fancyhdr}
    \usepackage{lastpage}
    \usepackage{extramarks}
    \usepackage{chngpage}
    \usepackage{soul,color}
    \usepackage{graphicx,float,wrapfig}
    \usepackage{afterpage}
    \usepackage{abstract}
    \usepackage{syllogism}

    % In case you need to adjust margins:
    \topmargin=-0.45in
    \evensidemargin=0in
    \oddsidemargin=0in
    \textwidth=6.5in
    \textheight=9.0in
    \headsep=0.25in

    % Setup the header and footer
    \pagestyle{fancy}
    \lhead{\LatexerName}
    \chead{\LectureClassName: \LectureTitle}
    \rhead{\LectureDate}
    \lfoot{\lastxmark}
    \cfoot{}
    \rfoot{Page\ \thepage\ of\ \pageref{LastPage}}
    \renewcommand\headrulewidth{0.4pt}
    \renewcommand\footrulewidth{0.4pt}

        %%%%%%%%%%%%%%%%%%%%%%%%%%%%%%%%%%%%%%%%%%%%%%%%%%%%%%%%%%%%%
    % Some tools
    \newcommand{\enterTopicHeader}[1]{\nobreak\extramarks{#1}{#1 continued on next page\ldots}\nobreak%
    \nobreak\extramarks{#1 (continued)}{#1 continued on next page\ldots}\nobreak}
\newcommand{\exitTopicHeader}[1]{\nobreak\extramarks{#1 (continued)}{#1 continued on next page\ldots}\nobreak%
   \nobreak\extramarks{#1}{}\nobreak}

\newlength{\labelLength}
\newcommand{\labelAnswer}[2]
{\settowidth{\labelLength}{#1}
\addtolength{\labelLength}{0.25in}
\changetext{}{-\labelLength}{}{}{}
\noindent\fbox{\begin{minipage}[c]{\columnwidth}#2\end{minipage}}
\marginpar{\fbox{#1}}

% We put the blank space above in order to make sure this
% \marginpar gets correctly placed.
\changetext{}{+\labelLength}{}{}{}}

\setcounter{secnumdepth}{0}
\newcommand{\TopicName}{}
\newcounter{TopicCounter}
\newenvironment{Topic}[1][Problem \arabic{TopicCounter}]
{\stepcounter{TopicCounter}
\renewcommand{\TopicName}{#1}
\section{\TopicName}
\enterTopicHeader{\TopicName}}
{\exitTopicHeader{\TopicName}}

\setcounter{secnumdepth}{0}
\newcommand{\ExampleSectionName}{}
\newcounter{ExampleSectionCounter}[TopicCounter]
\newenvironment{ExampleSection}[1][Example \arabic{ExampleSectionCounter}]
{\stepcounter{ExampleSectionCounter}
\renewcommand{\ExampleSectionName}{#1}
\section{\ExampleSectionName}
\enterTopicHeader{\ExampleSectionName}}
{\exitTopicHeader{\ExampleSectionName}}

\setcounter{secnumdepth}{0}
\newcounter{ExampleBoxCounter}[TopicCounter]
\newcommand{\examplebox}[1]
{
% We put this space here to make sure we're disconnected from the previous
% passage
\stepcounter{ExampleBoxCounter}
\noindent\fbox{\begin{minipage}[c]{\columnwidth}#1\end{minipage}}\enterTopicHeader{\ExampleSectionName}\exitTopicHeader{\ExampleSectionName}\marginpar{\fbox{\#\arabic{ExampleBoxCounter}}}
% We put the blank space above in order to make sure this
% \marginpar gets correctly placed.
\vskip10pt%
}

\renewcommand{\contentsname}{{\normalsize Topics Covered}}
\renewcommand{\abstractname}{\LectureTitle\ Summary}
\renewcommand{\absnamepos}{flushleft}

%%%%%%%%%%%%%%%%%%%%%%%%%%%%%%%%%%%%%%%%%%%%%%%%%%%%%%%%%%%%%
\begin{document}
    \begin{spacing}{1.2}
        \newpage
        \begin{itemize}
            \item Philosophy: Love os wisdom
            \item Ethics: The study of how we ought to live with others and ourselves.
            \begin{itemize}
                \item ``Should members of the same sex be allowed to marry?''
                \item ``Is abortion ever morally permissible?''
                \item ``Is it OK to give humans a higher moral status than animals?''
            \end{itemize}
            \item The philosophical approach to ethics is to defend one's answers with \emph{reason}.
            \item The philosophical approach then is to provide \emph{arguments}.
            \item An argument is a finite ser if statements, called \emph{premises}, that are intended to support the truth of another statement, the \emph{conclusion}.
            \item \syllog{God is utterly perfect}{A lack of existence is a lack of perfection}{God does not lack existence}
            \item \syllog{Philosophy is good.}{Love is good}{Philosophy is love.}
            \item Structure \(\alpha \circ \beta \implies \gamma \)
            \item Every good argument must meet \emph{two} conditions.
            \begin{itemize}
                \item Content
                \item Form
            \end{itemize}
            \item Good content means the premises are true
            \item Good form means the premises actually support the conclusion.
            \item Testing form: Assume premises, check conclusion.
            \item The philosophy is love argument has a fault because two sets with one congruent element are not the same set necessarily
            \item \(\alpha\in A; \alpha\in B \not\implies A = B\)
            \item A \emph{moral theory} is a theory about the nature of the right and the good and the proper method for making correct or justified moral decisions. A moral theory worth considering has two main aims: a theoretical aim (it should explain what makes right actions right and wrong actions wrong) and a practical aim (it should offer practical guidance for how we might arrive at correct or justified moral verdicts).
            \item A \emph{moral argument} is an argument about a moral issue
            that typically is based, explicitly or implicitly, on a moral
            theory. (Think of economic arguments; they too are usually
            based on economic theories.)
            \item John Rawls says that the main concepts of ethics are the \emph{right} and the \emph{good}. We say that a right, or good thing is \emph{valuable}. There are two different kinds of value:
            \begin{itemize}
                \item \emph{Intrinsic value}: something's being valuable because of features \emph{inherent} to it.
                \item \emph{Extrinsic value}: something's being valuable (at least partially) because of how it relates to something else that is intrinsically good.
            \end{itemize}
        \end{itemize}
    \end{spacing}
\end{document}