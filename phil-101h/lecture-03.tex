\documentclass{article}
    %%%%%%%%%%%%%%%%%%%%%%%%%%%%%%%%%%%%%%%%%%%%%%%%%%%%%%%%%%%%%
    % Lecture Specific Information to Fill Out
    %%%%%%%%%%%%%%%%%%%%%%%%%%%%%%%%%%%%%%%%%%%%%%%%%%%%%%%%%%%%%
    \newcommand{\LectureTitle}{Lecture \#3 Notes}
    %\newcommand{\LectureDate}{\today}
    \newcommand{\LectureDate}{January\ 31,\ 2017}
    \newcommand{\LectureClassName}{PHIL\ 101H}
    \newcommand{\LatexerName}{Bernardo\ Meurer}
    %%%%%%%%%%%%%%%%%%%%%%%%%%%%%%%%%%%%%%%%%%%%%%%%%%%%%%%%%%%%%

    % Change "article" to "report" to get rid of page number on title page
    \usepackage{amsmath,amsfonts,amsthm,amssymb}
    \usepackage{setspace}
    \usepackage{Tabbing}
    \usepackage{fancyhdr}
    \usepackage{lastpage}
    \usepackage{extramarks}
    \usepackage{chngpage}
    \usepackage{soul,color}
    \usepackage{graphicx,float,wrapfig}
    \usepackage{afterpage}
    \usepackage{abstract}
    \usepackage[margin=1in]{geometry}
    \usepackage{syllogism}


    % Setup the header and footer
    \pagestyle{fancy}
    \lhead{\LatexerName}
    \chead{\LectureClassName: \LectureTitle}
    \rhead{\LectureDate}
    \lfoot{\lastxmark}
    \cfoot{}
    \rfoot{Page\ \thepage\ of\ \pageref{LastPage}}
    \renewcommand\headrulewidth{0.4pt}
    \renewcommand\footrulewidth{0.4pt}

        %%%%%%%%%%%%%%%%%%%%%%%%%%%%%%%%%%%%%%%%%%%%%%%%%%%%%%%%%%%%%
    % Some tools
    \newcommand{\enterTopicHeader}[1]{\nobreak\extramarks{#1}{#1 continued on next page\ldots}\nobreak%
    \nobreak\extramarks{#1 (continued)}{#1 continued on next page\ldots}\nobreak}
\newcommand{\exitTopicHeader}[1]{\nobreak\extramarks{#1 (continued)}{#1 continued on next page\ldots}\nobreak%
   \nobreak\extramarks{#1}{}\nobreak}

\newlength{\labelLength}
\newcommand{\labelAnswer}[2]
{\settowidth{\labelLength}{#1}
\addtolength{\labelLength}{0.25in}
\changetext{}{-\labelLength}{}{}{}
\noindent\fbox{\begin{minipage}[c]{\columnwidth}#2\end{minipage}}
\marginpar{\fbox{#1}}

% We put the blank space above in order to make sure this
% \marginpar gets correctly placed.
\changetext{}{+\labelLength}{}{}{}}

\setcounter{secnumdepth}{0}
\newcommand{\TopicName}{}
\newcounter{TopicCounter}
\newenvironment{Topic}[1][Problem \arabic{TopicCounter}]
{\stepcounter{TopicCounter}
\renewcommand{\TopicName}{#1}
\section{\TopicName}
\enterTopicHeader{\TopicName}}
{\exitTopicHeader{\TopicName}}

\setcounter{secnumdepth}{0}
\newcommand{\ExampleSectionName}{}
\newcounter{ExampleSectionCounter}[TopicCounter]
\newenvironment{ExampleSection}[1][Example \arabic{ExampleSectionCounter}]
{\stepcounter{ExampleSectionCounter}
\renewcommand{\ExampleSectionName}{#1}
\section{\ExampleSectionName}
\enterTopicHeader{\ExampleSectionName}}
{\exitTopicHeader{\ExampleSectionName}}

\setcounter{secnumdepth}{0}
\newcounter{ExampleBoxCounter}[TopicCounter]
\newcommand{\examplebox}[1]
{
% We put this space here to make sure we're disconnected from the previous
% passage
\stepcounter{ExampleBoxCounter}
\noindent\fbox{\begin{minipage}[c]{\columnwidth}#1\end{minipage}}\enterTopicHeader{\ExampleSectionName}\exitTopicHeader{\ExampleSectionName}\marginpar{\fbox{\#\arabic{ExampleBoxCounter}}}
% We put the blank space above in order to make sure this
% \marginpar gets correctly placed.
\vskip10pt%
}

\renewcommand{\contentsname}{{\normalsize Topics Covered}}
\renewcommand{\abstractname}{\LectureTitle\ Summary}
\renewcommand{\absnamepos}{flushleft}

%%%%%%%%%%%%%%%%%%%%%%%%%%%%%%%%%%%%%%%%%%%%%%%%%%%%%%%%%%%%%
\begin{document}
\begin{spacing}{1.2}
    \newpage
    \section{Sexual Morality}
    \begin{itemize}
        \item How the vatican argues against non-marital sex
              \begin{itemize}
                  \item Assuming Natural Law we have that Sex is permissible only in the context where commitment is guaranteed (because of human sociability). And then we have
                  \item \syllog{Non-marital sex cannot ``guarantee'' the sincerity and fidelity of the relationship.}{Marriage can guarantee these things}{Sex outside of marriage is not permissible (that is, morally wrong)}
              \end{itemize}
        \item The Vatican's main argument against homosexuality
              \begin{itemize}
                  \item Assuming Natural Law Theory we have
                  \item \syllog{Sexual activity that violates proper function of the sex organs is morally wrong}{Homosexual activity violates the proper function of the sex organs}{Homosexual activity is wrong}
                  \item Note that this isn't an argument against \emph{homosexuals}, but rather an argument against homosexual \emph{activity}. It is perfectly fine to be a celibate homosexual (c.f. Michelangelo).
              \end{itemize}
        \item The Vatican also argues against the morality of masturbation, in much the same way it argues against homosexual activity.
        \item Notice the reliance on the idea that the sexual organs have a \emph{proper function}, and therefore, sexual activity does as well.
        \begin{itemize}
            \item But do they? If so, does the Vatican correctly identify that function?
        \end{itemize}
        \item Harm-based arguments against homosexuality
        \begin{itemize}
            \item Corvino classifies arguments to those that appeal to it being \emph{harmful} and those that appeal to its being \emph{unnatural}.
            \item Are the allegations of harm accurate?
            \item Are the alleged harms caused by homosexuality itself, or some extrinsic factor?
            \item What follows?
            \item Are the risks in a given case worth it?
        \end{itemize}
        \item Arguments that homosexuality is unnatural
        \begin{enumerate}
            \item Unnatural = unusual or abnormal
            \begin{itemize}
                \item Bad form: Being unusual doesn't imply it being wrong!
            \end{itemize}
            \item Unnatural = not practiced by other animals
            \begin{itemize}
                \item Bad content: Many other animals display homosexual behavior
                \item Bad form: Other animals not doing something doesn't imply it being wrong. (e.g. Playing piano, doing mathematics)
            \end{itemize}
            \item Unnatural = is not in accordance with innate desires
            \begin{itemize}
                \item Bad form: Just because something is in accordance to your innate desires that doesn't mean it's right (e.g. Psychopaths)
            \end{itemize}
            \item Unnatural = violates an organ's principal purpose
            \begin{itemize}
                \item Bad content: Must organs have purpose? Do organs have purpose? What purpose is it even and why?
            \end{itemize}
            \item Unnatural = disgusting or offensive
            \begin{itemize}
                \item Something being offensive/disgusting has nothing to do its morality.
            \end{itemize}
        \end{enumerate}
    \end{itemize}
\end{spacing}
\end{document}