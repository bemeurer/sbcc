\documentclass[12pt,letterpaper]{article}
    \usepackage[utf8]{inputenc}
    \usepackage[english]{babel}
    \usepackage{ifpdf}

    \usepackage{csquotes}
    \usepackage[colorinlistoftodos]{todonotes}
    \usepackage[margin=1in]{geometry}
    \usepackage[hidelinks]{hyperref}
    \urlstyle{same}
    \usepackage{setspace}
    \usepackage[notes,backend=biber]{biblatex-chicago}
    \bibliography{essay}

    \title{On the Morality of Digital Rights Management}
    \author{Bernardo Meurer}
    \date{April 26, 2018}

\begin{document}
\maketitle
\newpage
\begin{abstract}
	With the introduction of purely digital media formats such as WAVE for audio, and WebM for video, enforcement of proper usage of the media by the rights holder have become increasingly difficult.
	Since digital media allows for copying with no degree of degradation, and with relative ease compared to previous analog media formats, there have been growing levels of piracy, which constitutes an obvious infringement on the property rights of the copyright owner.
	In response to this, holders have created so-called Digital Rights Management (DRM) schemes, which aim to enforce proper usage at any cost. DRM, however, infringes on the rights of users, and of the Public as a whole, and is inherently immoral, as this analysis will show.
\end{abstract}
\onehalfspacing{}
\section{DRM}
Intellectual works have always been bound to a physical medium. The Digital Era began when this changed, and copyrighted works transitioned into the virtual, making it possible for them to be copied freely and without degradation, which represents a grave threat to the rights of creators and to market stability\autocite[\pno~117]{favale-2014}. In response to this threat, and in an attempt to halt unwarranted copying of copyrighted material, copyright holders created DRM.\@ Digital Rights Management aims to guarantee that certain digital content can only be used by someone insofar as the copyright owner allows them to. Practically, this means that DRM attempts to guarantee that the video being streamed from Netflix goes into your screen and subsequently to your eyes, and not to some form of recording device. Another practical example of DRM is the Content Scrambling System (CSS) found in DVDs, which stops users from copying the content of the disc into their computers. Here an issue already becomes evident due to the fact that, despite one having purchased and by all means owning the DVD, one cannot do with as they please.

Enforcing DRM, however, has been shown to be fundamentally infeasible, since there is no way for software alone to determine whether or not a camera is being pointed at the screen in order to copy the media on display. Furthermore, even the more feasible attempts, such as ensuring the output device is a bona-fide screen and not a capture device, have failed. Several companies and standards' bodies have proposed and implemented a plethora of DRM schemes, none of which stood the test of time, usually getting cracked within a year of release. DRM's ultimate failure manifests itself in the fact that not one single movie, song, or video game released in the past decade that reached a minimal degree of success is unavailable on the web free of charge. It is safe to say that if the entertainment industry, with all its resources and manpower, could not come up with a reliable implementation of DRM after over 20 years of constant development, then DRM, at least as intended by them, is not achievable.
y
\section{Property}
According to John Locke, ``The Labor of one's Body and the Work of his Hands, we may say, are properly his.''\autocite[\pno~287]{locke-1988} From this notion, that one appropriates something from the common by applying labor onto it, common law established the concept of property and the rights that come with it. Lockean property rights, however, don't protect just the laborer, be it an author or a hunter, but also the remaining public. While Locke grants the owner, and him alone, the rights of
use, transfer, and exclusion of his property, it also gives the public rights to the remaining common so that the community can thrive.\autocite[\pno~119]{favale-2014} This falls from the idea that, according to Locke, appropriation is only valid insofar as it leaves ``enough, and as good''\autocite[\pno~291]{locke-1988} of the common to the community. As an example of this principle, we can imagine a gatherer who is nourished by the apples he picked in the woods. Nobody can deny that the
apples are his, and yet when did they become his? According to Locke, if the first gathering did not make them the man's property, then nothing could; labor put a distinction between them (the apples) and the common.\autocite[\pno~288]{locke-1988}

Intellectual property has a peculiar characteristic, unlike material property; it lacks extension (it is physically inexhaustible). Because of this, in order to apply Lockean principles to intellectual property, and show copying as being analogue to stealing, we must appeal to Hegel. In his work ``Elements of the Philosophy of Right,'' Hegel will argue that property is fundamentally more than the result of one's labor; it is rather the expression of Man's personality\autocite[\ppno~84--88]{hegel-1991}. It is unquestionable that when one composes a song or writes a book, he is expressing his personality; not only in the will to do the work (as would be the case of the gatherer) but also in the resulting work itself. This is to say that when Shakespeare writes a piece, his personality is expressed not only in the exercise of his will while writing, but also in the piece itself. With this, and drawing a parallel between Lockean and Hegelian philosophy, one can firmly state that intellectual property is just another form of property, since it's an expression of its author's personality through and through.

As Wendy Gordon shows, copying, even though it doesn't deprive the owner of his creation, can harm him. Take, for example, a musician who composes not only for the sake of her own listening and pleasure, but also with the purpose of feeding herself by means of the royalties to her work. A bootleg copyist will undoubtedly harm her just as severely as if he had taken the physical sheet music from her, or stole the food she had bought.\autocite[\pno~1548]{gordon-1993} Therefore, in the same way, one could not eat of the gatherer's apples, for they were his property, one also cannot copy from others' intellectual creations, for they are also solely theirs. From this follows that there must be some form of anti-copying protection applied to intellectual property in order to safeguard the author's rights. Said protection is what modern law calls copyright, and it is that which DRM attempts to enforce on the digital realm.

\section{The Commons}
Western society is fundamentally based on strong property rights; the concept of appropriation and the enforcement of said rights are the basis of modern capitalism. Property rights grant the holder a range of privileges (transfer, use, and so on), but it also grants the public a set of complementary rights, namely some protection from a malicious party, as a rights holder can only constrain others from using his property insofar as that does not impair a community's ability to
thrive\autocite[\ppno~1555--1560]{gordon-1993}. Hence an individual's property should not harm the public's access to the common. Moreover, in Lockean theory the interests of the community must prevail over those of the individual when they clash\autocite[\pno~120]{favale-2014}, this reveals something fundamental, and often overlooked, about copyright, as Neil Netanel points out, ``while copyright may operate in the market, copyright's fundamental goals are not of the market.''\autocite[\pno~341]{netanel-2001} Copyright, therefore, has goals which go beyond the market itself; in other words, it hopes to achieve things other than market safety and profit protection. Natanel goes on to claim that copyright is a vehicle of democracy, pointing once more towards interpreting copyright as protecting the public first, and the author second. Furthermore, the first copyright legislation, the Copyright Act 1710, had as its subtitle ``An Act for the Encouragement of Learning \ldots''\autocite{wortley-1710}, meaning that since its inception copyright had the public's interest in mind primarily. With this it becomes clear that copyright goes far beyond both markets and authors, enlightening the fundamental flaw of DRM, which is its ignorance towards the rights of the remaining public.

As previously outlined, Lockean philosophy constrains one's property to leave ``enough and as good''\autocite[\pno~291]{locke-1988} to the public. While here, ``enough and as good'' is referring to the common, by which Locke meant that which Nature provides us, this concept can easily be generalized to apply to the immaterial as well. Fundamentally, this means that in the same way there is a common of things, given to us by Nature, there is also a common of intangibles, ideas, that every individual must be able to use freely. As Gordon points out, ``Our common encompasses not only our physical country but our culture as well''\autocite[\pno~1555]{gordon-1993}. An example of a non-physical common of ideas is public domain, which is composed of works that never were, or that no longer are, protected by copyright. It follows from this that no appropriation can be made of that which is in the common of ideas, and therefore no copyright can be claimed. Consider, for example, the happy birthday song. It is absurd to imagine that someone today could rightfully claim ownership of the melody and proceed to receive royalties from its use; the song is an integral part of American culture after all. The reason why no royalties are applied to it, or to any number of older works, is because they belong to the public domain (the common of ideas), since they either were never copyrighted, or have had their rights expire.

\section{DRM \& Ethics}
From an ethical perspective, the question on the morality of DRM is equivalent to analyzing how, and when, is it wrong to appropriate from the commons. It is trivial, by definition, that DRM appropriates from the common, and it follows from it's restrictions, outlined in the previous section, that it is an overarching appropriation. Why, then, is it wrong to broadly appropriate the commons? Since this paper aims for a survey-style approach, multiple answers shall be given to this question.

\subsection{Natural Law Theory}
Natural Law Theory, at least as proposed and outlined by Aquinas, relies on the claims that there are four basic intrinsic goods: Human life, Human procreation, Human knowledge, and Human sociability.
An action, therefore, is right if and only if it does not directly violate any of these basic values. There is, however, a secondary rule that applies to moral dilemmas. Namely, an action that causes at least one evil, and one good, effect is morally permissible if and only if: it's necessary, it's proportional, it's evil effect is non-intentional, and it has intrinsic permissibility; this is called the Doctrine of Double Effect (DDE).

When Locke speaks of allowing a community to thrive\autocite[\pno~119]{favale-2014}, it is simple to frame it in the eyes of Natural Law Theory. Allowing a community to thrive, therefore, is not performing an action that goes against the four basic intrinsic goods. In this sense, however, it becomes easy to argue that DRM \emph{does} infringe on the basic goods, since when we appropriate the commons we:

\begin{enumerate}
    \item Harm Human knowledge
    \item Harm Human sociability
\end{enumerate}

Statement 1.\ is clear; the \emph{raison d'être} for DRM is to limit access to some information only to a portion of society. The entire point of DRM is to be an über-enforcer of copyright, \emph{fiant argentum et pereat mundus}.
Statement 2.\ follows from the idea that genuine human interactions rely on a minimum intellectual common-ground. Since DRM's effects are heterogeneous across the socio-cultural strata, it deepens the interactive gap between said strata, reinforcing social
segregation and, to a degree, prejudice. It is clear that DRM's appropriation of the common is not applicable to the DDE, it's necessity is dubious, it's most definitely not proportional, and it's evil effect is intentional a-priori. According to Natural Law Theory, therefore, DRM is thoroughly immoral, infringing deeply on two intrinsic goods.

\subsection{Kantian Theory}
Kantian moral theory revolves around two equivalent formulations, the Universal Law Formulation and the Humanity Formulation, respectively:
\begin{itemize}
    \item \textbf{UL}: Act only in accordance with that maxim through which you can at the same time will that it become a universal law.\autocite{sep-kant-moral}
    \item \textbf{H}: Never act in such a way as to treat humanity, whether in ourselves or in others, as a means only but always as an end in itself.\autocite{sep-kant-moral}
\end{itemize}

Now, looking at the UL formulation, let's suppose we made of appropriation, and denial of access as is intentional of DRM, of the commons a universal law. By that logic, then, all cultural creations made by humanity, all intellectual material produced by us, would have been locked shut on a vault immediately after inception, with access only being allowed to a lucky few, for an unsure amount of time.
Since the human development is cross-referential and re-interpretative by nature, this would greatly harm any possible advances, to the point which society would perhaps be hundreds or thousands of years behind what it currently is.
It is completely unreasonable to desire such a thing, to will as a universal law that humanity not advance and produce knowledge, and therefore it follows that according to the UL formula, that appropriating the commons in the capacity which DRM does is morally unacceptable.

Using the Humanity formula the argument is significantly simpler, as mentioned before, DRM's motto could be seen as \emph{fiant argentum et pereat mundus}, which is a clear usage of others as means alone. The only point of DRM is to severely limit access to the commons, and in doing so increase all possible gains from the retained information/data; it gives nothing back, provides no benefits.

It is safe to say, therefore, that according to Kantian theory DRM is deeply flawed, both in it's intentions and implementations; as a universal law it severely harms human development, as an interaction with Humanity it only takes advantage of it.

\subsection{Consequentialism}
Act consequentialism is the claim that an act is morally right if and only if that act maximizes the good, that is, if and only if the total amount of good for all minus the total amount of bad for all is greater than this net amount for any incompatible act available to the agent on that occasion.\autocite{sep-consequentialism}
The vagueness in the consequentialist term ``amount of good'' means there are a plethora of different sub-theories, to avoid verbosity we will take a higher-kinded\footnote{In a type-theoretic sense.} approach to the argument.

It is trivial that when one makes an overarching appropriation of the commons that an unbalance is created, you deprive the remainder of humanity of what they need to thrive, intellectually or otherwise. In the case of the common of intangibles, an improper appropriation damages everyone but the holder, the greatest possible difference in ``amount of good.''
DRM, therefore, since it aims to deny access to information to everyone but a few, with no concern for necessity of it, is morally impermissible. Moreover, there are plenty of businesses who thrive in the digital world \emph{without} the use of DRM, so there are better compatible acts available to the agents using DRM.\@

\subsection{Virtue Theory}
According to Virtue Theory, an action is right if and only if (and because) it is what a virtuous agent would not avoid doing in the circumstances under consideration.\autocite[\pno~25]{timmons_2017} Therefore, it suffices to show that improperly appropriating the commons, as DRM does, is something a virtuous agent would avoid doing.

We know that upon creating some intellectual property there is nothing wrong with intending to harvest money from it 

\section{Conclusion}

\newpage
\printbibliography[heading=bibintoc]
\end{document}
