\documentclass{article}
    %%%%%%%%%%%%%%%%%%%%%%%%%%%%%%%%%%%%%%%%%%%%%%%%%%%%%%%%%%%%%
    % Lecture Specific Information to Fill Out
    %%%%%%%%%%%%%%%%%%%%%%%%%%%%%%%%%%%%%%%%%%%%%%%%%%%%%%%%%%%%%
    \newcommand{\LectureTitle}{Lecture \#4 Notes}
    %\newcommand{\LectureDate}{\today}
    \newcommand{\LectureDate}{February\ 05,\ 2017}
    \newcommand{\LectureClassName}{PHIL\ 101H}
    \newcommand{\LatexerName}{Bernardo\ Meurer}
    %%%%%%%%%%%%%%%%%%%%%%%%%%%%%%%%%%%%%%%%%%%%%%%%%%%%%%%%%%%%%

    % Change "article" to "report" to get rid of page number on title page
    \usepackage{amsmath,amsfonts,amsthm,amssymb}
    \usepackage{setspace}
    \usepackage{Tabbing}
    \usepackage{fancyhdr}
    \usepackage{lastpage}
    \usepackage{extramarks}
    \usepackage{chngpage}
    \usepackage{soul,color}
    \usepackage{graphicx,float,wrapfig}
    \usepackage{afterpage}
    \usepackage{abstract}
    \usepackage[margin=1in]{geometry}
    \usepackage{syllogism}


    % Setup the header and footer
    \pagestyle{fancy}
    \lhead{\LatexerName}
    \chead{\LectureClassName: \LectureTitle}
    \rhead{\LectureDate}
    \lfoot{\lastxmark}
    \cfoot{}
    \rfoot{Page\ \thepage\ of\ \pageref{LastPage}}
    \renewcommand\headrulewidth{0.4pt}
    \renewcommand\footrulewidth{0.4pt}

        %%%%%%%%%%%%%%%%%%%%%%%%%%%%%%%%%%%%%%%%%%%%%%%%%%%%%%%%%%%%%
    % Some tools
    \newcommand{\enterTopicHeader}[1]{\nobreak\extramarks{#1}{#1 continued on next page\ldots}\nobreak%
    \nobreak\extramarks{#1 (continued)}{#1 continued on next page\ldots}\nobreak}
\newcommand{\exitTopicHeader}[1]{\nobreak\extramarks{#1 (continued)}{#1 continued on next page\ldots}\nobreak%
   \nobreak\extramarks{#1}{}\nobreak}

\newlength{\labelLength}
\newcommand{\labelAnswer}[2]
{\settowidth{\labelLength}{#1}
\addtolength{\labelLength}{0.25in}
\changetext{}{-\labelLength}{}{}{}
\noindent\fbox{\begin{minipage}[c]{\columnwidth}#2\end{minipage}}
\marginpar{\fbox{#1}}

% We put the blank space above in order to make sure this
% \marginpar gets correctly placed.
\changetext{}{+\labelLength}{}{}{}}

\setcounter{secnumdepth}{0}
\newcommand{\TopicName}{}
\newcounter{TopicCounter}
\newenvironment{Topic}[1][Problem \arabic{TopicCounter}]
{\stepcounter{TopicCounter}
\renewcommand{\TopicName}{#1}
\section{\TopicName}
\enterTopicHeader{\TopicName}}
{\exitTopicHeader{\TopicName}}

\setcounter{secnumdepth}{0}
\newcommand{\ExampleSectionName}{}
\newcounter{ExampleSectionCounter}[TopicCounter]
\newenvironment{ExampleSection}[1][Example \arabic{ExampleSectionCounter}]
{\stepcounter{ExampleSectionCounter}
\renewcommand{\ExampleSectionName}{#1}
\section{\ExampleSectionName}
\enterTopicHeader{\ExampleSectionName}}
{\exitTopicHeader{\ExampleSectionName}}

\setcounter{secnumdepth}{0}
\newcounter{ExampleBoxCounter}[TopicCounter]
\newcommand{\examplebox}[1]
{
% We put this space here to make sure we're disconnected from the previous
% passage
\stepcounter{ExampleBoxCounter}
\noindent\fbox{\begin{minipage}[c]{\columnwidth}#1\end{minipage}}\enterTopicHeader{\ExampleSectionName}\exitTopicHeader{\ExampleSectionName}\marginpar{\fbox{\#\arabic{ExampleBoxCounter}}}
% We put the blank space above in order to make sure this
% \marginpar gets correctly placed.
\vskip10pt%
}

\renewcommand{\contentsname}{{\normalsize Topics Covered}}
\renewcommand{\abstractname}{\LectureTitle\ Summary}
\renewcommand{\absnamepos}{flushleft}

%%%%%%%%%%%%%%%%%%%%%%%%%%%%%%%%%%%%%%%%%%%%%%%%%%%%%%%%%%%%%
\begin{document}
\begin{spacing}{1.2}
    \newpage
    \begin{itemize}
        \item Hypothetical Imperatives
              \begin{itemize}
                  \item Write a love letter
                  \item Study hard for a class
                        \begin{itemize}
                            \item You can escape this command by just changing your desire, and accepting being comfortable with a bad grade.
                        \end{itemize}
                  \item All rules of etiquette
                        \begin{itemize}
                            \item Wearing shoes
                            \item Forks on the left side of the plate
                            \item \ldots
                        \end{itemize}
                  \item You have a \emph{desire}, you recognize that a certain course of action would help you get what you desire. So you conclude that you should follow the indicated plan.
                  \item There are \emph{hypothetical} Imperatives because you can escape them by just renouncing the desire.
              \end{itemize}
        \item Categorical Imperatives
              \begin{itemize}
                  \item \emph{Moral} requirements are different; they are not hypothetical Imperatives
                  \item You can't just escape a moral command by renouncing the relevant desire.
                  \item Moral requirements are \emph{Categorical}. They are \emph{absolute}. No matter what feelings, desires, and inclinations happen to be, and even what immediate consequences of your action might be, you should, for example, try to help needy people.
              \end{itemize}
        \item What makes us moral?
              \begin{itemize}
                  \item That we have \emph{good will}.
                  \item Good will means having the right \emph{intentions}
                  \item Consider the case of Abdullah and Leon
              \end{itemize}
        \item Good will
              \begin{itemize}
                  \item How can we be sure that, like Abdullah, our will is good?
                  \item For Kant all our (free) actions are connected to intentions (motives, plans of action).
              \end{itemize}
        \item Categorical Imperative
              \begin{itemize}
                  \item Kant thinks that the right motives are derives from a principle that every rational person must accept. He calls this the \emph{Categorical Imperative}
                  \item From the Categorical Imperative that we are able to derive, we have Categorical imperatives (moral rules that we ought to follow).
                  \item Kant formulates this in several ways. We will look at only two:
                        \begin{itemize}
                            \item The Universal Law of Formulation (UL)
                            \item Humanity Formulation (H)
                        \end{itemize}
                  \item If you really think some act is moral, you should have no problem with everyone doing it. If it's a moral thing to do, shouldn't you rationally will everyone to do it?
                  \item Act only according to the maxim by which you can at the same time will that it should become a universal law.
                  \item UL --- An action is right iff one can;
                        \begin{itemize}
                            \item Consistently conceive of everyone adopting and acting on a general policy of one's action
                            \item Consistently will that everyone act on that maxim
                        \end{itemize}
              \end{itemize}
    \end{itemize}
\end{spacing}
\end{document}