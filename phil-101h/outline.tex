\documentclass[12pt,letterpaper]{article}
\usepackage[utf8]{inputenc}
\usepackage[margin=1in]{geometry}
\usepackage[notes,backend=biber]{biblatex-chicago}
\bibliography{essay}
\usepackage[hidelinks]{hyperref}
\urlstyle{same}

\title{On the Morality of Digital Rights Management}
\author{Bernardo Meurer}
\date{April 12, 2018}

\begin{document}
\maketitle
\newpage
\section{Abstract}
    With the introduction of purely digital media formats such as WAVE for audio, and WebM for video, enforcement of proper usage of the media by the rights holder have become increasingly difficult.

    Since digital media allows for copying with no degree of degradation, and with relative ease compared to previous analog media formats, there have been growing levels of piracy, which constitutes an obvious infringement on the property rights of the copyright owner.

    In response to this, holders have created so-called Digital Rights Management (DRM) schemes, which aim to enforce proper usage at any cost. DRM, however, infringes on the rights of users, and of the Public as a whole, and is inherently immoral, as this analysis will show.With the introduction of purely digital media formats such as WAVE for audio, and WebM for video, enforcement of proper usage of the media by the rights holder have become increasingly difficult.
\section{DRM}
\begin{itemize}
    \item What is DRM?\autocite{favale-2014}
    \item How does DRM work?\autocite{goldstein-1996}
    \item Examples of DRM
\end{itemize}
\section{Property}
\begin{itemize}
    \item What is property?\autocite{locke-1988}
    \item What is intellectual property?\autocite{hegel-1991}
    \item What is copyright?\autocite{wortley-1710}
\end{itemize}
\section{The Common}
\begin{itemize}
    \item The common of things.\autocite{locke-1988}
    \item The common of ideas.\autocite{locke-1988}\autocite{gordon-1993}\autocite{hegel-1991}
    \item Moral analysis of violations on the common
    \begin{itemize}
        \item Try to bring as many different theories as possible, to show that any appropriation of the common is inherently immoral, and violates others' rights.
    \end{itemize}
\end{itemize}
\section{DRM's Infringement}
\begin{itemize}
    \item Show how DRM infringes on the common, and is therefore immoral.\autocite{favale-2014}\autocite{gordon-1993}
    \item Other shortfalls of DRM:\@ It doesn't work\autocite{favale-2014}
    \item DRM is immoral.
\end{itemize}
\newpage
\printbibliography[heading=bibintoc,title={Bibliography}]
\end{document}
