\documentclass{article}
    %%%%%%%%%%%%%%%%%%%%%%%%%%%%%%%%%%%%%%%%%%%%%%%%%%%%%%%%%%%%%
    % Lecture Specific Information to Fill Out
    %%%%%%%%%%%%%%%%%%%%%%%%%%%%%%%%%%%%%%%%%%%%%%%%%%%%%%%%%%%%%
    \newcommand{\LectureTitle}{Lecture \#2 Notes}
    %\newcommand{\LectureDate}{\today}
    \newcommand{\LectureDate}{January\ 29,\ 2017}
    \newcommand{\LectureClassName}{PHIL\ 101H}
    \newcommand{\LatexerName}{Bernardo\ Meurer}
    %%%%%%%%%%%%%%%%%%%%%%%%%%%%%%%%%%%%%%%%%%%%%%%%%%%%%%%%%%%%%

    % Change "article" to "report" to get rid of page number on title page
    \usepackage{amsmath,amsfonts,amsthm,amssymb}
    \usepackage{setspace}
    \usepackage{Tabbing}
    \usepackage{fancyhdr}
    \usepackage{lastpage}
    \usepackage{extramarks}
    \usepackage{chngpage}
    \usepackage{soul,color}
    \usepackage{graphicx,float,wrapfig}
    \usepackage{afterpage}
    \usepackage{abstract}
    \usepackage{syllogism}

    % In case you need to adjust margins:
    \topmargin=-0.45in
    \evensidemargin=0in
    \oddsidemargin=0in
    \textwidth=6.5in
    \textheight=9.0in
    \headsep=0.25in

    % Setup the header and footer
    \pagestyle{fancy}
    \lhead{\LatexerName}
    \chead{\LectureClassName: \LectureTitle}
    \rhead{\LectureDate}
    \lfoot{\lastxmark}
    \cfoot{}
    \rfoot{Page\ \thepage\ of\ \pageref{LastPage}}
    \renewcommand\headrulewidth{0.4pt}
    \renewcommand\footrulewidth{0.4pt}

        %%%%%%%%%%%%%%%%%%%%%%%%%%%%%%%%%%%%%%%%%%%%%%%%%%%%%%%%%%%%%
    % Some tools
    \newcommand{\enterTopicHeader}[1]{\nobreak\extramarks{#1}{#1 continued on next page\ldots}\nobreak%
    \nobreak\extramarks{#1 (continued)}{#1 continued on next page\ldots}\nobreak}
\newcommand{\exitTopicHeader}[1]{\nobreak\extramarks{#1 (continued)}{#1 continued on next page\ldots}\nobreak%
   \nobreak\extramarks{#1}{}\nobreak}

\newlength{\labelLength}
\newcommand{\labelAnswer}[2]
{\settowidth{\labelLength}{#1}
\addtolength{\labelLength}{0.25in}
\changetext{}{-\labelLength}{}{}{}
\noindent\fbox{\begin{minipage}[c]{\columnwidth}#2\end{minipage}}
\marginpar{\fbox{#1}}

% We put the blank space above in order to make sure this
% \marginpar gets correctly placed.
\changetext{}{+\labelLength}{}{}{}}

\setcounter{secnumdepth}{0}
\newcommand{\TopicName}{}
\newcounter{TopicCounter}
\newenvironment{Topic}[1][Problem \arabic{TopicCounter}]
{\stepcounter{TopicCounter}
\renewcommand{\TopicName}{#1}
\section{\TopicName}
\enterTopicHeader{\TopicName}}
{\exitTopicHeader{\TopicName}}

\setcounter{secnumdepth}{0}
\newcommand{\ExampleSectionName}{}
\newcounter{ExampleSectionCounter}[TopicCounter]
\newenvironment{ExampleSection}[1][Example \arabic{ExampleSectionCounter}]
{\stepcounter{ExampleSectionCounter}
\renewcommand{\ExampleSectionName}{#1}
\section{\ExampleSectionName}
\enterTopicHeader{\ExampleSectionName}}
{\exitTopicHeader{\ExampleSectionName}}

\setcounter{secnumdepth}{0}
\newcounter{ExampleBoxCounter}[TopicCounter]
\newcommand{\examplebox}[1]
{
% We put this space here to make sure we're disconnected from the previous
% passage
\stepcounter{ExampleBoxCounter}
\noindent\fbox{\begin{minipage}[c]{\columnwidth}#1\end{minipage}}\enterTopicHeader{\ExampleSectionName}\exitTopicHeader{\ExampleSectionName}\marginpar{\fbox{\#\arabic{ExampleBoxCounter}}}
% We put the blank space above in order to make sure this
% \marginpar gets correctly placed.
\vskip10pt%
}

\renewcommand{\contentsname}{{\normalsize Topics Covered}}
\renewcommand{\abstractname}{\LectureTitle\ Summary}
\renewcommand{\absnamepos}{flushleft}

%%%%%%%%%%%%%%%%%%%%%%%%%%%%%%%%%%%%%%%%%%%%%%%%%%%%%%%%%%%%%
\begin{document}
\begin{spacing}{1.2}
    \newpage
    \section{Natural Law Theory}
    \begin{itemize}
        \item ``There are objectively true moral principles that are grounded in Human Nature''
        \item There are objective facts about human nature that determine what human good consists in.
        \item St.\@Thomas Aquinas (1225--1274)
              \begin{itemize}
                  \item ``Good is to be done, evil is to be avoided''
                  \item The natural inclinations of human beings indicate which ends are intrinsically valuable.
                  \item These inclinations are a part of human nature. Thus morality is ultimately grounded in facts about human nature.
                  \item There are four basic intrinsic goods:
                        \begin{itemize}
                            \item Human life
                            \item Human procreation
                            \item Human knowledge
                            \item Human sociability
                        \end{itemize}
                  \item To destroy any intrinsic good is intrinsically bad (evil).
                  \item \textbf{NLT} --- An action is right if and only if (and because) in performing the action one does not directly violate any of these basic values.
                  \item For instance, \emph{killing} a human is usually wrong because it destroys humans human life.
                  \item Immoral too is the use of contraception because it inhibits human procreation.
                  \item Lying (almost always) undermines the human good of knowledge.
                  \item What is something that usually harms human sociability.
                        \begin{itemize}
                            \item Theft. Bullying. Discrimination.
                        \end{itemize}
                  \item Doctrine of Double Effect (DDE)
                  \item \textbf{DDE} --- An action that would cause at least one evil effect and at least one good effect is morally permissible if and only if all of the following conditions are met:
                        \begin{itemize}
                            \item Intrinsic permissibility
                            \item Necessity
                            \item Non-intentionality
                            \item Proportionality
                        \end{itemize}
              \end{itemize}
        \item What if, no matter what action you choose to do, you will violate one of the basic values and so cause some bad in the world?
              \begin{itemize}
                  \item E.g. When a pregnant woman's life is put in danger by her fetus?
              \end{itemize}
        \item According to many natural law proponents, \emph{both} alternatives (abortion \emph{and} going through with pregnancy) will bring about the destruction of human life.
        \item Criticisms
              \begin{itemize}
                  \item It's not always straightforward to apply
                        \begin{itemize}
                            \item What counts as human knowledge?
                            \item What counts as human life? (A 2-week old fetus?)
                        \end{itemize}
                  \item How do we apply the DDE correctly?
                        \begin{itemize}
                            \item Is the death penalty proportional to murder?
                            \item What's proportional to rape?
                        \end{itemize}
              \end{itemize}
    \end{itemize}
    \section{Sexual Morality}
    \begin{itemize}
        \item You can be liberal about moral issues, but conservative about economic issues, and conversely.
        \item You can even be liberal about certain moral issues, but conservative about others.
        \item On any moral issue, there is no one liberal, moderate, or conservative position.
        \item Vatican declaration on some questions of sexual ethics
              \begin{itemize}
                  \item The Vatican Declaration is hugely influenced by Aquinas' claim that one of the four basic human goods is \emph{human procreation}.
                  \item It assumes that an action is wrong if it is \emph{unnatural} (That it goes against \emph{proper function}).
                  \item Approved by Pope Paul VI.\@
                  \item The Vatican claims there are two sources of moral knowledge:
                        \begin{itemize}
                            \item Natural
                            \item Supernatural
                        \end{itemize}

              \end{itemize}
    \end{itemize}
\end{spacing}
\end{document}