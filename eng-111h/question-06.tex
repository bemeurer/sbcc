\documentclass[12pt,letterpaper]{article}
    \usepackage[utf8]{inputenc}
    \usepackage{ifpdf}
    \usepackage{mla}
    \usepackage[pass]{geometry}
    \usepackage{todonotes}
    \usepackage[hidelinks]{hyperref}

    \urlstyle{same}

\begin{document}
\begin{mla}{Bernardo}{Meurer}{Professor Eckford-Prossor}{English 111 H}{February 14, 2018}{Technopoly Chapter 9}
    `` It's just a coincidence that Smith is looking at this broken clock at one of the two rare moments it isn't wrong. By looking at the broken clock, does Smith come to \emph{know} that it is 1:17?'' (Nagel 46)
    \vspace{10pt}
    I feel like the true point made here is that knowledge, when founded in empiricism, is never certain. Even if we are to restrict it by saying ``It is certain if we verify it,'' which in the above case would have been Smith checking his wristwatch, or anything of the kind, which wouldn't have helped. More over, were Smith to have been assured that the station clock had been calibrated, the question of whether the calibration is accurate remains. Did they use another watch, known to be correct? How do they know that reference watch is indeed true to time? Did they use the sun? How can we be certain that at the time of calibration some extraordinary astronomical event happened that caused the sun to look as if it were in a different position than it truly is? Empiricism is never base for certainty, I'm afraid.
    \begin{workscited}
        \bibent Nagel, Jennifer. Knowledge: a Very Short Introduction. Oxford University Press, 2014.
    \end{workscited}
\end{mla}
\end{document}