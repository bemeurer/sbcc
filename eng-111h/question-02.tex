\documentclass[12pt,letterpaper]{article}
    \usepackage[utf8]{inputenc}
    \usepackage{ifpdf}
    \usepackage{mla}
    \usepackage[pass]{geometry}
    \usepackage{todonotes}
    \usepackage[hidelinks]{hyperref}

    \urlstyle{same}

\begin{document}
\begin{mla}{Bernardo}{Meurer}{Professor Eckford-Prossor}{English 111 H}{January 29, 2018}{Technopoly Chapters 2--3}
    ``To every Old World belief, habit, or tradition, there was and still is a technological alternative.  To prayer, the alternative is penicillin; to family roots, the alternative is mobility; to reading, the alternative is television; to restraint, the alternative is immediate gratification; to sin, the alternative is psychotherapy; to political ideology, the alternative is popular appeal established through scientific polling.'' (Postman 54)
    \vspace{10pt}


    I think this passage reflects well on the idea of the ``disenchantment of the world.'' worked by Schiller and Weber. It shows very clearly the replacement of belief and tradition by scientific understanding in all its forms. This is a staple of our secular, bureaucratic, Western society; and it begs the question, which is the subject matter of the book itself, of how does the disenchantment affect us on a Wittgensteinian level. How do the new ways with which we comprehend the world themselves change us, beyond what they allow us to experience?

    \begin{workscited}
        \bibent  Postman, Neil. \emph{Technopoly: the Surrender of Culture to Technology}. Vintage Books, 1993.
    \end{workscited}
\end{mla}
\end{document}