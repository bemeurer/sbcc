\documentclass[12pt,letterpaper]{article}
    \usepackage[utf8]{inputenc}
    \usepackage{ifpdf}
    \usepackage{mla}
    \usepackage{todonotes}
    \usepackage[pass]{geometry}
    \usepackage[hidelinks]{hyperref}

    \urlstyle{same}

\begin{document}
\begin{mla}{Bernardo}{Meurer}{Professor Eckford-Prossor}{English 111 H}{February 5, 2018}{Technology as a cultural byproduct}

    In ``Technopoly: the Surrender of Culture to Technology'' Neil Postman argues vehemently for his thesis that we, as a society, have surrendered culture to technology. What Postman, for all his vision and geniality, seems oblivious to, however, is the deep intertwining of culture and technology in its proper aspect. While his depiction is that of a timeless clash between culture and technology, the truth is that the latter is inherently a byproduct of the former. Technology is, through and through, inherently cultural.

    In order to argue consistently about the subject, one must carefully define and outline the terminology in place. Firstly, we define culture as being ``the integrated pattern of human knowledge, belief, and behavior\ldots'' (Merriam-Webster). Here, it is important to take a wide, but not cumbersome, definition of knowledge. Perhaps the most appropriate one would come from Jennifer Nagel's ``Knowledge: a Very Short Introduction'', and is simply to say that knowledge is information which is known by someone (Nagel 2). To the scope of this essay, and as far as we can tell, this means information that is known to Humans. Although this last definition might feel like it commits a circular dependency, since we use ``known'' to define ``knowledge'', it is reliable given the restriction made on things being known to humans.This creates a disconnect between ``knowing'' and ``knowledge'', represented, for example, in the fact that an animal's knowing of the presence of a predator is not knowledge, despite it being known, for knowledge is purely human, and therefore no recursion takes place.

    Secondly, we must define technology just as carefully. We should approach it's definition very broadly, which shields us from omitting facets of the term. With that, we say that technology is ``a capability given by the practical application of knowledge'' (Merriam-Webster). This is to say that technology is simply the materialization of knowledge, to a degree it is the result of the interfacing between humans and knowledge. Although here it already becomes visible that technology cannot exist without culture, a more careful proof shall follow.

    To begin the proof, we must show that for every technology that becomes popular (succeeds), there is beforehand a cultural setting that calls for that technology to exist at all, and that this is true independent of time, which is to say it is the case today as it was a thousand years ago, and before that still. If we look at the case of Snapchat, for example, we can very easily dissect what led to its overwhelming success. Modern, especially Western, culture has ingrained in its core the vice of narcissism, there is a constant, overwhelming, cult to the image of the self. With smartphones and front-facing cameras the public, reflecting it's cultural setting, created the concept of a ``selfie,'' a picture of oneself without the need of context, landscape, setting, or occasion. This new, inherently cultural, form of media lacked a proper medium for its exchange, sending ``selfies'' to others on ``normal,'' (contextual, persistent, chronological) messaging mediums is awkward, and because of this, and to fill this void created by culture itself, Snapchat was envisioned, and for the same reasons succeeded.

    Another example to be looked at is that of the loom, which is the machine used to manufacture fabrics. If it was not customary (cultural) for societies to wear fabric clothes, then the loom probably would never have been invented, for there was no cultural pressure to incentivize it, and had it been invented it most certainly would not have spread to the omnipresence we know of today, for there would have been no pressure to sustain it. This is to say that the ails that (successful) technology eases predate the technology, and are caused by (societal) cultural decisions.

    One can quite easily look at the previous case studies and claim that there is a form of ``recursion fallacy'' being committed, since for the ``selfie'' to be culturally conceptualized in the first place the smartphone, yet another technology, had to be created. To this, however, the exact same point can be (continuously) applied, the smartphone is a cultural byproduct and it fills a gap generated by culture. From a higher-kinded approach, the recursion exposed above can be seen as a product of the same lexico-semantical parametric form of the well known ``Which came first, the chicken or the egg?'' problem, so called causality dilemmas. Therefore, the entire discussion can be summarized into a discussion of whether technology predates culture, or vice-versa, in which case it suffices to show that humans are inherently and necessarily cultural to prove by induction that technology is then a byproduct of this cultural dimension of humanity.

    Another interesting flaw in Postman's argument which is auxiliary to this proof, lies in his analysis of the myth of Theuth (Plato 38). Postman points the fault in Thamus' argument to the fact that he ignores the benefits that writing would bring to society. While this is true, it also misses the deeper and more important omission made by Thamus: That society needed writing.As society developed, so did the amount of accumulated knowledge, and there is a physical limit to how much one can reliable know in one's own head. It would be impossible for society to advance much further, once this limit was nigh, and because of this society required writing. The point being that writing's downside, that it disincentivizes memory, was the point in itself, reliance on memory was no longer enough. Culture had grown too large for the mind of the individual, and so it petitioned technology a solution, which was writing.

    Again, culture is a union of three sets, (human) knowledge, beliefs, and behavior. As Descartes and may others famously showed, one knows beyond any doubt and without requirements of the existence of the self, for to become conscious of the question requires itself existence (Descartes). It is then, fair to say that humans have a priori knowledge of their own existence. It is also trivial to conclude that humans have a priori beliefs of their surroundings (that they have a body, that their vision is truthful, etc.) Finally, it is biologically unquestionable that humans have a set of inherent behaviors (instincts.) This is to say, however, that humans have culture a priori, for if the elements that compose it are inherent then it is itself inherent (assuming the consistency of the Axiom of Choice in ZFC.) With this result, it becomes clear that culture itself predates, and therefore defines, technology.

    Culture has not, and could not have, surrendered itself to technology. Technology is simply a facet of culture itself, one of its many drapings. Postman's error lies in his borderline anthropomorphic depiction of Technology, as if it were a united, sentient form somehow that attacks culture. In his attempt to value human culture, he himself left out humanity from the picture, treating technology as if it somehow wasn't our creation. The technological struggle is always performed by humans and, given that humans are deeply embedded in culture, this transmits to and shapes technology.

    \begin{workscited}

        \bibent Descartes, René, et al. ``Meditations on First Philosophy with Selections from the Objections and Replies''. Cambridge University Press, 2017.

        \bibent Merriam-Webster, “Culture”, Merriam-Webster, \url{www.merriam-webster.com}.

        \bibent Merriam-Webster, “Technology”, Merriam-Webster, \url{www.merriam-webster.com}.

        \bibent Nagel, Jennifer. ``Knowledge: a Very Short Introduction''. Oxford University Press, 2014.

        \bibent Plato. ``Phaedrus'', Infomotions, Inc., 2000. ProQuest Ebook Central, \url{www.ebookcentral.proquest.com/lib/sbcc-ebooks/detail.action?docID=3314645}.

        \bibent Postman, Neil. ``Technopoly: the Surrender of Culture to Technology''. Vintage Books, 1993.
    \end{workscited}
\end{mla}
\end{document}