\documentclass[12pt,letterpaper]{article}
\usepackage[utf8]{inputenc}
\usepackage[pass]{geometry}
\usepackage{mla}
\usepackage[hidelinks]{hyperref}

\urlstyle{same}

\begin{document}
\begin{mla}{Bernardo}{Meurer}{Professor Eckford-Prossor}{Honors 100}{\today}{For The Right to Be Forgotten}
    It is generally agreed that the condition of being human comes attached with a set of inherent rights, which every human should dispose of.
    Since 1948, the most widely agreed upon set of said rights is The Universal Declaration of Human Rights (UHDR).
    The UHDR is so ubiquitous that it's become synonymous with Human Rights, it's the de-facto global standard.
    Composed of just 30 short articles, the Declaration is rather diminutive, and focuses on the ``undeniable'' rights, and because of that it is grossly insufficient when it comes to privacy.

    Article 12 of the UHDR gives us the only provisions in it for privacy, and it's simply stated as ``No one shall be subjected to arbitrary interference with his privacy, family, home or correspondence, nor to attacks upon his honour and reputation. Everyone has the right to the protection of the law against such interference or attacks.'' Having it be so broad and unspecific means it fails at a fundamental level to provide meaningful protections to people's privacy.
    Moreover, given that the UHDR was written at a time where the concept of a ``digital persona'' did not exist, it also fails to protect our data, which is an inherent part of our self today. Despite the general aversion of this concept, there is no denying that we are one with our data today, and yet this part of ourselves is not protected under the UHDR.

    With this in mind, it is clear that a new, expanded, set of rights must be created, referring to the rights of one's data. Therefore we can propose the Universal Data Rights (UDR):
\begin{enumerate}
    \item Any and all data collection must be explicitely consented by the user, or their guardian in case of a child.
        \begin{enumerate}
            \item Consent must be opt-in; opt-out does not qualify as consent. The user must choose to share date.
            \item Consent, to be valid, must be in accordance to all expected regulations relating to accountability from the part of the user.
        \end{enumerate}
    \item Data stored on a person must be pseudonymized.
    \item Data collected and retained on a user must be kept to the minimum necessary for the controller to operate.
    \item Data may not be processed unless there is an explicit, lawful reason to do so.
    \item Everyone has the right to request their personal data, to know how it is being processed, who it is being shared with, and how it was obtained.
    \item Everyone has a right to be forgotten. One has the right to request complete and immediate deletion of any data collected of them; the fundamental rights of the individual trumps the genuine lawful interest of the controller.
    \item One has the right to transfer their personal data from one processing system to another, without interference from the controller.
\end{enumerate}
These 7 basic rights should be universal, and guarantee fundamental protections to people's data, and to their indentity. In Article 2, regarding pseudonymisation, it guarantees that resulting data from processing cannot be linked to a subject without the use of external information, which is paramount for guaranteeing privacy.

The justification, and intent, with each article is as follows 
\begin{enumerate}
    \item To guarantee that everyone knows how, when, and if their personal data is being collected.
    \item To assure that personal data, and it's final processed form are dissociated as much as possible, for security.
    \item To ensure only necessary data is collected, protecting individuals from misusage of their data.
    \item To safeguard individuals from unknown and improper analysis/processing of their data.
    \item To guarantee that everyone can know exactly what is known on them, how that information is being used, and who is benefitting from it.
    \item To ensure that people have control over their data, which is to say over themselves, by allowing them to withdraw their personal data from any controller's database.
    \item To assure that a controller cannot give the individual ``useless'' data, that the data must be portable and readable, and so that users cannot be locked down in a ``data monopoly'' to the controller.
\end{enumerate}

These rights guarantee that everyone, independent of nationality, condition, or faith, have thorough rights that protect their privacy in the digital era, and for the eras that are still to come.
\end{mla}
\end{document}

