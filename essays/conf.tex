\documentclass[12pt,letterpaper]{article}
\usepackage[utf8]{inputenc}
\usepackage[margin=1in]{geometry}
\usepackage[notes,backend=biber]{biblatex-chicago}
\bibliography{drm}
\usepackage[hidelinks]{hyperref}
\urlstyle{same}
\usepackage{setspace}


\title{DRM Considered Harmful}
\author{Bernardo Meurer}
\date{April 12, 2018}

\begin{document}
\maketitle
\newpage
\doublespacing%
In the Digital Era, copyright has become harder to enforce than ever before. Digital works are subject to copying without any quality degradation, which encourages piracy and can lead to market failure.\autocite[\ppno~118]{favale-2014} While some optimists\autocite[\ppno~236]{goldstein-1996} claim that digital rights management (DRM) is a successful technology, protecting copyrights in the digital world, and safeguarding markets and creators alike, the truth is quite the opposite. Since current implementations of DRM unjustly and broadly harden access to copyrighted works, they represent a disconnect with the fundamental concepts of copyright, breaking with the Lockean principles for appropriation which are the bedrock for property rights. DRM should be an instrument to defend copyright, not hinder it, and therefore it should either be implemented such that it respects the rights of users and owners, or not at all.

Intellectual works have always been bound to a physical medium. The Digital Era began when this changed, and copyrighted works transitioned into the virtual, making it possible for them to be copied freely and without degradation, which represents a grave threat to the rights of creators and to market stability\autocite[\ppno~117]{favale-2014}. In response to this threat, and in an attempt to halt unwarranted copying of copyrighted material, copyright holders created DRM.\@ Digital Rights Management aims to guarantee that certain digital content can only be used by someone insofar as the copyright owner allows them to. Practically, this means that DRM attempts to guarantee that the video being streamed from Netflix goes into your screen and subsequently to your eyes, and not to some form of recording device. Another practical example of DRM is the Content Scrambling System (CSS) found in DVDs, which stops users from copying the content of the disc into their computers. Here an issue already becomes evident due to the fact that, despite one having purchased and by all means owning the DVD, one cannot do with as they please.

Enforcing DRM, however, has been shown to be fundamentally infeasible, since there is no way for software alone to determine whether or not a camera is being pointed at the screen in order to copy the media on display. Furthermore, even the more feasible attempts, such as ensuring the output device is a bona-fide screen and not a capture device, have failed. Several companies and standards' bodies have proposed and implemented a plethora of DRM schemes, none of which stood the test of time, usually getting cracked within a year of release. DRM's ultimate failure manifests itself in the fact that not one single movie, song, or video game released in the past decade that reached a minimal degree of success is unavailable on the web free of charge. It is safe to say that if the entertainment industry, with all its resources and manpower, could not come up with a reliable implementation of DRM after over 20 years of constant development, then DRM, at least as intended by them, is not achievable.

According to John Locke, ``The Labor of one's Body and the Work of his Hands, we may say, are properly his.''\autocite[\ppno~287]{locke-1988} From this notion, that one appropriates something from the common by applying labor onto it, common law established the concept of property and the rights that come with it. Lockean property rights, however, don't protect just the laborer, be it an author or a hunter, but also the remaining public. While Locke grants the owner, and him alone, the rights of
use, transfer, and exclusion of his property, it also gives the public rights to the remaining common so that the community can thrive.\autocite[\ppno~119]{favale-2014} This falls from the idea that, according to Locke, appropriation is only valid insofar as it leaves ``enough, and as good''\autocite[\ppno~291]{locke-1988} of the common to the community. As an example of this principle, we can imagine a gatherer who is nourished by the apples he picked in the woods. Nobody can deny that the
apples are his, and yet when did they become his? According to Locke, if the first gathering did not make them the man's property, then nothing could; labor put a distinction between them (the apples) and the common.\autocite[\ppno~288]{locke-1988}

Intellectual property has a peculiar characteristic, unlike material property; it lacks extension (it is physically inexhaustible). Because of this, in order to apply Lockean principles to intellectual property, and show copying as being analogue to stealing, we must appeal to Hegel. In his work ``Elements of the Philosophy of Right,'' Hegel will argue that property is fundamentally more than the result of one's labor; it is rather the expression of Man's personality\autocite[\ppno~84--88]{hegel-1991}. It is unquestionable that when one composes a song or writes a book, he is expressing his personality; not only in the will to do the work (as would be the case of the gatherer) but also in the resulting work itself. This is to say that when Shakespeare writes a piece, his personality is expressed not only in the exercise of his will while writing, but also in the piece itself. With this, and drawing a parallel between Lockean and Hegelian philosophy, one can firmly state that intellectual property is just another form of property, since it's an expression of its author's personality through and through.

As Wendy Gordon shows, copying, even though it doesn't deprive the owner of his creation, can harm him. Take, for example, a musician who composes not only for the sake of her own listening and pleasure, but also with the purpose of feeding herself by means of the royalties to her work. A bootleg copyist will undoubtedly harm her just as severely as if he had taken the physical sheet music from her, or stole the food she had bought.\autocite[\ppno~1548]{gordon-1993} Therefore, in the same way, one could not eat of the gatherer's apples, for they were his property, one also cannot copy from others' intellectual creations, for they are also solely theirs. From this follows that there must be some form of anti-copying protection applied to intellectual property in order to safeguard the author's rights. Said protection is what modern law calls copyright, and it is that which DRM attempts to enforce on the digital realm.

Western society is fundamentally based on strong property rights; the concept of appropriation and the enforcement of said rights are the basis of modern capitalism. Property rights grant the holder a range of privileges (transfer, use, and so on), but it also grants the public a set of complementary rights, namely some protection from a malicious party, as a rights holder can only constrain others from using his property insofar as that does not impair a community's ability to
thrive\autocite[\ppno~1555--1560]{gordon-1993}. Hence an individual's property should not harm the public's access to the common. Moreover, in Lockean theory the interests of the community must prevail over those of the individual when they clash\autocite[\ppno~120]{favale-2014}, this reveals something fundamental, and often overlooked, about copyright, as Neil Netanel points out, ``while copyright may operate in the market, copyright's fundamental goals are not of the market.''\autocite[\ppno~341]{netanel-2001} Copyright, therefore, has goals which go beyond the market itself; in other words, it hopes to achieve things other than market safety and profit protection. Natanel goes on to claim that copyright is a vehicle of democracy, pointing once more towards interpreting copyright as protecting the public first, and the author second. Furthermore, the first copyright legislation, the Copyright Act 1710, had as its subtitle ``An Act for the Encouragement of Learning \ldots''\autocite{wortley-1710}, meaning that since its inception copyright had the public's interest in mind primarily. With this it becomes clear that copyright goes far beyond both markets and authors, enlightening the fundamental flaw of DRM, which is its ignorance towards the rights of the remaining public.

As previously outlined, Lockean philosophy constrains one's property to leave ``enough and as good''\autocite[\ppno~291]{locke-1988} to the public. While here, ``enough and as good'' is referring to the common, by which Locke meant that which Nature provides us, this concept can easily be generalized to apply to the immaterial as well. Fundamentally, this means that in the same way there is a common of things, given to us by Nature, there is also a common of intangibles, ideas, that every individual must be able to use freely. As Gordon points out, ``Our common encompasses not only our physical country but our culture as well''\autocite[\ppno~1555]{gordon-1993}. An example of a non-physical common of ideas is public domain, which is composed of works that never were, or that no longer are, protected by copyright. It follows from this that no appropriation can be made of that which is in the common of ideas, and therefore no copyright can be claimed. Consider, for example, the happy birthday song. It is absurd to imagine that someone today could rightfully claim ownership of the melody and proceed to receive royalties from its use; the song is an integral part of American culture after all. The reason why no royalties are applied to it, or to any number of older works, is because they belong to the public domain (the common of ideas), since they either were never copyrighted, or have had their rights expire.

Since the fundamental goal of copyright law lies beyond the market, being that it should encourage learning and democratic discourse, it must always protect the common, of resources and of ideas, from wrongful appropriation. It is here, in the public's entitlement to the common, that DRM falls short. Being that the protection schemes have no expiration date, the material can't be accessible to the public even after the copyright has expired, unless the author re-releases it stripped of DRM.\@ Moreover, given that DRM acts on the display of content, it blocks even legitimate uses of the material. Take the fair use doctrine, for example, a practice of the legislature to allow use of copyrighted works, despite the lack of permission from the owner, insofar as the usage satisfies some criteria, namely being transformative in nature, and not being a direct market competitor to the original. It is clear that in order for fair use to happen the user must have access to the work. If, however, said work is protected by DRM, then no access to the content can happen without authorization from the owner; hence no fair use can happen on the content, breaking with the doctrine. In practice, this means that copyright holders are effectively ignoring the legislature's interpretation of Copyright, and shaping the system in detriment of the public. It is thus clear that DRM infringes on the public common, and that therefore it infringes on copyright itself, even if its intention was to safeguard it.

The solution to the problem of copyright enforcement in the digital realm is, much like the problem itself, multifaceted. Firstly, public policy should encourage competition among a variety of DRM schemes, while also enforcing interoperability between them. It is a well know fact that competition between products is beneficial to the consumer, and the same will be true of DRM, competition among standards will benefit the public with schemes that serve the common interest. Secondly, the
government must actively police and monitor the use and implementation of DRM on the part of companies, changing policies as necessary to protect the public's entitlements. DRM systems should be a tool to reinforce the limitations imposed by copyright, not create new ones. Moreover, DRM should never interfere with legitimate research or any matters of public concern. Simply put, laws such as the Digital Millennium Copyright Act (DMCA), who include anti-circumvention clauses, must be kept from
damaging research and public discourse, and this protection must be guaranteed by law. Finally, and perhaps most importantly, any policy with the intent of reinforcing copyright should be limited to use cases where copyright interests are actually at stake\autocite{felten-2005}. The crucial thing is to constrain DRM in such a way that it cannot legally infringe on the common, regardless of whether or not the owner stands to lose with this; the common must always prevail over the rights
of the individual.

The dawn of the Digital Era did not change the rationale of copyright law. As shown, both Lockean and Hegelian philosophies agree that copyright has above all a social function, a duty towards the public, outweighing the rights it gives the author. The immaterial attribute of intellectual property has caused an increase in free-riding behavior, a reaping by who has not sown, in its attempt to solve this issue, DRM trumps public entitlements and shifts protection in favor of the rights holder. DRM in its current state is fundamentally anti-copyright, despite its intent, since its interest and value are contained in the market, for it aims only to protect the owner's profits, unlike the copyrights which it claims to enforce. DRM should not exist legally, for it infringes on the fundamental property rights our society is based on, and takes away from the public's access to the common of intangibles. As said by Salman Rushdie, ``those who do not have power over the story that dominates their lives, power to retell it, rethink it, deconstruct it, joke about it, and change it as times change, truly are powerless, because they cannot think new thoughts,''\autocite{rushdie-1991} and no technology, no matter how lucrative, should take this power away from us, the people.

\newpage
\singlespace
\printbibliography[heading=bibintoc,title={Bibliography}]
\end{document}
