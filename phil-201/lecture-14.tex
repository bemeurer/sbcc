\documentclass{article}
    %%%%%%%%%%%%%%%%%%%%%%%%%%%%%%%%%%%%%%%%%%%%%%%%%%%%%%%%%%%%%
    % Lecture Specific Information to Fill Out
    %%%%%%%%%%%%%%%%%%%%%%%%%%%%%%%%%%%%%%%%%%%%%%%%%%%%%%%%%%%%%
    \newcommand{\LectureTitle}{Lecture \#14 Notes}
    %\newcommand{\LectureDate}{\today}
    \newcommand{\LectureDate}{November\ 08,\ 2017}
    \newcommand{\LectureClassName}{PHIL\ 201}
    \newcommand{\LatexerName}{Bernardo\ Meurer}
    %%%%%%%%%%%%%%%%%%%%%%%%%%%%%%%%%%%%%%%%%%%%%%%%%%%%%%%%%%%%%

    % Change "article" to "report" to get rid of page number on title page

    \usepackage[utf8]{inputenc}
    \usepackage{amsmath,amsfonts,amsthm,amssymb}
    \usepackage{setspace}
    \usepackage{Tabbing}
    \usepackage{fancyhdr}
    \usepackage{lastpage}
    \usepackage{extramarks}
    \usepackage{chngpage}
    \usepackage{soul,color}
    \usepackage{graphicx,float,wrapfig}
    \usepackage{afterpage}
    \usepackage{abstract}

    % In case you need to adjust margins:
    \topmargin=-0.45in
    \evensidemargin=0in
    \oddsidemargin=0in
    \textwidth=6.5in
    \textheight=9.0in
    \headsep=0.25in

    % Setup the header and footer
    \pagestyle{fancy}
    \lhead{\LatexerName}
    \chead{\LectureClassName: \LectureTitle}
    \rhead{\LectureDate}
    \lfoot{\lastxmark}
    \cfoot{}
    \rfoot{Page\ \thepage\ of\ \pageref{LastPage}}
    \renewcommand\headrulewidth{0.4pt}
    \renewcommand\footrulewidth{0.4pt}

    %%%%%%%%%%%%%%%%%%%%%%%%%%%%%%%%%%%%%%%%%%%%%%%%%%%%%%%%%%%%%

    \begin{document}
    \begin{spacing}{1.2}
    \newpage
        \section*{John Locke}
        \begin{flushright}
            1632 -- 1704
        \end{flushright}
        The philosophers we looked at so far (Descartes, Spinoza, Conway, Leibniz) are all Rationalists. The philosophers we will se now on are all Empiricists (Locke, Berkeley, Hume.) The main difference is that for rationalists at least \emph{some} of our ideas are innate. Descartes, for example, argues that the idea of God is innate. For Empiricists there are no innate ideas -- all of our ideas ultimately derive from sense-experience.

        Locke challenges innatism, the view that humans come ``hardwired'' with certain ideas: propositional \& non-propositional. For example, ``Whatever is, is'' is a propositional idea, ``God Exists'', ``A is A'',and ``2+2 = 4'' are all examples of propositional ideas as well. A non-propositional idea is, for example ``the idea of existence'', or ``the idea of God,'' these clearly are not statements (in a Leibnizian sense) and have no truth or falsity to them.

        Basic strategy: If we can show that humans can come to all the knowledge that we currently posses without the help of innate ideas, this would be enough to show that innatism is false.

        (Check Robert Nozick out)

        ``To say that a principle is innate is to say that it is true and that we are \emph{capable} of assenting to it.''
        Locke Replies:
        ``This dilutes the doctrine of innate ideas so much that an empiricist would probably agree! I have no problem with the doctrine of innate capacity.''
        Leibniz contests: ``To say that a principle is innate is to say that it's true and that the mind has a natural disposition or tendency to formulate it.''

        ``All ideas come from sensation or reflection''

        Sensation: The various ways external objects affect the senses by transmitting or communicating to the mind perceptions of those objects.

        Reflection: When the mind reflects on and considers its perceptions, the mind gains ``another ser of ideas'' --- this occurs when we think, doubt, believe, reason, know, will, etc.

        Locke seems to think we cannot acquire ideas via the sense unless we know that we are sensing. The mind ``must necessarily be conscious  of its own perceptions.'' The mind is forced to watch not only its acquisition of ideas of reflection, but also the acquisition of its sense-perceptions. In other words, all perceptions are apperceptions.

    \end{spacing}
    \end{document}