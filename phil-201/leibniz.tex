%cSpell:words fancyhdr extramarks amsthm amsfonts algpseudocode usetikzlibrary hmwk Bobro Meurer topmargin evensidemargin oddsidemargin headsep linespread firstxmark lastxmark headrulewidth footrulewidth nobreak stepcounter setcounter secnumdepth newcounter Alph pagebreak megagon syllog

\documentclass{article}

    \usepackage[utf8]{inputenc}
    \usepackage{fancyhdr}
    \usepackage[margin=1in]{geometry}
    \usepackage{extramarks}
    \usepackage{amsmath}
    \usepackage{amsthm}
    \usepackage{amsfonts}
    \usepackage{syllogism}

    %
    % Homework Details
    %   - Title
    %   - Due date
    %   - Class
    %   - Section/Time
    %   - Instructor
    %   - Author
    %

    \newcommand{\hmwkTitle}{On Leibniz's Truth}
    \newcommand{\hmwkDueDate}{October 23, 2017}
    \newcommand{\hmwkClass}{Philosophy 201}
    \newcommand{\hmwkClassTime}{}
    \newcommand{\hmwkClassInstructor}{Professor Marc Bobro}
    \newcommand{\hmwkAuthorName}{\textbf{Bernardo Meurer}}


    %
    % Basic Document Settings
    %

    \topmargin=-0.45in
    \evensidemargin=0in
    \oddsidemargin=0in
    \textwidth=6.5in
    \textheight=9.0in
    \headsep=0.25in

    \linespread{1.2}

    \pagestyle{fancy}
    \lhead{\hmwkAuthorName}
    \chead{\thepage}
    \rhead{\hmwkDueDate}
    \lfoot{}
    \cfoot{}

    \renewcommand\headrulewidth{0.4pt}
    \renewcommand\footrulewidth{0.4pt}

    \setlength\voffset{-0.15in}

    %
    % Title Page
    %

    \title{
        \textbf{\hmwkTitle}\\
        \normalsize\vspace{0.1in}\small{\hmwkDueDate}
        \author{\hmwkAuthorName}
    }
    \date{}

\begin{document}
\maketitle
\pagebreak
We begin by defining a statement as a sentence composed of a subject and a predicate. The sentences ``Washington crossed the Delaware'', and ``Blue is blue,'' for example, qualify as statements. In the first, ``Washington'' is the subject, and ``crossed the Delaware'' the predicate. Here it is important to note that the predicate does not require an object for the sentence which embodies it to be a statement, take for example the sentence ``A is.'' It is still a statement (regarding A's existence) albeit it need not have A as it's object. Statements are said to be \emph{true} when the concept of the predicate is contained in the concept of the subject (Leibniz's Concept-Containment Theory of Truth). This brings us to \emph{identity statements}, which are of type

    \begin{center}
        ``X is X''
    \end{center}

\noindent These are axiomatically (and in fact self-evidently) true. To make any arguments one must first have a working system of logic, and since the law of identity (``things are what they are'') is a fundamental axiom in all logic systems, it follows that without it no arguments could be made. Without identity there can be no logic, as far as human understanding goes, so we must take it as being true in order to produce anything meaningful. In Leibniz's language this translates to the predicate-concept and the subject-concept being the same. These statements are called \emph{Primary Truths}.

Let's now take a closer look again at the first example of a statement given,
    \begin{center}
        ``Washington crossed the Delaware''
    \end{center}
\noindent Here, unlike in the identity statements, the predicate-concept is \emph{not} the same as the subject-concept, and most importantly the concept of predicate is not obviously contained in the concept of the subject. Proof of this is that one can very easily imagine Washington as \emph{not} having crossed the Delaware. While the identity statements seemed to be ``self-contained'', independent of any external factors, this statement seems wholly dependent on a large set of externalities. For Washington to have crossed the Delaware, the Delaware must exist, and so must Washington, and the Earth on which which they interact, and so on. It seems, therefore, that the statement could very well be false, since there is no obvious ``concept-containment.'' Statements like this are called \emph{Secondary Truths}. One can, however, show a non-identitarian statement to be True, and to do so we must reduce it to a primary truth. Let us look at ``Washington crossed the Delaware'' once more. Washington here is more than a person, but fundamentally a concept, most importantly the concept of Washington is not atomic, but can be divided into a set of sub-concepts that compose it. We can write this as
    \[C = \{\alpha_1, \alpha_2, \ldots, \alpha_n\}\]
Denoting a concept as being composed of $n$ characteristics such that it is unique. This is to say that there can be no such \(C' = \{\alpha_1, \alpha_2, \ldots, \alpha_n\}\) for if that were to happen it would follow that \(C \subset C'\) and \(C' \subset C\) which trivially implies \(C = C'\). This is to say that every concept is uniquely defined by the union of it's parts, and if two concepts share exactly the same parts then they are in fact the same concept. Applying this to our statement we can write that there exists a concept \(W\) of Washington, composed of $n$ $\alpha$-elements, one of which takes form \(\alpha_k = \text{crossed the Delaware}\) for \(1 < k < n\). This is to say
    \[W = \{\alpha_1, \alpha_2, \ldots, \alpha_k, \ldots, \alpha_n\}\]
And fundamentally that ``Washington is Washington'' if and only if
    \[W = \bigcup_{i=1}^{n} \alpha_i\]
Let us now suppose that Washington did \emph{not} cross the Delaware. This means we now have a concept \(W' = \{\beta_1, \beta_2, \ldots, \beta_n\}\) where
    \[(\forall \alpha_i\in W\setminus \{\alpha_k\})(\exists \beta_j\in W') \text{ s.t. } \alpha_i=\beta_j\]
That is, that \(W'\) contains all elements of \(W\) \emph{except} \(\alpha_k\). This, however, means that while \(W' \subset W\), the inverse is not true and \(W \not\subset W'\). Since we had previously defined that the only way for two concepts to be the same (namely \(C = C'\)) is for \(C \subset C' \land C' \subset C\), which here we showed to not be true, we conclude that
    \[W \neq W'\]
Fundamentally this means that our idea of a Washington that did not cross the Delaware is in fact \emph{not an idea of Washington at all}. It is a separate concept, without equivalence. Moreover we can see that there is a set of concepts
    \[\Omega = \mathcal P(W)\setminus W\]
Which constitutes all the ``lesser forms'' of Washington, meaning concepts which are proper subsets of the main concept \(W\). This shows us something curious, which is that every concept that exists is the largest (in terms of cardinality) that it could be. In other words, concept is defined by it's set of characteristics, any proper subset of which defines something which is not that concept. Formally:
    \[(\forall \omega \in \Omega)(|\omega| < |W|)\]
Finally, the fact that it is trivially true that ``Washington is Washington,'' means that any change in the composition of Washington (\(W\)) yields a new concept which is not the original, and therefore statements are only true insofar as their predicate is part of the composition of the subject, for otherwise we violate the law of identity. In other words, the statement \(C\gamma\) (where \(C\) is a subject and \(\gamma\) is a predicate) is true iff \(\gamma\in C\).


\end{document}
