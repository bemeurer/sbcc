\documentclass{article}
    %%%%%%%%%%%%%%%%%%%%%%%%%%%%%%%%%%%%%%%%%%%%%%%%%%%%%%%%%%%%%
    % Lecture Specific Information to Fill Out
    %%%%%%%%%%%%%%%%%%%%%%%%%%%%%%%%%%%%%%%%%%%%%%%%%%%%%%%%%%%%%
    \newcommand{\LectureTitle}{Lecture \#3 Notes}
    %\newcommand{\LectureDate}{\today}
    \newcommand{\LectureDate}{August\ 30,\ 2017}
    \newcommand{\LectureClassName}{PHIL\ 201}
    \newcommand{\LatexerName}{Bernardo\ Meurer}
    %%%%%%%%%%%%%%%%%%%%%%%%%%%%%%%%%%%%%%%%%%%%%%%%%%%%%%%%%%%%%
    
    % Change "article" to "report" to get rid of page number on title page
    
    \usepackage[utf8]{inputenc}
    \usepackage{amsmath,amsfonts,amsthm,amssymb}
    \usepackage{setspace}
    \usepackage{Tabbing}
    \usepackage{fancyhdr}
    \usepackage{lastpage}
    \usepackage{extramarks}
    \usepackage{chngpage}
    \usepackage{soul,color}
    \usepackage{graphicx,float,wrapfig}
    \usepackage{afterpage}
    \usepackage{abstract}
    
    % In case you need to adjust margins:
    \topmargin=-0.45in
    \evensidemargin=0in
    \oddsidemargin=0in
    \textwidth=6.5in
    \textheight=9.0in
    \headsep=0.25in
    
    % Setup the header and footer
    \pagestyle{fancy}
    \lhead{\LatexerName}
    \chead{\LectureClassName: \LectureTitle}
    \rhead{\LectureDate}
    \lfoot{\lastxmark}
    \cfoot{}
    \rfoot{Page\ \thepage\ of\ \pageref{LastPage}}
    \renewcommand\headrulewidth{0.4pt}
    \renewcommand\footrulewidth{0.4pt}
    
    %%%%%%%%%%%%%%%%%%%%%%%%%%%%%%%%%%%%%%%%%%%%%%%%%%%%%%%%%%%%%
    % Some tools
    \newcommand{\enterTopicHeader}[1]{\nobreak\extramarks{#1}{#1 continued on next page\ldots}\nobreak%
                                        \nobreak\extramarks{#1 (continued)}{#1 continued on next page\ldots}\nobreak}
    \newcommand{\exitTopicHeader}[1]{\nobreak\extramarks{#1 (continued)}{#1 continued on next page\ldots}\nobreak%
                                       \nobreak\extramarks{#1}{}\nobreak}
    
    \newlength{\labelLength}
    \newcommand{\labelAnswer}[2]
      {\settowidth{\labelLength}{#1}
       \addtolength{\labelLength}{0.25in}
       \changetext{}{-\labelLength}{}{}{}
       \noindent\fbox{\begin{minipage}[c]{\columnwidth}#2\end{minipage}}
       \marginpar{\fbox{#1}}
    
       % We put the blank space above in order to make sure this
       % \marginpar gets correctly placed.
       \changetext{}{+\labelLength}{}{}{}}
    
    \setcounter{secnumdepth}{0}
    \newcommand{\TopicName}{}
    \newcounter{TopicCounter}
    \newenvironment{Topic}[1][Problem \arabic{TopicCounter}]
      {\stepcounter{TopicCounter}
       \renewcommand{\TopicName}{#1}
       \section{\TopicName}
       \enterTopicHeader{\TopicName}}
      {\exitTopicHeader{\TopicName}}
    
    \setcounter{secnumdepth}{0}
    \newcommand{\ExampleSectionName}{}
    \newcounter{ExampleSectionCounter}[TopicCounter]
    \newenvironment{ExampleSection}[1][Example \arabic{ExampleSectionCounter}]
      {\stepcounter{ExampleSectionCounter}
       \renewcommand{\ExampleSectionName}{#1}
       \section{\ExampleSectionName}
       \enterTopicHeader{\ExampleSectionName}}
      {\exitTopicHeader{\ExampleSectionName}}
    
    \setcounter{secnumdepth}{0}
    \newcounter{ExampleBoxCounter}[TopicCounter]
    \newcommand{\examplebox}[1]
      {
      % We put this space here to make sure we're disconnected from the previous
       % passage
       \stepcounter{ExampleBoxCounter}
       \noindent\fbox{\begin{minipage}[c]{\columnwidth}#1\end{minipage}}\enterTopicHeader{\ExampleSectionName}\exitTopicHeader{\ExampleSectionName}\marginpar{\fbox{\#\arabic{ExampleBoxCounter}}}
       % We put the blank space above in order to make sure this
       % \marginpar gets correctly placed.
       \vskip10pt%
       }
    
    \renewcommand{\contentsname}{{\normalsize Topics Covered}}
    \renewcommand{\abstractname}{\LectureTitle\ Summary}
    \renewcommand{\absnamepos}{flushleft}
    
    %%%%%%%%%%%%%%%%%%%%%%%%%%%%%%%%%%%%%%%%%%%%%%%%%%%%%%%%%%%%%
    
    \begin{document}
    \begin{spacing}{1.2}
    \newpage
    René Descartes was born 1596 in a town called La Haye en Touraine
    (now known as Descartes). He wen to a Jesuit school and was trained to
    become a lawyer. However, instead of going into law Descartes joins the
    military, and it is there that he ``envisions'' his philosophy. In the 
    military Descartes decides to ``rebuild the philosophy from it's 
    foundational principles in order to ground the sciences''. Here,
    when Descartes says ``sciences'' Philosophy, and Maths are included;
    it is a more general interpretation of the term than a modern one.

    Descartes feared persecution in France, so he left. Descartes lived in 
    the Netherlands thereon after, and there wrote most of his works. In 
    his lifetime Descartes was a true generalist, writing philosophy (e.g. 
    Meditations), working on mathematics (Cartesian plane, and many other woks)
    he also worked on some vivisection and a varied range of scientific topics.

    René died in 1649 at age 53, in Sweden, where he was at the invitation of
    the Queen. Some say his death was due to the early sessions of
    philosophical discussions, Descartes had always been a late riser. Also, 
    in part, the Swedish winter did not aid him in his advanced age.

    There are six Meditations written by Descartes, they are modelled after
    St.Ignatius' Spiritual Exercises. You're meant to read ``one per day'',
    you read it, and then you contemplate on it. He wrote in many different
    styles, sometimes dialogue, sometimes copying Euclid's Geometry, using 
    very clear, very ``step-by-step'', meticulous argumentation. Another 
    interesting thing about the Meditations is that they are written in an 
    auto-biographical way, they are, in a sense, a report on how he came to
    his conclusions. It's important to note that Descartes took a lot of
    liberty with this; we do not known that he came to his conclusions as 
    described.

    An interesting dimension of the Meditations is that Descartes wants to
    make it exceedingly clear that his philosophy is compatible with 
    Christianity, and the Church. Descartes is not an Atheist, he truly 
    believed his philosophy was compatible with Catholicism. 

    ``Several years have now passed since I first realized how numerous were
    the false opinions that in my youth I had taken to be true, and thus how
    doubtful were all those that I had subsequently built upon them.''%
\footnotemark[1]%
    --- In a letter, Descartes later exemplifies one such ``false opinion'', he
    talks about how he believed that ``the soul was a body''; that it was some
    form of spirit in the body.

    ``And thus I realized that once in my life I had to raze everything to the
    ground and begin again from the original foundations, if I wanted to establish
    anything firm and lasting in the sciences.''%
\footnotemark[1]%
    --- What Descartes is getting at is that if there is \emph{any root for doubt}
    in an idea, it cannot be used as a foundation. He's seeking \emph{certainty}
    in the sciences, new knowledge must rest on beliefs that cannot rationally
    be doubted. 
    
    \begin{center}
        \(\text{Belief}\hspace{5pt}\underset{\text{certainty}}{\xrightarrow{%
        \text{Indubitability}}}\hspace{5pt}\text{Knowledge}\)
    \end{center}

    Descartes wants to find beliefs that cannot be doubted, the problem is that
    he has far too many beliefs to go over them one-by-one. Because of this, 
    he divides his beliefs into two classes: beliefs based on the senses 
    (\emph{a posteriori}), and beliefs based on pure reason (\emph{a priori}).
    He argues that if you've been deceived by your senses once, then you can be
    deceived again, therefore they \emph{cannot} provide certainty; only pure
    reason could truly be certain.




    \footnotetext[1]{Meditations I}
    \end{spacing}
    \end{document}