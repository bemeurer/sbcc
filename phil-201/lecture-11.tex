\documentclass{article}
    %%%%%%%%%%%%%%%%%%%%%%%%%%%%%%%%%%%%%%%%%%%%%%%%%%%%%%%%%%%%%
    % Lecture Specific Information to Fill Out
    %%%%%%%%%%%%%%%%%%%%%%%%%%%%%%%%%%%%%%%%%%%%%%%%%%%%%%%%%%%%%
    \newcommand{\LectureTitle}{Lecture \#11 Notes}
    %\newcommand{\LectureDate}{\today}
    \newcommand{\LectureDate}{October\ 2,\ 2017}
    \newcommand{\LectureClassName}{PHIL\ 201}
    \newcommand{\LatexerName}{Bernardo\ Meurer}
    %%%%%%%%%%%%%%%%%%%%%%%%%%%%%%%%%%%%%%%%%%%%%%%%%%%%%%%%%%%%%
    
    % Change "article" to "report" to get rid of page number on title page
    
    \usepackage[utf8]{inputenc}
    \usepackage{amsmath,amsfonts,amsthm,amssymb}
    \usepackage{setspace}
    \usepackage{Tabbing}
    \usepackage{fancyhdr}
    \usepackage{lastpage}
    \usepackage{extramarks}
    \usepackage{chngpage}
    \usepackage{soul,color}
    \usepackage{graphicx,float,wrapfig}
    \usepackage{afterpage}
    \usepackage{abstract}
    
    % In case you need to adjust margins:
    \topmargin=-0.45in
    \evensidemargin=0in
    \oddsidemargin=0in
    \textwidth=6.5in
    \textheight=9.0in
    \headsep=0.25in
    
    % Setup the header and footer
    \pagestyle{fancy}
    \lhead{\LatexerName}
    \chead{\LectureClassName: \LectureTitle}
    \rhead{\LectureDate}
    \lfoot{\lastxmark}
    \cfoot{}
    \rfoot{Page\ \thepage\ of\ \pageref{LastPage}}
    \renewcommand\headrulewidth{0.4pt}
    \renewcommand\footrulewidth{0.4pt}
    
    %%%%%%%%%%%%%%%%%%%%%%%%%%%%%%%%%%%%%%%%%%%%%%%%%%%%%%%%%%%%%
    
    \begin{document}
    \begin{spacing}{1.2}
    \newpage
        God has an infinite number of infinite attributes.
        We, as humans, can come up with two attributes that we can conceive: thought and extension.

        From this, Spinoza draws three conclusions: 
        \begin{enumerate}
            \item ``There cannot exist two substances without their differing entirely in essence''
            \item ``Since to exist is part of its essence, a substance cannot be produced''
            \item ``Each substance must be infinite, that, is, most perfect''
        \end{enumerate}
        Substances are truly self-sufficient, fully independent, and from this it follows that there is only one substance (since each must be infinite).

        ``Nature is not intrinsically beautiful, ugly, orderly, or confused. It is only with respect to our imagination that things can be told to be beautiful or orderly.''

        \subsection*{The Ethics}
        \begin{flushright}
            1677
        \end{flushright}
        Follows the ``order of logic'' rather than the ``order of discovery'', unlike Descartes' Meditations. Mimics Euclid's ``Elements of Geometry''.

        Spinoza's definitions are \emph{not} arbitrarily chosen, nor are they intended to mirror ordinary usage --- that is, they are \emph{not} dictionary definitions. He begins by defining eight things:
        \begin{itemize}
            \item Self-caused thing
            \begin{itemize}
                \item That whose essence involves existence or that whose nature can be conceived only as existing. Here what is meant is that it never \emph{came} into existence; it has always existed.
            \end{itemize}
            \item Finite thing
            \item Substance
            \item Attribute
            \begin{itemize}
                \item That which the intellect perceives of substance as constituting its essence. 
            \end{itemize}
            \item Mode
            \item God
            \item Free Thing
            \item Eternity
            \begin{itemize}
                \item Eternity \(\neq\) Sempiternity\footnote{Always living}. When Spinoza speaks of eternity he means absolute limitlessness.
            \end{itemize}
        \end{itemize}
        \newpage
        \subsection*{Propositions}
        \begin{enumerate}
            \item ``Substance is by nature prior to it's affections.''
            \begin{itemize}
                \item The modes of substance can only be understood through substance and not the other way around.
                \item Here \emph{mode} and \emph{modification} can be used interchangeably.
            \end{itemize}
            \item ``Two substances having different attributes have nothing in common.''
            \item ``When things have nothing in common one cannot be the cause of the other''
            \begin{itemize}
                \item For something to be the effect of other, there must be some likeness between the two. 
                \item i.e. Cartesian dualism is bogus.
            \end{itemize}
            \item ``Two or more distinct things are distinguished from one another either by ...''
            \begin{itemize}
                \item It is by a difference in attribute or in modes that things are distinguished from each other. 
            \end{itemize}
            \item ``In the same substance there cannot be two or more substances that have the same attribute.''
            \item ``One substance cannot be produced by another substance. ''
            \item ``Existence belongs to the essence\footnote{nature} of substance.''
            \begin{enumerate}
                \item Substances cannot causally interact. (Prop. 6)
                \item Everything must have a cause (Axiom 3)
                \\Therefore
                \item Substances must be self-caused
                \\Therefore
                \item At least one substance must exist (Definition 1)
            \end{enumerate}
            \item ``Each substance is necessarily infinite''
            \item ``The more real or perfect a thing is, the more attributes it has.''
            \item ``Each attribute of a substance must be conceived through itself''
            \item ``God (substance consisting of infinite attributes) necessarily exists.''
            \begin{enumerate}
                \item Try to conceive of God not existing
            \end{enumerate}
            \item ``There is no attribute of substance that renders substance capable of division''
            \begin{enumerate}
                \item Strictly speaking a substance cannot have parts.
            \end{enumerate}
        \end{enumerate}

        So, to reiterate, the crucial points when understanding Spinoza are
        \begin{itemize}
            \item Substance
            \item Attribute
            \item God (Substance with infinite attributes)
            \item Modes
            \begin{itemize}
                \item Infinite -- A permanent and general feature of an entire attribute (e.g. The laws of physics, the laws of logic)
                \item Finite -- A transitory and specific, determinant manifestation of an infinite mode.
            \end{itemize}
        \end{itemize}
        Each attribute can be thought of as a fundamental \underline{way of being}.Each is a different \emph{aspect} of the same reality.
        \subsection*{Appendix}
        Aristotle's four ``causes''\footnote{explanations}
        \begin{itemize}
            \item Material
            \item Formal
            \item Efficient
            \item Final
            \begin{itemize}
                \item For Spinoza, there is no divine final or teleological causation.
            \end{itemize}
        \end{itemize}
    \end{spacing}
    \end{document}